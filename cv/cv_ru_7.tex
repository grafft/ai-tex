\documentclass[11pt,a4paper,sans]{moderncv}
\moderncvstyle{casual}
\moderncvcolor{blue}

\usepackage[scale=0.75]{geometry}

\usepackage{cmap}                       % Поддержка поиска русских слов в PDF (pdflatex)
\usepackage[utf8]{inputenc} % utf8 encoding
\usepackage[T2A]{fontenc}
\usepackage{lmodern}
\usepackage[english,russian]{babel}

%\usepackage{fontspec}
%\usepackage{polyglossia}
%\setdefaultlanguage{english}
%\setmainlanguage{russian}
%\setmainfont[Ligatures=TeX]{Calibri}
%\newfontfamily\cyrillicfont[Ligatures=TeX]{Calibri}
%\newfontfamily\cyrillicfontrm[Ligatures=TeX]{Calibri}
%\newfontfamily\cyrillicfontsf[Ligatures=TeX]{Calibri}
%\newfontfamily\cyrillicfonttt[Ligatures=TeX]{Calibri}
%----------------------------------------------------------------------------------------
%	NAME AND CONTACT INFORMATION SECTION
%----------------------------------------------------------------------------------------

\firstname{Александр} % Your first name
\familyname{Панов} % Your last name

% All information in this block is optional, comment out any lines you don't need
\title{Curriculum Vitae}
\address{117312, Москва}{пр-т 60-летия Октября, 9}
\mobile{+7 (916) 144 5255}
\phone{+7 (499) 137 5457}
%\fax{(000) 111 1113}
\email{pan@isa.ru,apanov@hse.ru}
\homepage{hse.ru/staff/apanov}{hse.ru/staff/apanov}
%\extrainfo{additional information}
\photo[100pt][0.4pt]{../images/mine/cv_photo_2} 
%\quote{"A witty and playful quotation" - John Smith}

\hyphenation{РФФИ}
\usepackage[
	language=auto,
	autolang=langname,
	defernumbers=true,
	backend=biber,	
	bibstyle=gost-numeric,
	sorting=ynt,
	maxbibnames=15
]{biblatex}

\addbibresource{cv_ru_7.bib}
\DeclareSourcemap{
	\maps[datatype=bibtex, overwrite]{
		\map{
			\step[fieldset=langid, fieldvalue=english]
			\step[fieldset=doi, null]
			\step[fieldset=issn, null]
			\step[fieldset=isbn, null]
			\step[fieldset=url, null]
			\step[fieldsource=language, fieldset=langid, origfieldval]
		}
	}
}

\begin{document}

\makecvtitle % Print the CV title

%----------------------------------------------------------------------------------------
%	EDUCATION SECTION
%----------------------------------------------------------------------------------------


\section{Образование}

\cventry{2011--2015}{Кандидат физико-математических наук по направлению <<05.13.17 – Теоретические основы информатики>>}{Институт системного анализа РАН}{}{Москва}{Тема диссертации <<Исследование методов, разработка моделей и алгоритмов формирования элементов знаковой картины мира субъекта деятельности>>, науч. руководитель – Г.\,С.~Осипов} 

\cventry{2009--2011}{Магистр прикладных математики и физики по направлению <<Прикладные математика и физика>>}{Московский физико-технический институт}{}{Москва}{Тема диссертации <<Исследование и моделирование поведения коллектива интеллектуальных агентов с различной функциональностью>>, науч. руководитель – Г.\,С.~Осипов}

\cventry{2005--2009}{Бакалавр физики по направлению <<Физика>>}{Новосибирский государственный университет}{Новосибирск}{}{}


\section{Опыт научно-педагогической работы}

\cventry{2011--по н.в.}{Доцент}{Московский физико-технический институт}{ФПМИ, базовая кафедра ФИЦ ИУ РАН}{Москва}{<<Основы операционных систем>> (семинары), <<Основы объектно-ориентированного программирования>> (семинары), <<Интеллектуальные системы управления в робототехнике>> (лекции, семинары)}
\cventry{2015--по н.в.}{Доцент}{Высшая школа экономики}{ФКН, базовая кафдера ФИЦ ИУ РАН}{Москва}{Майнор <<Интеллектуальные анализ данных>> (лекции, семинары).}
\cventry{2011--2016}{Ассистент}{Российский университет дружбы народов}{кафедра информационных технологий}{Москва}{<<Интеллектуальные динамические системы>> (лекции, семинары), <<Теоретические основы информатики>> (лекции, семинары), <<Интеллектуальный анализ данных>> (лекции, семинары).}  % Arguments not required can be left empty

\section{Опыт научной работы}

\cventry{2010--по н.в.}{Старший научный сотрудник}{\textsc{ФИЦ <<Информатика и управление>> РАН}}{лаборатория <<Динамические интеллектуальные системы>>}{Москва}{
	\begin{itemize}
		\item \textit{Компьютерное когнитивное моделирование}: исследование и моделирование процессов восприятия, планирования поведения, целеполагания и других высших когнитивных функций человека.
		\begin{itemize}
			\item Предложены модели некоторых когнитивных функций на основе знакового опосредования.
			\item Исследован процесс образования элементов картины мира субъекта деятельности (знаков).
			\item Предложены и исследованы модели компонент знака на основе нейрофизиологических данных.
			\item Разработаны алгоритмы распределения ролей в коалиции когнитивных агентов.
		\end{itemize}
		\item \textit{Машинное обучение}: разработка алгоритмов логического и гибридного методов анализа данных, разработка биологически правдоподобных алгоритмов машинного обучения.
		\begin{itemize}
			\item Разработан гибридный метод выявления причинно-следственных связей в массиве слабоструктурированной информации. 
			\item Предложен нейроморфный метод машинного обучения - гетерархическая каузальная сеть.
		\end{itemize}
		\item \textit{Многоагентные системы и системы управления}: исследование распределения ролей в коллективе агентов, разработка многоуровневых архитектур управления коллективами сложных технических объектов.
		\begin{itemize}
			\item Разработана многоуровневая система управления коллективом когнитивных робототехнических систем STRL. 
		\end{itemize}
	\end{itemize}
}

\cventry{2018--по н.в.}{Заместитель заведующего лабораторией}{\textsc{Московский физико-технический институт}}{лаборатория когнитивных динамических систем}{Москва}{
	\begin{itemize}
		\item \textit{Обучение с подкреплением}: разработка новых методов обучения с подкреплением для практических задач, в том числе робототехнических.
		\begin{itemize}
			\item Предложен новый метод иерархического обучения с подкреплением на основе иерархии абстрактных автоматов.
		\end{itemize}
	\end{itemize}	
} 

\cventry{2015--2018}{Научный сотрудник}{\textsc{Высшая школа экономики}}{лаборатория процессно-ориентированных информационных систем}{Москва}{
	\begin{itemize}
		\item \textit{Компьютерное когнитивное моделирование}: исследование методов обучению в задаче планирования поведения на основе знаковой картины мира.
	\end{itemize}	
}  % Arguments not required can be left empty

%----------------------------------------------------------------------------------------
%	GRANTS
%----------------------------------------------------------------------------------------

\section{Научные гранты}

\subsection{В качестве руководителя}

	\cventry{2018--2020}{Гранты для постдоков}{Российский научный фонд (РНФ)}{}{}{Иерархическое обучение с подкреплением в задаче приобретения концептуальных процедурных знаний когнитивными агентами.}
		
	\cventry{2016--2019}{Гранты ориентированных фундаментальных исследований}{Российский фонд фундаментальных исследований (РФФИ)}{}{}{Разработка новых методов формирования баз знаний, поиска и адаптации прецедентов о существующих научно-технических решениях и технологиях по их текстовым описаниям на основе теории семантических сетей.}
	
	\cventry{2016--2018}{Гранты для постдоков}{Российский фонд фундаментальных исследований (РФФИ)}{}{}{Исследование механизмов и построение моделей обучения, основанных на знаковых представлениях, в задаче планирования коллективного поведения.}
	
	%------------------------------------------------
	
\subsection{В качестве ответственного исполнителя}
	
	\cventry{2018--2020}{Инициативные проекты}{Российский фонд фундаментальных исследований (\mbox{РФФИ})}{руководитель: Г.\,С.~Осипов}{}{Взаимодействие поведения и рассуждений в знаковой картине мира.}
	
	\cventry{2017--2020}{Ориентированные фундаментальные исследования}{\mbox{РФФИ}}{руководиетль: Н.\,В.~Чудова}{}{Сетевая модель знаковой картины мира и реализация в ней когнитивных функций.}
		
	\cventry{2016--2018}{Гранты по приоритетным направлениям исследований}{Российский научный фонд (РНФ)}{руководитель: Г.\,С.~Осипов}{}{Создание теории, методов и моделей децентрализованного управления поведением коллективов когнитивных робототехнических систем в недетерминированной среде.}
	
	\cventry{2015--2017}{Инициативные проекты}{Российский фонд фундаментальных исследований (\mbox{РФФИ})}{руководитель: Г.\,С.~Осипов}{}{Нейрофизиологические и психологические основания знаковой картины мира и моделей когнитивных функций.}	

%----------------------------------------------------------------------------------------
%	INTERESTS SECTION
%----------------------------------------------------------------------------------------

\section{Научные интересы}

\renewcommand{\listitemsymbol}{-~} % Changes the symbol used for lists

\cvlistdoubleitem{компьютерное когнитивное моделирование}{многоагентные системы}
\cvlistdoubleitem{семиотика}{планирование поведения}
\cvlistdoubleitem{когнитивная робототехника}{обучение с подкреплением}


%\section{Grants and Awards}
%
%\cvitem{Title}{\emph{Money Is The Root Of All Evil -- Or Is It?}}
%\cvitem{Supervisors}{Professor James Smith \& Associate Professor Jane Smith}
%\cvitem{Description}{This thesis explored the idea that money has been the cause of untold anguish and suffering in the world. I found that it has, in fact, not.}

\section{Научные награды, общества}
\cvitem{2017}{Лауреат медали Российской академии наук для молодых ученых 2017.}
\cvitem{2016--н.в.}{Организатор международных конференций и школ по ИИ: \href{isyt2017.spiiras.nw.ru}{ИСиТ 2017}, .\href{http://2018.icsa-conf.ru}{ИУСА 2018}, \href{http://2018.rncai.ru}{КИИ 2018}, \href{http://bicasociety.org/meetings/}{BICA 2016, 2017}, \href{http://school.bicasociety.org}{Fierces on BICA 2016, 2017}.}
\cvitem{2016--2018}{Член редколлегии журнала \textit{Biologically Inspired Cognitive Architectures}: \href{http://www.journals.elsevier.com/biologically-inspired-cognitive-architectures/}{BICA Journal}.}
\cvitem{2016--по н.в.}{Член Сообщества биологически инспирированных когнитивных архитектур: \href{http://bicasociety.org}{BICA Society}.}
\cvitem{2016--по н.в.}{Ментор студенческой лаборатории по ИИ: \href{www.slabai.ru}{SLabAI}.}
\cvitem{2015--по н.в.}{Член Российской ассоциации искусственного интеллекта: \href{www.raai.org}{РААИ}.}


\section{Наукометрические индикаторы}
\cvitem{РИНЦ}{SPIN: \href{http://elibrary.ru/author_profile.asp?authorid=724544}{5115-9360}, AuthorID: \href{http://elibrary.ru/author_profile.asp?authorid=724544}{724544}, $N=31$, $N_5=27$, $h=6$, $n_{cit}=122$}

\cvitem{Scopus}{AuthorID: \href{https://www.scopus.com/authid/detail.uri?authorId=56504794900}{56504794900}, ORCID: \href{http://orcid.org/0000-0002-9747-3837}{0000-0002-9747-3837}, $N=17$, $N_5=17$, $h=4$, $n_{cit}=44$}

\cvitem{WebOfScience}{ResearcherID: \href{http://www.researcherid.com/rid/L-9171-2013}{L-9171-2013}, $N=12$, $N_5=12$, $h=2$, $n_{cit}=17$}

\cvitem{Scholar}{Google Scholar ID: \href{https://scholar.google.ru/citations?user=6pijIbMAAAAJ&hl=ru}{6pijIbMAAAAJ}, $N=45$, $N_5=36$, $h=7$, $n_{cit}=157$}

%----------------------------------------------------------------------------------------
%	COMPUTER SKILLS SECTION
%----------------------------------------------------------------------------------------

%\section{Computer skills}
%
%\cvitem{Basic}{\textsc{java}, Adobe Illustrator}
%\cvitem{Intermediate}{\textsc{python}, \textsc{html}, \LaTeX, OpenOffice, Linux, Microsoft Windows}
%\cvitem{Advanced}{Computer Hardware and Support}

%----------------------------------------------------------------------------------------
%	LANGUAGES SECTION
%----------------------------------------------------------------------------------------

%\section{Languages}
%
%\cvitemwithcomment{English}{Mothertongue}{}
%\cvitemwithcomment{Spanish}{Intermediate}{Conversationally fluent}
%\cvitemwithcomment{Dutch}{Basic}{Basic words and phrases only}

\nocite{*}
\printbibliography[title={Основные публикации}]

\end{document}