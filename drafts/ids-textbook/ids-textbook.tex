\documentclass[b5paper,11pt]{book}

\usepackage{geometry} 					% поля страницы

\usepackage{cmap}                       % Поддержка поиска русских слов в PDF (pdflatex)
\usepackage[T2A]{fontenc}				% Поддержка русских букв
\usepackage[utf8]{inputenc}            	% Выбор языка и кодировки
\usepackage[english, russian]{babel}	% Языки: русский, английский

\usepackage[unicode]{hyperref}			% Русский язык для оглавления pdf
\usepackage{bookmark}					% Оглавление в pdf
\usepackage{graphicx} 					% Подключаем пакет работы с графикой

\usepackage{amsmath,amssymb}

\graphicspath{{../../images/}} 			% Пути к изображениям

\geometry{left=2cm,right=2cm,top=2cm,bottom=2cm}	% Геомтерия страницы

\usepackage[
%	autolang=hyphen,
language=auto,
autolang=other,
backend=biber,
style=gost-numeric
]{biblatex}
\addbibresource{../../biblio/library.bib}

\DeclareSourcemap{
	\maps[datatype=bibtex, overwrite]{
		\map{
			\step[fieldset=langid, fieldvalue=english]
			\step[fieldset=doi, null]
			\step[fieldset=issn, null]
			\step[fieldset=isbn, null]
			\step[fieldset=url, null]
			\step[fieldsource=language, fieldset=langid, origfieldval]
		}
	}
}

\let\cleardoublepage\clearpage

\begin{document}
	\begin{titlepage}
		\begin{center}
			{\bfseries  Федеральное государственное автономное \\
				образовательное учреждение высшего образования\\
				<<Российский университет дружбы народов>>
				
			}

			\vspace{-5pt}
			\noindent\rule{\textwidth}{2pt}
			
			\vspace{50pt}
			{\Large\bfseries А.\,И.~Панов}
			
			\vspace{100pt}
			{\Huge\bfseries Интеллектуальные динамические системы}
			
			\vspace{20pt}
			{\Large\itshape Учебно-методическое пособие}
			
			\vfill
			{\bfseries Москва\\
				Российский университет дружбы народов\\
				2015
			}
		\end{center}
	\end{titlepage}
	
	\chapter*{}
	
	В пособии рассмотрены основные подходы и методы анализа текстов на естественных языках, а также программные инструментальные средства, которые можно использовать для решения различных практических задач, связанных с обработкой текстов. Пособие имеет практическую направленность, не требует от читателя глубоких познаний в области лингвистики, и предназначено для
	
	
	\tableofcontents %% содержание
		
	\chapter*{Введение}
	\addcontentsline{toc}{chapter}{Введение}
	Немного о целях курса
	
	\chapter{Представление знаний}
	
	\chapter{Логика предикатов первого порядка}
	
	
	\printbibliography
\end{document}