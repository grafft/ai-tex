\documentclass[b5paper,11pt]{book}

\usepackage{geometry} 					% поля страницы

\usepackage{cmap}                       % Поддержка поиска русских слов в PDF (pdflatex)
\usepackage[T2A]{fontenc}				% Поддержка русских букв
\usepackage[utf8]{inputenc}            	% Выбор языка и кодировки
\usepackage[english, russian]{babel}	% Языки: русский, английский

\usepackage[unicode]{hyperref}			% Русский язык для оглавления pdf
\usepackage{bookmark}					% Оглавление в pdf
\usepackage{graphicx} 					% Подключаем пакет работы с графикой

\usepackage{amsmath,amssymb}

\graphicspath{{../../images/}} 			% Пути к изображениям

\geometry{left=2cm,right=2cm,top=2cm,bottom=2cm}	% Геомтерия страницы

\usepackage[
%	autolang=hyphen,
language=auto,
autolang=other,
backend=biber,
style=gost-numeric
]{biblatex}
\addbibresource{ids.bib}

\DeclareSourcemap{
	\maps[datatype=bibtex, overwrite]{
		\map{
			\step[fieldset=langid, fieldvalue=english]
			\step[fieldset=doi, null]
			\step[fieldset=issn, null]
			\step[fieldset=isbn, null]
			\step[fieldset=url, null]
			\step[fieldsource=language, fieldset=langid, origfieldval]
		}
	}
}

\let\cleardoublepage\clearpage

\begin{document}
	\begin{titlepage}
		\begin{center}
			{\bfseries  Федеральное государственное автономное \\
				образовательное учреждение высшего образования\\
				<<Российский университет дружбы народов>>
				
			}

			\vspace{-5pt}
			\noindent\rule{\textwidth}{2pt}
			
			\vspace{50pt}
			{\Large\bfseries А.\,И.~Панов}
			
			\vspace{100pt}
			{\Huge\bfseries Интеллектуальные динамические системы}
			
			\vspace{20pt}
			{\Large\itshape Учебно-методическое пособие}
			
			\vfill
			{\bfseries Москва\\
				Российский университет дружбы народов\\
				2015
			}
		\end{center}
	\end{titlepage}
	
	\chapter*{}
	
	В пособии рассмотрены основные методы, применяющиеся при построении интеллектуальных динамических систем (ИДС). Одним из основных свойств ИДС является свойство иерархичности, уровневости организации всех процессов, связанных с ИДС, начиная от управления такими системами и заканчивая процессами самоорганизации в их базе знаний.
	
	
	\tableofcontents %% содержание
		
	\chapter*{Введение}
	\addcontentsline{toc}{chapter}{Введение}
	Динамические интеллектуальные системы "--- результат интеграции интеллектуальных систем с динамическими системами. В общем случае это двухуровневые динамические модели, где один из уровней отвечает за стратегию поведения системы (или, как иногда говорят, носит делиберативный характер), а другой уровень отвечает за реализацию конкретной (в том числе, математической) модели.
	
	К таким системам относятся сложные естественные системы, такие как экологические, социальные и политические системы, а также такие динамические системы, в которых зависимости настолько сложны, что не допускают своего обычного аналитического представления. Сложность задач управления, в которых существенная роль принадлежит экспертным суждениям и знаниям человека, заставляет в дополнение к количественным методам или вместо них применять такие подходы, в которых в качестве значений переменных допускаются не только числа, но и слова или предложения искусственного или естественного языка. 
	
	Потребность в моделях такого рода назрела в связи с развитием, например, беспилотных средств транспортного и иного назначения. В частности, в беспилотных автономных самолетах и вертолётах одним из уровней управления должен являться делиберативный уровень управления, решающий задачи, например, планирования полёта или выбора траектории или выбора цели. Другой уровень управления "--- назовем его активным "--- реализует требуемые действия. Например, на  делиберативном уровне управления беспилотным вертолётом принимается решение о зависании над целью, тогда на активном уровне начинает работать математическая модель зависания, вырабатывающая требуемые управления для исполнительных механизмов.
	
	\chapter{Представление статических знаний}
	
	\section{Логика предикатов первого порядка}
	\section{Атрибутивная логика}
	\section{Семантические сети}
	


	\chapter{Представление процедурных знаний}
	
	\section{Системы правил}
	\section{Семиотическое представление}
	


	\chapter{Пополнение знаний}
	
	\section{Проблема привязки символов}
	\section{Биологически правдоподобные методы}
	\section{Выявление причинно-следственных связей}
	
	
	
	\chapter{Планирование поведения}
	
	\section{Классические алгоритмы планирования}
	\subsection{Планирование как доказательство теорем}
	\subsection{Планирование в пространстве состояний}
	\subsection{Планирование на основе прецедентов}
	
	\section{Планирование с удовлетворением ограничений}
	\section{Графические системы планирования}
		
		
	
	\chapter{Системы, основанные на правилах}
	
	\section{Состояния и траектории}
	\section{Синтез управления}
	\section{Синтез обратной связи}
	\section{Основы теории управляемости}
	


	\chapter{Практические задания в системе Jadex}
	
	\section{Внешняя среда и типы агентов}
	\section{Задание состояний}
	\section{Задание правил и стратегий}
	\section{Планирование поведения}
	\section{Задачи по планированию}
	
	

	\chapter*{Заключение}
	\addcontentsline{toc}{chapter}{Заключение}
	Немного о итогах курса
	\printbibliography
\end{document}