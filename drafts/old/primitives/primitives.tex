\documentclass[a4paper,12pt]{article}

\usepackage{geometry} 					% поля страницы

\usepackage{cmap}                       % Поддержка поиска русских слов в PDF (pdflatex)
\usepackage[T2A]{fontenc}				% Поддержка русских букв
\usepackage[utf8]{inputenc}            	% Выбор языка и кодировки
\usepackage[english, russian]{babel}	% Языки: русский, английский

\usepackage[unicode]{hyperref}			% Русский язык для оглавления pdf
\usepackage{bookmark}					% Оглавление в pdf
\usepackage{graphicx} 					% Подключаем пакет работы с графикой

\usepackage{amsmath,amssymb}
\usepackage[plain]{algorithm}
\usepackage[noend]{algpseudocode}

\usepackage[affil-it]{authblk}

\geometry{left=3cm,right=2cm,top=2cm,bottom=2cm}	% Геомтерия страницы
\graphicspath{{../../images/}} 			% Пути к изображениям
\bibliographystyle{gost2008s}

\begin{document}
	\title{Функциональные примитивы мозга}
	\author{А.\,И.~Панов}
	\affil{Институт системного анализа РАН}
	
	\maketitle{}
	
	% оформление аннотации
	\begin{abstract}
		Главная идея
	\end{abstract}
	
	\section*{Введение}
	Можно упомянуть работу Эндрю Коварада \cite{Coward2014} о примитивах в качестве примера подобных работ.
	
	\section{Базовые принципы}
	
	\section{Алгоритмы ФПМ и их нейрофизиологические корреляты}
	
	\section{Алгоритм EIK}
	
	\section{Верификация моделей ФПМ}
	
	\section{Верицикация EIK}
	\subsection{Низкоуровневая верицикация: восприятие}
	\subsection{Высокоуровневая верицикация: планирование}
	
	\section*{Заключение}
	
    \bibliography{../../biblio/umain}
    
\end{document} 