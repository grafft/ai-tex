\documentclass[a4paper,12pt]{article}

\usepackage{geometry} 					% поля страницы

\usepackage{cmap}                       % Поддержка поиска русских слов в PDF (pdflatex)
\usepackage[T2A]{fontenc}				% Поддержка русских букв
\usepackage[utf8]{inputenc}            	% Выбор языка и кодировки
\usepackage[english, russian]{babel}	% Языки: русский, английский
\usepackage{csquotes}

\usepackage[unicode]{hyperref}			% Русский язык для оглавления pdf
\usepackage{bookmark}					% Оглавление в pdf
\usepackage{graphicx} 					% Подключаем пакет работы с графикой
\usepackage{textgreek}					% Греческий текст без переключения в math-mode

\usepackage{amsmath,amssymb}
\usepackage[plain]{algorithm}
\usepackage[noend]{algpseudocode}

\usepackage[affil-it]{authblk}			% Красивая аффиляция авторов

\geometry{left=3cm,right=2cm,top=2cm,bottom=2cm}	% Геомтерия страницы
\graphicspath{{../../images/}} 			% Пути к изображениям

\usepackage[
%	autolang=hyphen,
language=auto,
autolang=other,
backend=biber,
style=gost-numeric
]{biblatex}
\addbibresource{../../biblio/library.bib}

\DeclareSourcemap{
	\maps[datatype=bibtex, overwrite]{
		\map{
			\step[fieldset=langid, fieldvalue=english]
			\step[fieldset=doi, null]
			\step[fieldset=issn, null]
			\step[fieldset=isbn, null]
			\step[fieldset=url, null]
			\step[fieldsource=language, fieldset=langid, origfieldval]
		}
	}
}

\begin{document}
	\title{Знак как элемент картины мира}
	\author{А.\,И.~Панов}
	\affil{Институт системного анализа РАН}
	
	\maketitle{}
	
	% оформление аннотации
	\begin{abstract}
		Главная идея
	\end{abstract}
	
	\section*{Введение}
	Основная цель работы "--- формализовать понятие знака и построить алгоритм его образования.
	
	\section{Понятие знака в науке}
	Понятие знака вводится и используются во многих отраслях знания. В первую очередь необходимо отметить, что знакам посвящено отдельное научное направление под названием \textit{семиотика} (от др.-греч. \textsigma\texteta\textmu\textepsilon\textiota\textomikron\textnu ~--- <<знак, признак>>), берущее своё начало в конце XIX~в. с работ американского философа Чарльза Пирса (Charles Sanders Peirce, 1839--1914) и французского лингвиста Фердинанда де Соссюра (Ferdinand de Saussure, 1857--1913). Семиотика является междисциплинарным направлением, предоставляющим общую терминологическую базу для исследователей из разных предметных областей. 
	
	Семиотику можно рассматривать некоторым исходным базисом, на основе которого в XX~в. стали развиваться новые научные направления уже в рамках конкретных предметных областей. Наиболее органично понятие знака было включено в лингвистику и языкознание, где принято говорить о языковом знаке как части некоторой языковой системы. В след за Соссюром центральное место знаку в структуре языка отводилось Луи Ельмслевом (Louis Trolle Hjelmslev, 1899-1965), Томасом Себеоком (Thomas Albert Sebeok, 1920-2001).
	
	Философскую ветвь семиотики в первую очередь принято связывать с Готтлобом Фреге (Friedrich Ludwig Gottlob Frege, 1848-1925), для которого понятие знака служило для разграничения смысла и значения, что в свою очередь имело большое значение для развития логики. Дальнейшее развитие философских аспектов знака получило в работах Чарльза Морриса (Charles William Morris, 1901-1979), Ролана Барта (Roland Barthes, 1915-1980).
	
	Отдельно следует отметить научное исследование знака и его роли в культуре и литературе. Литературоведческая ветвь семиотики пополнялась работами Альгирдаса Греймеса (Algirdas Julius Greimas, 1917-1992), Умберто Эко (Umberto Eco, род. 1932). Большой вклад в этом направлении был сделан отечественными литературоведами Романом Осиповичем Якобсоном (1896-1982) и Юрием Михайловичем Лотманом (1922-1993).
	
	Связь семиотики и психологии была впервые наглядно продемонстрирована в культурно~--историческом подходе Льва Семёновича Выготского (1896-1934). Психология выработало своё представление о структуре и роли знака в деятельности человека, достаточно сильно отличающееся от семиотических представлений. Дальнейшее развитие психологическая ветвь семиотики получила в работах Алексея Николаевича Леонтьева (1903-1979), Петра Яковлевича Гальперина (1902-1988).
	
	Связью семиотики, информационных и технических наук посвящена так называемая прикладная ветвь семиотики: компьютерная или прикладная семиотика. Это направление получило своё начало в работах отечественных специалистов по искусственному интеллекту Дмитрия Александровича Поспелова (род. 1932) и Геннадия Семёновича Осипова (род. 1948). Семиотическая парадигма в вопросах управления и представления знаний является новым и достаточно многообещающим подходом, в том числе направленным и на создание конкретных прикладных программных и технических систем.
	
	\subsection{Первые семиотические работы}
	
	Как отмечает Лотман, ещё в конце XVII~в. английский философ~--материалист Джон Локк (John Locke, 1632-1704) достаточно точно,~с точки зрения современных представлений, определил сущность и <<объём>> семиотики \cite[8]{Lotman2000}. По Локку задача семиотики "--- <<рассмотреть природу знаков, которыми ум пользуется для понимания вещей или для передачи своего знания другим>>. Однако, несмотря на столь раннее введение понятия знака в научную лексику, семиотические идеи ещё почти двести лет не получали своего развития.
	
	\subsubsection{Семиотика по Пирсу}
	Отчёт научного направления под названием семиотика ведут с работ Пирса, чьи идеи получили широкое распространение только в 1930-е годы с опубликованием его архивов (русскоязычные переводы \cite{Pierce2000a,Pierce2000b, Pierce2009}). По Пирсу знак "--- это сущность, к которой мы проявляем заинтересованность не как к таковой, а как передающей некоторую идею о другой сущности, или другими словами, знак есть нечто, что замещает собой нечто (свой \textit{объект}) для кого-то в некотором отношении или качестве, отсылая к некоторой идее (к \textit{основанию} знака). Знак адресуется кому-то, создавая в уме этого человека эквивалентный знак (\textit{интерпретант}).
	
	По Пирсу наука семиотика, в следствие строения знака, имеет три раздела: чистую грамматику, занимающуюся основаниями знаков, логика, занимающаяся объектами знаков, и чистая риторика, занимающаяся интерпретантами знаков. Объектами знаков могут быть воспринимаемые, воображаемые и несуществующие (невообразимые) сущности. Обычно знак материально не совпадает со своим объектом и может иметь несколько объектов.
	
	Пирс создал несколько классификаций знаков. По одной из трихотомий он определял виды знаков следующим образом:
	\begin{itemize}
		\item знаки"--~подобия, или иконы, репрезентируют объекты, просто имитируя их (полностью или только некоторые их характерные признаки) и не имея непосредственной связи с объектом (например, набросок статуи, звуки"--~подобия и т.~п.);
		\item знаки"--~указатели,~или индексы, физически связаны с репрезентируемым объектом, указывая не  конкретный объект или множество объектов (например, дорожный знак, флюгер, походка враскачку и др.);
		\item общие знаки, или символы, ассоциируются со своим объектом благодаря некоторой договорённости или привычке, указывая не на конкретный объект, а на некоторый общий тип (например, любое обычное слово).
	\end{itemize}
	
	По Пирсу во всех своих рассуждениях человек употребляет смесь икон, индексов и символов, хотя символизм и превалирует, так как именно благодаря символам создаются абстракции. Эта смесь в свою очередь является символом, т.~е. символ включает в себя отсылку на икону и индекс. Знаки обладают материальными качествами, не связанными с его репрезентативной функцией (трёхбуквенность слова <<man>> и плоскость изображения на бумаге). Пирс отдельно подчёркивал, что знак должен быть связан (реально, а не мысленно) либо с ещё одним знаком того же объекта, либо с самим этим объектом.
	
	\subsubsection{Смысл и значение Фреге}
	
	Занимаясь определением и описанием свойств понятий, Фреге в своих первых трудах не разделял смысл и значение. Только в статье <<О смысле и значении>> 1892~г., исследуя вопрос о природе равенства, он определил различия между этими двумя терминами \cite{Frege2000, Birukov1960}. У имени, или знака (Фреге рассматривал только знаки"--~символы по терминологии Пирса), имеется \textit{значение} (нем. Sinn), которое является тем предметом, который обозначается этим именем, и \textit{смысл} (нем. Bedeutung), являющийся информацией, заключённой в имени.
	
	В первую очередь Фреге рассматривал собственные имена, т.~е. имена, называющие определённый предмет, но не функцию и не отношение, например <<Аристотель>>, <<Венера>>, <<воспитатель Александра Македонского и ученик Платона>>, <<тот, кто открыл эллиптическую формул планетных орбит>> и т.~п. Имена <<Вечернаяя звезда>> и <<Утренняя звезда>> имеют одно и то же значение (что и имя <<Венера>>), но разные смысла, так как передают различную информацию о том, когда можно увидеть эту звезду.
	
	Кроме смысла и значения, Фреге выделял также связанное со знаком представление. В отличие от значения, которое является непосредственно воспринимаемым предметом, представление является субъективным внутренним образом, возникшим на основе чувственного восприятия. Смысл, в противоположность представлению, является объективным достоянием целого коллектива людей, занимая по отношению к самому предмету промежуточное положение между представлением и значением.
	
	Фреге предполагал взаимно-однозначное соответствие между знаком и именем. В имени с помощью закрепления языковыми средствами выражается информация о предмете, однозначно характеризующая этот предмет. В логике у имени (знака) обязательно должен быть только один смысл и одно значение. В случае естественного языка у знака может быть несколько смыслов и может не быть значений (в этом случае Фреге называл такое имя мнимым). В свою очередь значение (предмет) могут иметь несколько имён, а значит, и смыслов. Случай, когда у нескольких имён один смысл, является случаем синонимии.
	
	Фреге делил собственные имена на просты и сложные (описания, предложения). Последние состоят из осмысленных элементов, составляющих имён. Грамматика языка определяет те правила, по которым образуются осмысленные сложные имена.  Фреге в <<Исчислении понятий>> впервые определил для формального языка логики исходные имена и строгие грамматические правила, по которым образовывались <<правильно составленные имена>>. Значением предложения считается его истинностное значение, т.~е. что оно либо истинно, либо ложно.
	
	Смысл сложного имени строится с помощью смыслов составляющих имён и грамматических правил. Смысл (информация) же простых имён заключается в том определение, которые имеется для этого имени у человека, как члена некоторого коллектива. <<Колебания смысла>> возможны в естественном языке, но не допустимы при построении науки.
	
	По работам Фреге можно определить следующие основные принципы определения смысла и значения сложных имён \cite{Birukov1960}:
	\begin{itemize}
		\item \textit{принцип независимости}: смысл сложного имени не зависит от значения составляющих имён;
		\item \textit{первый принцип замены}: если одно из составляющих имён, входящих в данное сложное имя, заменить именем, имеющим то же, что и у заменяемого, значение, то сложное имя, получившееся в результате такой замены, будет иметь значение, совпадающее со значением исходного сложного имени (например, при замене составляющего имени <<Платон>> на имя <<основатель Академии>> значение сложного имени не изменится);
		\item \textit{второй принцип замены}: если одно из составляющих имён, входящих в данное сложное имя, заменить другим именем с тем же, что и у заменяемого, смыслом (т.~е. его синонимом) новое сложное имя выражает тот же смысл, что и первоначальное; в силу принципа замены на равнозначное полученное сложное имя будет иметь значение, совпадающее со значением исходного сложного имени;
		\item \textit{третий принцип замены}: если одно из составляющих имён, входящих в данное сложное имя, заменить именем, смысл которого отличен от смысла заменяемого имени, то полученное таким образом новое сложное имя выражает уже иной смысл, нежели первоначальное; при этом значение полученного сложного имени "--- в силу многозначного характера отношения смысла к значению "--- может совпадать со значением исходного имени, но может оказаться и отличным от него;
		\item \textit{четвёртый принцип замены}: если составляющее имя заменяется именем, значение которого иное, нежели у заменяемого, сложное имя, получающееся в результате, может иметь значение как отличное от значения первоначального имени, так и совпадающее с ним; смысл в этом случае всегда отличен от смысла исходного имени;
		\item \textit{принцип сохранения неозначенности}: если составляющее имя не имеет значения, то и сложное имя, в состав которого это составляющее имя входит, значения не имеет.
	\end{itemize}
	
	Бесспорной заслугой Фреге является то, что он уделял серьёзное внимание отличию знака от обозначаемого им предмета. Такое смешение может возникнуть, когда средствами языка обозначаются сами элементы или свойства языка. Примером может служить косвенная речь, когда используются знаки знаков: <<Сенека писал: ``Rationale animal est homo''>>. Значением такого метазнака <<``Rationale animal est homo''>> как является знак <<Rationale animal est homo>>. В косвенной речи, все имена имеют косвенный смысл и косвенное значение. Косвенный смысл имени <<А>> совпадает со смыслом слов <<смысл имени ``А''>>. Косвенное значение имени "--- это его обычный смысл.
	
	\subsubsection{Феноменологический подход Гуссерля}
	
	 Теория знака немецкого философа и основателя феноменологии Эдмунда Гуссерля (Edmund Husserl, 1859--1938) была изложена им во второй части <<Логических исследований>>, опубликованных в самом начале XX~в. Гуссерль определял и отделял друг от друга смысл и значение отличным от Фреге способом. Смысл знака (Sinn) "--- это особая характеристика когнитивного, познавательного акта, рассмотренного со стороны его идеальной априорной структуры "--- интенциональности (направленности на \dots). Значение (Bedeutung) же является лингвистически оформленным смыслом (Sinn) \cite{Zaitceva2002}. Предметной характеристикой когнитивного акта является объект (Gegenstand). Другими словами, понятие значения Фреге распадается у Гуссерля на смысловую (Bedeutung) и предметную составляющую (Gegenstand).
	 
	 Классификация знаков по Гуссерлю включает в себя знаки"--~признаки и знаки"--~выражения. Если первые не обладают значением (смыслом), являясь оповещением и указанием на что-то (например, клеймо "--- признак раба, флаг "--- признак нации), то функция вторых "--- придавать конкретное значение. Знаки"--~признаки <<ничего не выражают, разве что наряду с функцией оповещения они выполняют еще некоторую функцию значения>> \cite[35]{Gusserl2001}. Их функция "--- это оповещение. Знаки"--~признаки могут иметь значение по отношению к действию их создания, либо по отношению к оповещению, которое они несут.
	 
	 Знаком"--~выражением является любая речь, как внутренняя, так и коммуникативная, и любая её составная часть. Выражение, обладающее именем, <<извещает>> о некоторых связанных с этим выражением психических переживаниях, имеет значение (смысл), т.~е. <<содержание>> номинативного представления, и именует некоторый объект, предмет представления. В процессе коммуникации знаки"--~выражениям являются одновременно и признаками, оповещая о смыслодающих психических переживаниях собеседника \cite[43]{Gusserl2001}. Наедине с самим собой, выражения уже не несут оповещающей функции. С выражением связаны два акта (мыслительных действия): акт, придающий значение, или интенция значения, и акт осуществления, установления, подтверждения значения, актуализирующие предметное отношение. Например, у выражения <<круглый квадрат>> есть значение, т.~е. выполняется интенция значения, но осуществление значения не происходит. Значения, по Гуссерлю, "--- идеальные единства, они не могут меняться; а <<колебание значений есть, собственно, колебание акта придания значения>>.
	 	
	\subsection{Лингвистические работы}
	
	\subsection{Философские исследования}
	
	\subsection{Прикладная семиотика}
	
	\subsection{Формализация понятия}
	
	\section{Структура знака}
	\section*{Заключение}
	
   \printbibliography
\end{document} 