\documentclass[a4paper,12pt]{article}

\usepackage{geometry} 					% поля страницы

\usepackage{cmap}                       % Поддержка поиска русских слов в PDF (pdflatex)
\usepackage[T2A]{fontenc}				% Поддержка русских букв
\usepackage[utf8]{inputenc}            	% Выбор языка и кодировки
\usepackage[english, russian]{babel}	% Языки: русский, английский
\usepackage{csquotes}

\usepackage[unicode]{hyperref}			% Русский язык для оглавления pdf
\usepackage{bookmark}					% Оглавление в pdf
\usepackage{graphicx} 					% Подключаем пакет работы с графикой

\usepackage{amsmath,amssymb}

\usepackage[affil-it]{authblk}			% Красивая аффиляция авторов

\geometry{left=3cm,right=2cm,top=2cm,bottom=2cm}	% Геомтерия страницы
\graphicspath{{../../images/}} 			% Пути к изображениям

\usepackage[
%	autolang=hyphen,
language=auto,
autolang=other,
backend=biber,
style=gost-numeric
]{biblatex}
\addbibresource{../../biblio/library.bib}

\DeclareSourcemap{
	\maps[datatype=bibtex, overwrite]{
		\map{
			\step[fieldset=langid, fieldvalue=english]
			\step[fieldset=doi, null]
			\step[fieldset=issn, null]
			\step[fieldset=isbn, null]
			\step[fieldset=url, null]
			\step[fieldsource=language, fieldset=langid, origfieldval]
		}
	}
}

\begin{document}
	\title{Принципы организации сознательной деятельности человека}
	\author{А.\,И.~Панов}
	\affil{Институт системного анализа ФИЦ ИУ РАН}
	
	\maketitle{}
	
	% оформление аннотации
	\begin{abstract}
		Главная идея
	\end{abstract}
	
	\section*{Введение}

	\section{Основные принципы}
	
	<<Высшие формы сознательной деятельности человека, конечно, осуществляются мозгом и опираются на законы высшей нервной деятельности. Однако они порождаются сложнейшими взаимоотношениями человека с общественной средой и формируются в условиях общественной жизни, которая способствует возникновению новых функциональных систем, в соответствии с которыми работает мозг, и поэтому попытки вывести законы этой сознательной деятельности из самого мозга, взятого вне социальной среды, обречены на неудачу>> \cite{Luria1977}.
	
	<<Включение системы речевых связей в значительное число процессов, которые раньше имели непосредственный характер, является важнейшим фактором формирования высших психических функций, которыми человек отличается от животного и которые тем самым приобретают характер сознательности и произвольности>> \cite{Luria2000}.
	
	<<Высшие психические процессы являются функцией всего мозга и что работу мозговой коры можно рассматривать лишь в тесной связи с анализом более низко расположенных нервных аппаратов>> \cite{Luria2000}.
	
	<<Восприятие осуществляется при совместном участии всех функциональных блоков мозга, из которых первый обеспечивает нужный тонус коры, второй осуществляет анализ и синтез поступающей информации, а третий обеспечивает направленные поисковые движения, создавая тем самым активный характер воспринимающей деятельности>> \cite{Luria2003}.
	
	<<В целом, в современной нейрофизиологии и нейропсихологии сложилось представление о том, что произвольные движения "--- это очень сложно афферентированные систем, которые реализуются при участии почти всей коры больших полушарий>> \cite{Homskaya2015}.
	
	По Выготскому \cite{Vygotsky1984} основу новой психологии эмоций должны составлять:
	\begin{itemize}
		\item принципы историзма, системности, социальной природы психики с одной стороны и
		\item принципы особой биологической (органической, мозговой) реализация эмоциональных явлений, отличные от принципов реализации познавательных процессов "--- с другой.		
	\end{itemize}
	
	Само сознание трактовалось Л.\,С.~Выготским как сложное системное и смысловое образование, в формировании которого центральную роль играет  речь, слово. Он считал, что речь, слово (т.~е. использование языка в процессе общения) является коррелятом сознания, а не мышления. Л.\,С.~Выготский различал \textit{значение} как объективно сложившуюся устойчивую систему обобщений, стоящую за словом, и \textit{смысл} как индивидуальный, субъективный аспект значения \cite{Homskaya2015, Vygotsky1960}.
	
	По Выготскому в основе культурно-исторической теории развития высших психических функций лежит учение о системном и смысловом строении сознания человека, исходящее из первостепенного значения следующих явлений \cite{Homskaya2015}:
	\begin{itemize}
		\item изменчивости межфункциональных связей и отношений;
		\item образования сложных динамических систем, интегрирующих целый ряд элементарных функций;
		\item обобщающего отражения действительности в сознании.
	\end{itemize}
	
	Лурия и ученики рассматривают сознание как высшую форму отражения человеком внешнего (объективного) и внутреннего (субъективного) мира в виде символов (слов, знаков) и образов, как интегративный обобщенный <<образ мира>> <<образ своего Я>>, как продукт деятельности мозга \cite{Homskaya2015}.
	
	Нейрофизиологическая модель сознания включает в себя принцип иерархической организации психических явлений, который по отношению к сознанию означает множественную вертикальную представленность мозговых аппаратов сознания на разных его уровнях "--- полностью осознаваемых, частично осознаваемых и бессознательных форм психической деятельности. Взаимодействие этих уровней создает возможность перехода одной формы в другую в определенных условиях (например, в ситуации психотерапевтического воздействия). Уровневый иерархический принцип мозговой организации сознания предполагает участие в его мозговом обеспечении как корковых, так и подкорковых "--- глубинных "--- мозговых образований (что соответствует психоаналитическим представлениям об общей структуре сознания) \cite{Homskaya2015}.
	
	Разные отделы мозга вносят следующий вклад в мозговую организацию сознания \cite{Homskaya2015}:
	\begin{itemize}
		\item неспецифические мозговые механизмы разных уровней ствола обеспечивают \textit{активационную}, <<количественную>> составляющую сознания, его <<ясность>> или <<угнетенность>>;
		\item неспецифические образования лимбической системы (поясная кора, амигдала, гиппокамп и др.) обеспечивают \textit{эмоционально-аффективную} составляющую сознания, осознание собственного эмоционального опыта, своего Я;
		\item корковые зоны левого полушария обеспечивают речевое и другое \textit{знаковое символическое опосредования} сознания, его представленность в виде речевых и других символических семантических конструктов; они обеспечивают левополушарный (опосредованный символами) тип отражения пространства и времени;
		\item корковые зоны правого полушария обеспечивают \textit{образное опосредование} сознания (его \textit{<<чувственную ткань>>}), его представленность в виде обобщенных образных конструктов; они ответственны за правополушарный (непосредственный) тип отражения пространственных и временных координат сознания;
		\item корково-подкорковые (в частности, корково-таламические) связи обоих полушарий обеспечивают \textit{вертикальную уровневую организацию} явлений сознания, их подвижный характер, переход сознательных форм психической деятельности в плохо осознаваемое и совсем неосознаваемое (и наоборот);
		\item префронтальные отделы коры больших полушарий обеспечивают \textit{произвольное управление} процессами сознания, способность человека произвольно фиксировать сознание на определенных объектах внешней среды или внутренних ощущениях и осознавать себя в качестве субъекта психической или физической деятельности;
		\item срединные структуры мозга (мозолистое тело и другие комиссуры), объединяющие оба полушария в единый орган, обеспечивают \textit{целостный интегративный характер} сознания, возможность одновременного "--- и символического, и образного "--- отражения внешнего и внутреннего мира.
	\end{itemize}
	
 	\section*{Заключение}
	
   \printbibliography
\end{document} 