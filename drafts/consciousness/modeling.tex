\documentclass[a4paper,12pt]{article}

\usepackage{geometry} 					% поля страницы

\usepackage{cmap}                       % Поддержка поиска русских слов в PDF (pdflatex)
\usepackage[T2A]{fontenc}				% Поддержка русских букв
\usepackage[utf8]{inputenc}            	% Выбор языка и кодировки
\usepackage[english, russian]{babel}	% Языки: русский, английский
\usepackage{csquotes}

\usepackage[unicode]{hyperref}			% Русский язык для оглавления pdf
\usepackage{bookmark}					% Оглавление в pdf
\usepackage{graphicx} 					% Подключаем пакет работы с графикой

\usepackage{amsmath,amssymb}

\usepackage[affil-it]{authblk}			% Красивая аффиляция авторов

\geometry{left=3cm,right=2cm,top=2cm,bottom=2cm}	% Геомтерия страницы
\graphicspath{{../../images/}} 			% Пути к изображениям

\usepackage[
%	autolang=hyphen,
language=auto,
autolang=other,
backend=biber,
style=gost-numeric
]{biblatex}
\addbibresource{../../biblio/library.bib}

\DeclareSourcemap{
	\maps[datatype=bibtex, overwrite]{
		\map{
			\step[fieldset=langid, fieldvalue=english]
			\step[fieldset=doi, null]
			\step[fieldset=issn, null]
			\step[fieldset=isbn, null]
			\step[fieldset=url, null]
			\step[fieldsource=language, fieldset=langid, origfieldval]
		}
	}
}

\begin{document}
	\title{Принципы организации сознательной деятельности человека}
	\author{А.\,И.~Панов}
	\affil{Институт системного анализа ФИЦ ИУ РАН}
	
	\maketitle{}
	
	% оформление аннотации
	\begin{abstract}
		Главная идея
	\end{abstract}
	
	\section*{Введение}

	\section{Основные принципы}
	
	<<Высшие формы сознательной деятельности человека, конечно, осуществляются мозгом и опираются на законы высшей нервной деятельности. Однако они порождаются сложнейшими взаимоотношениями человека с общественной средой и формируются в условиях общественной жизни, которая способствует возникновению новых функциональных систем, в соответствии с которыми работает мозг, и поэтому попытки вывести законы этой сознательной деятельности из самого мозга, взятого вне социальной среды, обречены на неудачу>> \cite{Luria1977}.
	
	<<Включение системы речевых связей в значительное число процессов, которые раньше имели непосредственный характер, является важнейшим фактором формирования высших психических функций, которыми человек отличается от животного и которые тем самым приобретают характер сознательности и произвольности>> \cite{Luria2000}.
	
	<<Высшие психические процессы являются функцией всего мозга и что работу мозговой коры можно рассматривать лишь в тесной связи с анализом более низко расположенных нервных аппаратов>> \cite{Luria2000}.
	
	<<Восприятие осуществляется при совместном участии всех функциональных блоков мозга, из которых первый обеспечивает нужный тонус коры, второй осуществляет анализ и синтез поступающей информации, а третий обеспечивает направленные поисковые движения, создавая тем самым активный характер воспринимающей деятельности>> \cite{Luria2003}.
	\section*{Заключение}
	
   \printbibliography
\end{document} 