\documentclass[11pt]{memoir}

\usepackage{geometry} 					% поля страницы

\usepackage{cmap}                       % Поддержка поиска русских слов в PDF (pdflatex)
\usepackage[T2A]{fontenc}				% Поддержка русских букв
\usepackage[utf8]{inputenc}            	% Выбор языка и кодировки
\usepackage[english, russian]{babel}	% Языки: русский, английский

\usepackage[unicode]{hyperref}			% Русский язык для оглавления pdf
\usepackage{bookmark}					% Оглавление в pdf
\usepackage{graphicx} 					% Подключаем пакет работы с графикой
\usepackage{memhfixc}

\usepackage{amsmath,amssymb}

\graphicspath{{../../../images/}} 			% Пути к изображениям

\geometry{left=2cm,right=2cm,top=2cm,bottom=2cm}	% Геомтерия страницы

\usepackage[
	%	autolang=hyphen,
	language=auto,
	autolang=other,
	backend=biber,
	style=gost-numeric
]{biblatex}
\addbibresource{rl.bib}

\DeclareSourcemap{
	\maps[datatype=bibtex, overwrite]{
		\map{
			\step[fieldset=langid, fieldvalue=english]
			\step[fieldset=doi, null]
			\step[fieldset=issn, null]
			\step[fieldset=isbn, null]
			\step[fieldset=url, null]
			\step[fieldsource=language, fieldset=langid, origfieldval]
		}
	}
}

\let\cleardoublepage\clearpage

\begin{document}
	\pagestyle{empty}
		\begin{center}
			{\bfseries  Федеральное государственное автономное \\
				образовательное учреждение высшего образования\\
				<<Высшая школа экономики>>
				
			}

			\vspace{-5pt}
			\noindent\rule{\textwidth}{2pt}
			
			\vspace{50pt}
			{\Large\bfseries А.\,И.~Панов}
			
			\vspace{100pt}
			{\Huge\bfseries Методы и алгоритмы машинного обучения с подкреплением}
			
			\vspace{20pt}
			{\Large\itshape Учебно-методическое пособие}
			
			\vfill
			{\bfseries Москва\\
				Высшая школа экономики\\
				2018
			}
		\end{center}

	
	\frontmatter
	
	В пособии рассмотрены основные 
	
	\clearpage
	\tableofcontents %% содержание
		
	\mainmatter
	
	\chapter*{Введение}
	\addcontentsline{toc}{chapter}{Введение}
	Агент, среда, подкреплением, марковский процесс.

	\chapter{Табличные методы}
		\section{Марковский процесс принятия решений}
		
		\section{Динамическое программирование}
	
		\section{Методы Монте-Карло}
		
		\section{Q-обучение}
	
	\chapter{Приближенные методы}
		\section{Предсказание с изменением стратегии}
		
		\section{Предсказание без изменения стратегии}
		
		\section{Нейронные сети как аппроксиматоры}
	
	\chapter{Перспективные направления}
	
		\section{Иерархическое обучение с подкреплением}
			\subsection{Иерархия действий: Options}
			
			\subsection{Иерархия автоматов: HAM}
			
			\subsection{Оптимизация функции оценки: MaxQ}
			
			\subsection{Автоматическое формирование иерархий}
		\section{Внутренняя мотивация}
		
	
	\chapter{Обучение с подкреплением и другие науки}

		\section{Психология}
		
		\section{Нейрофизиология}

		\section{Робототехника}
	

	\chapter*{Заключение}
	\addcontentsline{toc}{chapter}{Заключение}
	Немного о целях
	\printbibliography
\end{document}