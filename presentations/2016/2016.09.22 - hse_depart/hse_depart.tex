\documentclass[default]{beamer}
\setbeamertemplate{navigation symbols}{}

\usetheme{CambridgeUS}
\useoutertheme{infolines}
\useinnertheme{circles}
\usecolortheme{seahorse}

\usepackage{cmap}							% Поддержка поиска русских слов в PDF (pdflatex)
\usepackage[T2A]{fontenc}       			%поддержка кириллицы
\usepackage[utf8]{inputenc}					% Выбор языка и кодировки
\usepackage[english, russian]{babel}
\usepackage{csquotes}
\usepackage{tikz}
\usepackage{textcomp}
\usepackage{textpos}
\usepackage{calc}

\usetikzlibrary{calc}

\usepackage[
	language=auto,
	autolang=other,
	backend=biber,
	style=authortitle,
	sorting=ydnt
]{biblatex}
\addbibresource{common-cog.bib}
				
\DeclareSourcemap{
	\maps[datatype=bibtex, overwrite]{
		\map{
			\step[fieldset=langid, fieldvalue=english]
			\step[fieldset=doi, null]
			\step[fieldset=issn, null]
			\step[fieldset=isbn, null]
			\step[fieldset=url, null]
			\step[fieldsource=language, fieldset=langid, origfieldval]
		}
	}
}


\graphicspath{{../../images/}} 			% Пути к изображениям

\makeatletter
\setbeamertemplate{footline}
{
	\leavevmode%
	\hbox{%
		\begin{beamercolorbox}[wd=.333333\paperwidth,ht=2.25ex,dp=1ex,center]{author
				in head/foot}%
			\usebeamerfont{author in
				head/foot}\insertshortauthor
		\end{beamercolorbox}%
		\begin{beamercolorbox}[wd=.333333\paperwidth,ht=2.25ex,dp=1ex,center]{title in
				head/foot}%
			\usebeamerfont{title in head/foot}\insertshorttitle
		\end{beamercolorbox}%
		\begin{beamercolorbox}[wd=.333333\paperwidth,ht=2.25ex,dp=1ex,right]{date in
				head/foot}%
			\usebeamerfont{date in head/foot}\insertshortdate{}\hspace*{2em}
			\insertframenumber{}\hspace*{2ex} 
		\end{beamercolorbox}
	}%
	\vskip0pt%
}

\addtobeamertemplate{frametitle}{}{
	\begin{textblock*}{100mm}(\textwidth-10pt,-25pt)
		\includegraphics[height=0.8cm]{misc/logos/hse.png}
	\end{textblock*}
}

\renewcommand*{\bibfont}{\footnotesize}
\newenvironment{proenv}{\only{\setbeamercolor{local structure}{fg=green}}}{}
\newenvironment{conenv}{\only{\setbeamercolor{local structure}{fg=red}}}{}

\begin{document}
	
	\title[ИТСАУ ФИЦ ИУ РАН]{Базовая кафедра <<Интеллектуальные технологии системного анализа и управления>> ФИЦ ИУ РАН}
	\author[Попков Ю.С., Панов А.И.]{Заведующий кафедрой\\директор ИСА РАН ФИЦ ИУ РАН, чл.-корр. РАН\\Попков Юрий Соломонович\\ Доцент кафедры\\с.н.с. ИСА РАН ФИЦ ИУ РАН, к.ф.-м.н.\\Панов Александр Игоревич}

	\date{22 сентября 2016~г.} 
	
	\begin{frame}
		\titlepage
	\end{frame}

	\begin{frame}
		\frametitle{Базовая организация}
		\begin{columns}
			\begin{column}{0.5\textwidth}
				\centering
				\includegraphics[width=0.9\textwidth]{misc/logos/frccsc}
			\end{column}
			\begin{column}{0.5\textwidth}
				\centering
				\includegraphics[width=0.9\textwidth]{misc/logos/ras}
			\end{column}
		\end{columns}
		\par\bigskip
		\par\bigskip
		Федеральный исследовательский центр <<Информатика и управление>> Российской академии наук - крупнейший научный центр в области компьютерных наук:
		\begin{itemize}
			\item Институт системного анализа (ИСА РАН),
			\item Вычислительный центр (ВЦ РАН),
			\item Институт проблем информатики (ИПИ РАН).
		\end{itemize}
	\end{frame}
	
	\begin{frame}
		\frametitle{Сотрудники}
		\scriptsize
		\begin{itemize}
			\item Директор ИСА РАН ФИЦ ИУ РАН, чл.-корр. РАН Ю.С. Попков (заведующий кафедрой),
			\item \textbf{зам. директора ФИЦ ИУ РАН, чл.-корр. РАН К.В. Рудаков,}
			\item зам. директора ФИЦ ИУ РАН, д.ф.-м. н., проф. Г.С. Осипов (заместитель заведующего кафедрой),
			\item г.н.с. ФИЦ ИУ РАН, д.ф.-м. н., проф. М.Г. Дмитриев,
			\item г.н.с. ФИЦ ИУ РАН, д.т.н. Б.С. Дарховский,
			\item \textbf{зав. отделением ФИЦ ИУ РАН, д.ф-м.н., доцент М.В. Посыпкин,}
			\item зав. сектором ФИЦ ИУ РАН, д.т.н., проф. В.Ф. Хорошевский,
			\item \textbf{зав. лаб. ФИУ ИУ РАН, к.ф-м.н. В.И. Швецов,}
			\item с.н.с. ФИЦ ИУ РАН, к.ф-м.н. А.И. Панов,
			\item \textbf{с.н.с. ФИЦ ИУ РАН, к.ф-м.н. А.Б. Мерков,}
			\item с.н.с. ФИЦ ИУ РАН, к.т.н. А.В. Булычев,
			\item с.н.с. ФИЦ ИУ РАН, к.ф-м.н. К.С. Яковлев,
			\item с.н.с. ФИЦ ИУ РАН, к.ф-м.н. Д.А. Макаров,
			\item \textbf{математик ФИЦ ИУ РАН, Ю.А. Дубнов.}
			
		\end{itemize}
	\end{frame}
	
	\begin{frame}
		\frametitle{Направления обучения и исследований}
		
		\begin{itemize}
			\item \textbf{Методы планирования целенаправленного поведения сложных технических систем.}
			\item \textbf{Технологии системного анализа и управления.}			
			\item \textbf{Системы управления сложными техническими объектами.}
			\item Приложения теории макросистем к моделированию социально-экономических объектов (демоэкономика). 
			\item Методы семантического поиска в локальных и глобальных телекоммуникационных сетях.
			\item Модели, методы и системы аналитики на знаниях. 
			\item Прогнозирование в слабо формализованных областях.
			\item Когнитивное моделирование: модели рефлексии, интроспекции и целеполагания.
		\end{itemize}
	\end{frame}
	
	\begin{frame}
		\frametitle{Партнерство: ++ для ВШЭ}
		
		\begin{itemize}
			\item У студентов есть выбор - развиваться в академическом направлении на примере успешных молодых ученых кафедры.
			\item Научный потенциал, фундаментальный научный задел - научные школы.
			\item Чтение курсов по важным разделам CS - разделы искусственного интеллекта (не только и не столько анализ данных), системный анализ, управление и др.
			\item Проектная научная работа со студентами, аспирантура.
			\item Исследовательская работа на передовом рубеже науки (публикации и конференции).
			\item Организация конференций и семинаров.
			\item Трудоустройство выпускников в наш центр.
		\end{itemize}
	\end{frame}
	
	\begin{frame}
		\frametitle{Партнерство: ++ для ФИЦ ИУ РАН}
		
		\begin{itemize}
			\item Лучшие студенты сейчас - лучшие сотрудники в дальнейшем.
			\item Студенты с высокой практической и фундаментальной подготовкой способны выполнять новые исследования (студенческая лаборатория ИИ).
		\end{itemize}
	\end{frame}
	
	\begin{frame}
		\frametitle{Позитивный опыт}
		
		\begin{itemize}
			\item Учебные курсы
			\begin{itemize}
				\item Курсы по машинному обучению (в рамках майнора).
				\item Курс по выбору <<Введение в искусственный интеллект>> (200 записавшихся в 2016 г.).
				\item МАГОЛЕГО <<Основы и методы системного анализа>>.
				\item В магистратуре <<Проблемы и методы вероятностной диагностики>>.
				\item В магистратуре <<Основы проектирования и реализации систем искусственного интеллекта>>.
			\end{itemize}
			\item Индивидуальная работа со студентами:
			\begin{itemize}
				\item Проектные, курсовые работы студентов, летняя практика.
				\item Стажировка для студентов в лабораториях (2 студента 3го курса ПМИ).
			\end{itemize}
			
			\item Научная работа
			\begin{itemize}
				\item Проект РФФИ мол\_а\_дк - планирование поведения и обучение планированию.
				\item Публикации с аффилиацией ВШЭ (5 публикаций в международных журналах и сборниках за последний год).
			\end{itemize}
		\end{itemize}
	\end{frame}	
	
	\begin{frame}
		\frametitle{Ближайшие планы}
		\small
			\begin{itemize}
				\item Новые учебные курсы:
				\begin{itemize}
					\footnotesize
					\item Когнитивные архитектуры управления робототехническими системами.
					\item Моделирование и управление робототехнических систем.
				\end{itemize}
				\item Научные исследования с привлечением студентов, аспирантов, сотрудников ФКН:
				\begin{itemize}
					\footnotesize
					\item Проект РФФИ мол\_а\_дк: <<Исследование механизмов и построение моделей обучения, основанных на знаковых представлениях, в задаче планирования коллективного поведения>> (руководитель Панов А.И.).
					\item Проект РФФИ а: <<Модели и методы решения задач интеллектуального управления коалицией сложных технических объектов>> (руководитель Макаров Д.А.).
					\item Научно-учебная группа <<Обучающиеся интеллектуальные роботы>>.
				\end{itemize}
				\item Летние и зимние стажировки для студентов в лабораториях ФИЦ ИУ РАН.
				\item Проектная работа со студентами.
				\item Курсовые и выпускные работы студентов, аспирантура.
			\end{itemize}
		
	\end{frame}		
\end{document}
	
	
