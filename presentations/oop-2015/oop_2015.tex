\documentclass[default]{beamer}
\setbeamertemplate{navigation symbols}{}

\usetheme{CambridgeUS}
\useoutertheme{infolines}
%\usecolortheme{crane}

\usepackage{cmap}	% Поддержка поиска русских слов в PDF (pdflatex)
\usepackage[T2A]{fontenc}       %поддержка кириллицы
\usepackage[utf8]{inputenc}	% Выбор языка и кодировки
\usepackage[english, russian]{babel}

\usepackage{inconsolata}
\usepackage{color}
\usepackage{listings}

\graphicspath{{../../images/oop/}} 			% Пути к изображениям

\makeatletter
\setbeamertemplate{footline}
{
	\leavevmode%
	\hbox{%
		\begin{beamercolorbox}[wd=.333333\paperwidth,ht=2.25ex,dp=1ex,center]{author
				in head/foot}%
			\usebeamerfont{author in
				head/foot}\insertshortauthor~~\beamer@ifempty{\insertshortinstitute}{}{(\insertshortinstitute)}
		\end{beamercolorbox}%
		\begin{beamercolorbox}[wd=.333333\paperwidth,ht=2.25ex,dp=1ex,center]{title in
				head/foot}%
			\usebeamerfont{title in head/foot}\insertshorttitle
		\end{beamercolorbox}%
		\begin{beamercolorbox}[wd=.333333\paperwidth,ht=2.25ex,dp=1ex,right]{date in
				head/foot}%
			\usebeamerfont{date in head/foot}\insertshortdate{}\hspace*{2em}
			\insertframenumber{}\hspace*{2ex} 
		\end{beamercolorbox}}%
		\vskip0pt%
	}

	\makeatletter
	\definecolor{pblue}{rgb}{0.13,0.13,1}
	\definecolor{pgreen}{rgb}{0,0.5,0}
	\definecolor{pred}{rgb}{0.9,0,0}
	\definecolor{pgrey}{rgb}{0.46,0.45,0.48}
	\lstset{language=Java,
		showspaces=false,
		showtabs=false,
		breaklines=true,
		showstringspaces=false,
		breakatwhitespace=true,
		commentstyle=\color{pgreen},
		keywordstyle=\color{pblue},
		stringstyle=\color{pred},
		basicstyle=\ttfamily,
		%moredelim=[il][\textcolor{pgrey}]{$$},
		moredelim=[is][\textcolor{pgrey}]{\%\%}{\%\%}	
	}

	\begin{document}
	
	\title[ООП. Лабораторные]{Основы объектно"--~ориентированного программирования.
		Лабораторные}
	\author[Панов]{Александр Панов}
	\institute[МФТИ]{Московский физико-технический институт}
	\date{февраль 2015 г.} 
	
	\begin{frame}
	\titlepage
	\end{frame}
	
	\section {Семинар 1}
	
	\begin{frame}
	\frametitle{Цели курса}
	
	\begin{itemize}
	\item Освоить идеологию объектно"--~ориентированного программирования.
	\item Понять принципы программирования структур данных и типовых решений
	(patterns).
	\item Научиться писать программы на объектно"--~ориентированном языке (Java,
	C++, Python).
	\item Начать создавать безопасные и легко понимаемые программы.
	\item Научиться работать в команде с использованием средств командной
	разработки кода.
	\item Освоить основы параллельного программирования.
	\item Начать пользоваться стандартными и сторонними библиотеками для решения
	своих задачах.
	\item Овладеть инструментами компиляции, отладки и сборки сложных программ.
	\end{itemize}
	\end{frame}
	
	\begin{frame}
	\frametitle{TIOBE Index}
	
	Индекс, оценивающий популярность языков программирования. Основан на подсчёте
	результатов поисковых запросов, содержащих название языка (Google, Blogger,
	Wikipedia, YouTube, Baidu, Yahoo!, Bing, Amazon).
	http://www.tiobe.com/index.php/content/paperinfo/tpci/index.html
	
	\begin{columns}
	\begin{column}{0.5\textwidth}
	\begin{figure}
	\includegraphics[width=0.8\textwidth]{tiobe_graph}
	\end{figure}
	\end{column}
	\begin{column}{0.5\textwidth}
	\begin{figure}
	\includegraphics[width=0.8\textwidth]{tiobe_table}
	\end{figure}
	\end{column}	
	\end{columns}
	\end{frame}
	
	\begin{frame}
	\frametitle{ООП на примере языка C++}
	
	\begin{itemize}
	\item История с 1980~г.: изначально <<C with classes>>, крайняя версия "---
	C++11.
	\item Стандартизация с 1996~г. 
	\item Ключевая особенность "--- полная совместимость с C.
	\item Высокая производительность.
	\item Наличие совместимости с C приводит к путанице при использовании
	устаревших функций.
	\item Большое количество библиотек, в том числе и с дублирующими функциями.
	\end{itemize}		
	
	\end{frame}
	
	\begin{frame}
	\frametitle{ООП на примере языка Java}
	
	\begin{itemize}
	\item История с 1995~г.: 6 версий "--- крайняя JDK 1.8.
	\item Поддержка Sun"--~Oracle http://docs.oracle.com/javase/8/docs/
	\item Ключевая особенность "--- программы транслируются в байт-код,
	выполняемый виртуальной машиной Java (JVM). JVM реализована для всех типов
	операционных систем.
	\item Облегченное управление памятью "--- сборка мусора garbage collector
	(GC).
	\item Программные стеки: JavaSE (desktop"--~приложения), JavaEE
	(web"--~приложения), JavaFX (rich"--~приложения), Android (мобильные
	приложения).
	\item Богатый набор уже написанного кода и большое количество библиотек и
	фреймворков (frameworks), решающих огромное количество задач.
	\end{itemize}
	\end{frame}
	
	\begin{frame}
	\frametitle{Литература}
	
	\bibliographystyle{gost2008p}
	\nocite{*}
	\inputencoding{cp1251}
	\bibliography{../../biblio/misc}
	\inputencoding{utf8}
	\end{frame}
	
	\begin{frame}
	\frametitle{Компоненты языка Java}
	
	\begin{figure}
	\includegraphics[width=0.8\textwidth]{java_stack}
	\end{figure}
	\end{frame}	
	
	\begin{frame}
	\frametitle{Инструменты языка C++}
	
	\begin{itemize}
	\item STandart Library (STL) "--- библиотека шаблонов.
	\item Boost "--- одна из самых известных библиотек инструментов.
	\item make "--- инструмент сборки программ.
	\item gdb "--- инструмент отладки.
	\end{itemize}
	\end{frame}	
	
	\begin{frame}
	\frametitle{Работа в семестре}
	
	\begin{itemize}
	\item Сформировать команды минимум по 3 человека, максимум "--- 5 (конец
	февраля).
	\item Определиться с языком программирования в команде и темой курсового
	проекта (конец февраля).
	\item Подготовить презентацию своего проекта (конец марта).
	\item Выполнить две семестровых задачи (конец марта).
	\item Сдать курсовой проект (май).
	\end{itemize}
	\par\bigskip
	Среда разработки и система контроля версий "--- по своему усмотрению.
	\end{frame}
	
	\begin{frame}[fragile]
	\frametitle{Примеры на Java}
	
	\begin{lstlisting}
	double a = 1, b = 1, c = 6; 
	double D = b * b - 4 * a * c; 
	if (D >= 0) { 
		double x1 = (-b + Math.sqrt (D)) / (2 * a);
		double x2 = (-b - Math.sqrt (D)) / (2 * a); 
	}
	
	int x = 2; 
	int y = 0; 
	/* if (x > 0) 
	y = y + x*2; 
	else 
	y = -y - x*4; */ 
	y = y*y;// + 2*x;
	
	\end{lstlisting}
	\end{frame}
	
	\begin{frame}[fragile]
	\frametitle{Hello World! на Java}
	
	\begin{lstlisting}
	public class Demo { 
		
	   public static void main (String args[]) {
	      System.out.println("Hello, world!");
	   }
	}
	\end{lstlisting}
	\par\bigskip
	javac Demo.java
	
	java Demo
	\end{frame}
	
	\begin{frame}
	\frametitle{Лексика языка}
	
	\begin{itemize}
		\item Идентификаторы "--- это имена,~которые даются различным элементам языка
		для упрощения доступа к ним. Имена имеют пакеты, классы, интерфейсы, поля,
		методы, аргументы и локальные переменные.
		\item Ключевые слова "--- это зарезервированные слова, состояшие из
		ASCII"--~символов и выполняющие различные задачи языка: abstract, double, int,
		class, public, void и т.~п.
		\item Литералы позволяют задать в программе значения для числовых, символьных и строковых выражений, а также null"--~литералов.
		\item Операторы используются в различных операциях "--- арифмтеических, логических, битовых, опреациях сравнения и присваивания: =, ==, >, <, +, - и т.~п.
	\end{itemize}
	\end{frame}

	\begin{frame}
		\frametitle{Интернет на виртуальных контейнерах}
		
		ping 64.0.0.0 -c 2 -w2 || wget -qO - ``login.telecom.mipt.ru/bin/login.cgi?login=LOGIN\&memorize=on\&password=
		\$((wget login.telecom.mipt.ru/bin/getqc.cgi -qO -; echo -n PASSWORD) | md5sum - | head -c32)''
		
	\end{frame}
	
	%	\begin{frame}
	%		\frametitle{Цели курса}
	%		
	%		\begin{itemize}
	%			\item
	%		\end{itemize}
	%	\end{frame}
	
\end{document}
	
	
