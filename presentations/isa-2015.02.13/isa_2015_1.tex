% Sources: 
% 1) http://mechanoid.kiev.ua/neural-net-art1.html
% 2) http://alife.narod.ru/lectures/neural/Neu_ch11.htm
% 3) http://www.intuit.ru/studies/courses/88/88/info
% 4) http://scorcher.ru/neuro/science/attention/mem19.php
\documentclass[default]{beamer}
\setbeamertemplate{navigation symbols}{}

\usetheme{CambridgeUS}
\useoutertheme{infolines}
%\usecolortheme{crane}

\usepackage{cmap}							% Поддержка поиска русских слов в PDF (pdflatex)
\usepackage[T2A]{fontenc}       			%поддержка кириллицы
\usepackage[utf8]{inputenc}					% Выбор языка и кодировки
\usepackage[english, russian]{babel}

\graphicspath{{../../images/mpf/}} 			% Пути к изображениям

\makeatletter
\setbeamertemplate{footline}
{
	\leavevmode%
	\hbox{%
		\begin{beamercolorbox}[wd=.333333\paperwidth,ht=2.25ex,dp=1ex,center]{author
				in head/foot}%
			\usebeamerfont{author in
				head/foot}\insertshortauthor~~\beamer@ifempty{\insertshortinstitute}{}{(\insertshortinstitute)}
		\end{beamercolorbox}%
		\begin{beamercolorbox}[wd=.333333\paperwidth,ht=2.25ex,dp=1ex,center]{title in
				head/foot}%
			\usebeamerfont{title in head/foot}\insertshorttitle
		\end{beamercolorbox}%
		\begin{beamercolorbox}[wd=.333333\paperwidth,ht=2.25ex,dp=1ex,right]{date in
				head/foot}%
			\usebeamerfont{date in head/foot}\insertshortdate{}\hspace*{2em}
			\insertframenumber{}\hspace*{2ex} 
		\end{beamercolorbox}
	}%
	\vskip0pt%
}

\setbeamertemplate{bibliography entry title}{}
\setbeamertemplate{bibliography entry location}{}
\setbeamertemplate{bibliography entry year}{}
\setbeamertemplate{bibliography entry note}{}
\bibliographystyle{gost2008p}

\begin{document}
	
	\title[ART Гроссберга]{Теория адаптивного резонанса Гроссберга}
	\author[Панов]{Александр Панов}
	\institute[ИСА РАН]{ИСА РАН}
	\date{13 февраля 2015~г.} 
	
	\begin{frame}
		\titlepage
	\end{frame}
	
	\begin{frame}
		\frametitle{Гроссберг}
		
		\begin{itemize}
			\item Стефан Гроссберг, 75 лет "--- специалист по когнитивным наукам, нейрофизиолог,~математик.
			\item Профессор Бостонского университета.
			\item Scopus: более 380 статей, h-индекс "--- 64.
			\item Работы с количеством цитирований более 1000:

			{
				\scriptsize
				\nocite{*}
				\inputencoding{cp1251}
				\bibliography{../../biblio/ikm}
				\inputencoding{utf8}
			}
		\end{itemize}
	\end{frame}
	
	\begin{frame}
		\frametitle{Основные проблемы}
		
		\begin{itemize}
			\item Проблема соотношения стабильности и пластичности в восприятии.
			\item Принцип дополняющих когнитивных процессов.
			\item Принцип послойной организации всех когнитивных процессов.
			\item Все состояния сознания являются состояниями резонанса.
		\end{itemize}
		
		\begin{columns}
			\begin{column}{0.5\textwidth}
				\begin{figure}
					\includegraphics[width=\textwidth]{grossberg_compl}
				\end{figure}
			\end{column}
			\begin{column}{0.5\textwidth}
				\begin{figure}
					\includegraphics[width=\textwidth]{art_illus}
				\end{figure}
			\end{column}
		\end{columns}
	\end{frame}	

	\begin{frame}
		\frametitle{Что, где и как в коре головного мозга}
		
		\begin{columns}
			\begin{column}{0.6\textwidth}
				\footnotesize
				Дорсальный (Что?) и вентральный (Где?, Как?) пути обработки информации.
				\begin{itemize}
					\item Retina/LGN "--- сетчатка/боковое коленчатое ядро.
					\item V1, V2, V3, V4 "--- первичная и вторичные зоны зрительной коры.
					\item IT "--- нижневисочная (инферотемпоральная) зона коры.
					\item ORB "--- глазнично"--~лобная (орбитофронтальная) зона коры.
					\item LIP "--- латеральная межтеменная (интрапариетальная) зона коры.
					\item PPC "--- задняя теменная (париетальная) зона коры.
					\item PFC "--- передняя лобная зона коры.
					\item FEF "--- глазодвигательная кора.
				\end{itemize}
			\end{column}
			\begin{column}{0.4\textwidth}
				\vspace*{-5mm}
				\begin{figure}
					\includegraphics[width=0.9\textwidth]{grossberg_streams}
				\end{figure}
				
			\end{column}
		\end{columns}
	\end{frame}

	\begin{frame}
		\frametitle{Нейронная сеть ART-1}
		\onslide<1->{
			\begin{figure}
				\includegraphics[width=0.35\textwidth]{art_1}
			\end{figure}
		}
		\begin{overlayarea}{\textwidth}{.5\textheight}
			\only<1>{
				\begin{itemize}
					\item Четыре функциональных модуля: слой сравнения, слой распознавания и два управляющих нейрона: сброса $R$ и управления $G$.
					\item Вектора $X, Y^0, Y^1$ "--- двоичные вектора.
					\item Для нейронов слоя сравнения используется правило $2/3$.
				\end{itemize}
			}
			\only<2>{
				\footnotesize
				\begin{itemize}
					\item \textbf{Начальная фаза}: $G=0$ при $X=\{0,\dots,0\}$; при поступлении ненулевого $X$ принимаем $G=1$, $Y^0=X$ и определяем $Y^1=\{y_1^1,\dots,y_n^1\}$, $\forall i\not =w\ y_i^1=0, y_w^1=1$, где $w$ "--- индекс нейрона-победителя ($\max_j\sum_i y_i^0*b_{ij}$) в слое распознавания, обнуляем управление $G=0$.
					\item \textbf{Фаза сравнения}: Новый выход слоя сравнения $Y^0=\{t_{1w},\dots,t_{mw}\}\wedge X$; если $\|Y^0\|/\|X\|>\rho$, то завершаем процесс, иначе возникает сигнал сброса, подавляется нейрон-победитель и начинается фаза поиска.
					\item \textbf{Фаза поиска}: так как нейрон победитель подавлен, то $R$ обнуляется, $G=0$, $Y^0=X$, определяется новый нейрон-победитель и снова переходим к фазе сравнения.
				\end{itemize}
			}
			\only<3>{
				\footnotesize
				Окончание итерационного процесса:
				\begin{itemize}
					\item Найдётся запомненная категория, сходство которой с входным вектором X будет достаточным для успешной классификации. После этого запускается \textbf{фаза обучения}, в котором модифицируются веса $b_{iw}$ и $t_{iw}$ матриц $B$ и $T$ для победившего нейрона.
					\item В процессе поиска все запомненные категории окажутся проверенными, но ни одна из них не дала требуемого сходства. В этом случае входной образ $X$ объявляется новым для нейросети, и ему выделяется новый нейрон в слое распознавания.
				\end{itemize}
			}	
		\end{overlayarea}
	
	\end{frame}

	\begin{frame}
		\frametitle{Фаза обучения}
		
		\begin{itemize}
			\item Начальный значения матриц $B$ и $T$ должны удовлетворять соотношениям: $b_{ij}<L/(L-1+m)$, $t_{ij}=1$, $m$ "--- размерность входных данных (вектора $X$).
			\item При нахождении подходящей категории по вектору $Y^0=\{y_1^0,\dots,y_m^0\}$ $b_{ij}=(L*y_i^0)/(L-1+\sum_k y_k^0)$, $\forall i\ t_{ij}=y_i^0$.
			\item Обучение, таким образом, сопровождается занулением все большего числа компонент матрицы $T$, оставшиеся ненулевыми компоненты определяют множество критических черт найденных категории.
		\end{itemize}
		\begin{figure}
			\includegraphics[width=0.3\textwidth]{art_learn}
		\end{figure}
	\end{frame}

	\begin{frame}
		\frametitle{Некоторые свойства модели}
		
		\begin{itemize}
			\item По достижении стабильного состояния в результате обучения предъявление входного вектора будет сразу приводить к правильной классификации без фазы поиска, на основе прямого доступа.
			\item Процесс поиска устойчив.
			\item Процесс обучения устойчив. Обучение весов нейрона"--~победителя не приведёт в дальнейшем к переключению на другой нейрон.
			\item Процесс обучения конечен. Итоговое состояние для заданного набора образов будет достигнуто за конечное число итерации, при этом дальнейшее предъявление этих образов не вызовет циклических изменений значений весов.
		\end{itemize}
	\end{frame}
	
	\begin{frame}
		\frametitle{Принцип многослойности}
		
		\begin{figure}
			\includegraphics[width=0.45\textwidth]{art_neuro}
		\end{figure}
	\end{frame}
			
	\begin{frame}
		\frametitle{Дальнейшее развитие}
		
		\begin{itemize}
			\item ART-2 "--- переход от двоичных значений к непрерывным (действительным).
			\item ART-3 "--- добавление нескольких слоёв для сжатия информации и абстрагирования. Резонанс возникает на некотором уровне иерархии. Введён механизм рефрактерного торможения.
			\item FuzzyART "--- введение элементов нечёткой логики.
			\item ARTMAP, pART "--- комбинация двух блоков ART-1 и ART-2 для регулировки параметра сброса (чувствительности).
			\item Много прикладных моделей: dARTEX, ARTPHONE и др.
		\end{itemize}
	\end{frame}
	%	\begin{frame}
	%		\frametitle{Цели курса}
	%		
	%		\begin{itemize}
	%			\item
	%		\end{itemize}
	%	\end{frame}
	
\end{document}
	
	
