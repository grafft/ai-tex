	\section{Модели когнитивных функций}
	
	\begin{frame}
		\small
		\tableofcontents
	\end{frame}
	
	\begin{frame}
		\frametitle{Образование нового знака}
		\centering
		\includegraphics[width=0.5\textwidth]{signs/en/sign_naming_colored_en}
	\end{frame}		
	
	\begin{frame}
		\frametitle{Отношения на множестве компонент знака}
		\centering
		\includegraphics[page=1,width=0.8\textwidth]{sign-schemas/sign_relations}
		
		Сходство образов
	\end{frame}	
	
	\begin{frame}
		\frametitle{Отношения на множестве компонент знака}
		
		\begin{columns}
			\begin{column}{0.3\textwidth}
				\centering
				Включение образов 
				\par\bigskip
				\par\bigskip
				\par\bigskip
				\par\bigskip
				\par\bigskip
				Противопоставление образов 
				
			\end{column}
			\begin{column}{0.7\textwidth}
				\includegraphics[page=2,width=0.8\textwidth]{sign-schemas/sign_relations}
				
				\includegraphics[page=3,width=0.8\textwidth]{sign-schemas/sign_relations}
			\end{column}
		\end{columns}
		
	\end{frame}	
	
	\begin{frame}
		\frametitle{Отношения на множестве компонент знака}
		\centering
		\includegraphics[page=4,width=0.7\textwidth]{sign-schemas/sign_relations}
		
		Сценарий на значениях
	\end{frame}	
	
	\begin{frame}
		\frametitle{Отношения на множестве компонент знака}
		
		\begin{columns}
			\begin{column}{0.3\textwidth}
				\centering
				Поглощение личностных смыслов 
				\par\bigskip
				\par\bigskip
				\par\bigskip
				\par\bigskip
				\par\bigskip
				Противопоставление личностных смыслов 
				
			\end{column}
			\begin{column}{0.7\textwidth}
				\includegraphics[page=5,width=0.8\textwidth]{sign-schemas/sign_relations}
				\par\bigskip
				\includegraphics[page=6,width=0.8\textwidth]{sign-schemas/sign_relations}
			\end{column}
		\end{columns}
	\end{frame}	

	\subsection{Операции в семиотической сети}	

	\begin{frame}
		\frametitle{Операции на множестве компонент знака}
		
		\begin{columns}
			\begin{column}{0.5\textwidth}
				\centering
				\includegraphics[page=7,width=\textwidth]{sign-schemas/sign_relations}
				\par\bigskip
				Замыкание по значениям
			\end{column}
			\begin{column}{0.5\textwidth}
				\centering
				\includegraphics[page=8,width=\textwidth]{sign-schemas/sign_relations}
				\par\bigskip
				Агглютинация личностных смыслов
			\end{column}
		\end{columns}
		\nocite{*}
		\printbibliography[keyword={signoper}, resetnumbers=true]
	\end{frame}	
	
	\begin{frame}
		\frametitle{Модель функции планирование поведения}
		\centering
		\includegraphics[width=0.9\textwidth]{algo/ru/plan_alg_ru}
		\nocite{*}
		\printbibliography[keyword={signb}, resetnumbers=true]
		\nocite{*}
		\printbibliography[keyword={symbsign}, resetnumbers=true]
	\end{frame}	
	
	\subsection{Планирование поведения}
	\begin{frame}
		\frametitle{Алгоритм планирования поведения}
		
		\begin{columns}
			\begin{column}{0.6\textwidth}
				\includegraphics[width=\textwidth]{algo/ru/beh_plan2_ru}
				\vspace{10pt}
				\nocite{*}
				\printbibliography[keyword={plan}, resetnumbers=true]
				\printbibliography[keyword={causnet}]
			\end{column}
			\begin{column}{0.4\textwidth}
				\scriptsize
				Иерархический процесс планирования начинается с конченой ситуации и стремится достичь начальной ситуации.
				\par\bigskip
				MAP-итерация:
				\begin{itemize}
					\item \textit{S-step} -- поиск прецедентов выполнения действия в текущих условиях,
					\item \textit{M-step} -- поиск применимых действий на сети значений,
					\item \textit{A-step} -- генерация личностных смыслов, соответствующих найденным значениям,
					\item \textit{P-step} -- построение новой текущей ситуации по множеству признаков условий найденных действий.
					
				\end{itemize}
			\end{column}
		\end{columns}
		
	\end{frame}		
	
	\subsection{Примеры}
	\begin{frame}
		\frametitle{Пример: фрагмент сети на значениях}
		\begin{columns}
			\begin{column}{0.7\textwidth}
				\centering
				\includegraphics[page=2,width=\textwidth]{examples/plan/plan_nets}
			\end{column}
			\begin{column}{0.3\textwidth}
				\centering
				\includegraphics[page=1,width=\textwidth]{examples/plan/block_world}
			\end{column}
		\end{columns}
	\end{frame}	
	
	\begin{frame}
		\frametitle{Пример: сеть на смыслах - начальная ситуация}
		
		\centering
		\includegraphics[page=3,width=0.7\textwidth]{examples/plan/plan_nets}
		\par\bigskip
		\includegraphics[page=2,width=0.5\textwidth]{examples/plan/block_world}
		
		
	\end{frame}	
	
	\begin{frame}
		\frametitle{Пример: сеть на смыслах - целевая ситуация}
		\begin{columns}
			\begin{column}{0.7\textwidth}
				\centering
				\includegraphics[page=1,width=\textwidth]{examples/plan/plan_nets}
			\end{column}
			\begin{column}{0.3\textwidth}
				\centering
				\includegraphics[page=1,width=\textwidth]{examples/plan/block_world}
			\end{column}
		\end{columns}
	\end{frame}	
	
	\begin{frame}
		\frametitle{Пример: фрагмент сети на значениях}
		\begin{columns}
			\begin{column}{0.7\textwidth}
				\centering
				\includegraphics[page=5,width=\textwidth]{examples/plan/plan_nets}
			\end{column}
			\begin{column}{0.3\textwidth}
				\centering
				\includegraphics[page=3,width=\textwidth]{examples/plan/block_world}
			\end{column}
		\end{columns}
	\end{frame}
	
	\begin{frame}
		\frametitle{Пример: генерация личностного смысла}
		
		\begin{tikzpicture}[overlay,remember picture,xshift=165pt,yshift=-50pt]
		\onslide<1->{
			\tikzstyle{ell}=[draw, thick, align=center, color=blue]
			\tikzstyle{ellf}=[draw, thick, align=center, color=blue, fill=blue]
			
			\node[ell, ellipse, minimum height = 30, minimum width = 100] (block) at (5 pt,0){};
			\predmatr{-30}{0}{block1}
			\predmatr{-10}{0}{block2}
			\predmatr{10}{0}{block3}	
			\predmatr{30}{0}{block4}
			\node at (35 pt, 20 pt) {``block''};
			
			\node[ell, ellipse, minimum height = 20, minimum width = 40] (c) at (32.5 pt, -50 pt){};
			\predmatr{30}{-50}{c1}
			\node at (10 pt, -35 pt) {``c''};		
			
			\node[ell, ellipse, minimum height = 20, minimum width = 40] (d) at (82.5 pt, -50 pt){};
			\predmatr{80}{-50}{d1}			
			\node at (60 pt, -35 pt) {``d''};	
			
			\node[ell, ellipse, minimum height = 20, minimum width = 40] (x) at (-37.5 pt, 50 pt){};
			\predmatr{-40}{50}{x1}
			\node at (-85 pt, 50 pt) {``block?x''};			
			
			\node[ell, ellipse, minimum height = 20, minimum width = 40] (y) at (42.5 pt, 50 pt){};
			\predmatr{40}{50}{y1}	
			\node at (85 pt, 50 pt) {``block?y''};
			
			\path[-latex'] (c.north) edge [out = 90, in = -80] node[above, black] {\scriptsize 1} node[above, black, near start] {\scriptsize 1} ([xshift=-3]block3.south);
			\path[-latex'] (d.north) edge [out = 90, in = -80] node[above, black, near end] {\scriptsize 1} node[above, black, near start] {\scriptsize 1} ([xshift=-3]block4.south);	
		}
		\onslide<1>{
			\node[ell, ellipse, minimum height = 20, minimum width = 40] (a) at (-77.5 pt, -50 pt){};
			\predmatr{-80}{-50}{a1}
			\node at (-100 pt, -35 pt) {``a''};
			
			\node[ell, ellipse, minimum height = 20, minimum width = 40] (b) at (-27.5 pt, -50 pt){};
			\predmatr{-30}{-50}{b1}
			\node at (-50 pt, -35 pt) {``b''};		
		}
		\onslide<1-2>{
			\path[-latex'] (a.north) edge [out = 90, in = -120] node[above, black, near end] {\scriptsize 1} node[above, black, near start] {\scriptsize 1} ([xshift=-3]block1.south);
			\path[-latex'] (b.north) edge [out = 90, in = -120] node[above, black, near end] {\scriptsize 1} node[above, black, near start] {\scriptsize 1} ([xshift=-3]block2.south);
		}			
		\onslide<1-3>{
			\path[-latex'] ([xshift=-10]block.north) edge [out = 90, in = -80] node[above, black] {\scriptsize 1} node[above, black, near start] {\scriptsize 0} ([xshift=-3]x1.south);
			\path[-latex'] ([xshift=10]block.north) edge [out = 90, in = -100] node[above, black] {\scriptsize 1} node[above, black, near start] {\scriptsize 0} ([xshift=-3]y1.south);
		}
		\onslide<1-4>{
			\node[ell, ellipse, minimum height = 20, minimum width = 40] (unstack) at (2.5 pt, 100 pt){};
			\predmatr{0}{100}{unstack1}
			\node at (-15 pt, 120 pt) {``unstack''};
			
			\node[ell, ellipse, minimum height = 20, minimum width = 40] (on) at (-77.5 pt, 100 pt){};
			\predmatr{-80}{100}{on1}
			\node at (-115 pt, 100 pt) {``on''};
			
			\path[-latex'] (x.north) edge [out = 90, in = -100] node[above, black, near end] {\scriptsize -1} node[above, black, near start] {\scriptsize 1} ([xshift=2]unstack1.south);
			\path[-latex'] (y.north) edge [out = 90, in = -80] node[above, black, near end] {\scriptsize -2} node[above, black, near start] {\scriptsize 1} ([xshift=5]unstack1.south);
			\path[-latex'] (on.east) edge [out = 0, in = 180] node[above, black, near end] {\scriptsize 1} node[above, black, near start] {\scriptsize 1} ([yshift=-3]unstack1.west);
			
			\path[-latex'] ([xshift=-10]x.north) edge [out = 100, in = -140] node[above, black] {\scriptsize 1} node[above, black, very near start] {\scriptsize 1} ([xshift=-3]on1.south);	
			\path[-latex'] ([xshift=-10]y.north) edge [out = 150, in = -30] node[above, black, near end] {\scriptsize -1} node[above, black, very near start] {\scriptsize 1} ([xshift=3]on1.south);					
		}			
		\onslide<1-5>{
			\node[ell, ellipse, minimum height = 20, minimum width = 40] (holding) at (82.5 pt, 120 pt){};
			\predmatr{80}{120}{holding1}
			\node at (125 pt, 120 pt) {``holding''};
			
			\node[ell, ellipse, minimum height = 20, minimum width = 40] (clear) at (82.5 pt, 90 pt){};
			\predmatr{80}{90}{clear1}
			\node at (120 pt, 90 pt) {``clear''};
			
			\path[-latex'] (clear.west) edge [out = 180, in = 0] node[above, black, near end] {\scriptsize -2} node[above, black, near start] {\scriptsize 1} ([yshift=-3]unstack1.east);
			\path[-latex'] (holding.west) edge [out = 180, in = 0] node[above, black, near end] {\scriptsize -1} node[above, black, near start] {\scriptsize 1} ([yshift=3]unstack1.east);
			
		}
		\onslide<2->{
			\tikzstyle{ell}=[draw, thick, align=center, color=green!70!black]
			\tikzstyle{ellf}=[draw, thick, align=center, color=green!70!black, fill=green!70!black]
			
			\node[ell, ellipse, minimum height = 20, minimum width = 40] (a) at (-77.5 pt, -50 pt){};
			\predmatr{-80}{-50}{a1}
			\node at (-100 pt, -35 pt) {``a''};
			
			\node[ell, ellipse, minimum height = 20, minimum width = 40] (b) at (-27.5 pt, -50 pt){};
			\predmatr{-30}{-50}{b1}
			\node at (-50 pt, -35 pt) {``b''};
		}
		\onslide<3->{
			\path[-latex', very thick] (a.north) edge [out = 90, in = -120] node[above, black, near end] {\scriptsize 1} node[above, black, near start] {\scriptsize 1} ([xshift=-3]block1.south);
			\path[-latex', very thick] (b.north) edge [out = 90, in = -120] node[above, black, near end] {\scriptsize 1} node[above, black, near start] {\scriptsize 1} ([xshift=-3]block2.south);	
		}
		\onslide<4->{
			\path[-latex', very thick] ([xshift=-10]block.north) edge [out = 90, in = -80] node[above, black] {\scriptsize 1} node[above, black, near start] {\scriptsize 0} ([xshift=-3]x1.south);
			\path[-latex', very thick] ([xshift=10]block.north) edge [out = 90, in = -100] node[above, black] {\scriptsize 1} node[above, black, near start] {\scriptsize 0} ([xshift=-3]y1.south);
		}
		\onslide<5->{
			\node[ell, ellipse, minimum height = 20, minimum width = 40] (unstack) at (2.5 pt, 100 pt){};
			\predmatr{0}{100}{unstack1}
			\node at (-15 pt, 120 pt) {``unstack''};
			
			\node[ell, ellipse, minimum height = 20, minimum width = 40] (on) at (-77.5 pt, 100 pt){};
			\predmatr{-80}{100}{on1}
			\node at (-115 pt, 100 pt) {``on''};
			
			\path[-latex', very thick] (on.east) edge [out = 0, in = 180] node[above, black, near end] {\scriptsize 1} node[above, black, near start] {\scriptsize 1} ([yshift=-3]unstack1.west);
			
			\path[-latex', very thick] ([xshift=-10]x.north) edge [out = 100, in = -140] node[above, black] {\scriptsize 1} node[above, black, very near start] {\scriptsize 1} ([xshift=-3]on1.south);	
			\path[-latex', very thick] ([xshift=-10]y.north) edge [out = 150, in = -30] node[above, black, near end] {\scriptsize -1} node[above, black, very near start] {\scriptsize 1} ([xshift=3]on1.south);
			\path[-latex', very thick] (x.north) edge [out = 90, in = -100] node[above, black, near end] {\scriptsize -1} node[above, black, near start] {\scriptsize 1} ([xshift=2]unstack1.south);
			\path[-latex', very thick] (y.north) edge [out = 90, in = -80] node[above, black, near end] {\scriptsize -2} node[above, black, near start] {\scriptsize 1} ([xshift=5]unstack1.south);	
		}
		\onslide<6->{
			\node[ell, ellipse, minimum height = 20, minimum width = 40] (holding) at (82.5 pt, 120 pt){};
			\predmatr{80}{120}{holding1}
			\node at (125 pt, 120 pt) {``holding''};
			
			\node[ell, ellipse, minimum height = 20, minimum width = 40] (clear) at (82.5 pt, 90 pt){};
			\predmatr{80}{90}{clear1}
			\node at (120 pt, 90 pt) {``clear''};
			
			\path[-latex', very thick] (clear.west) edge [out = 180, in = 0] node[above, black, near end] {\scriptsize -2} node[above, black, near start] {\scriptsize 1} ([yshift=-3]unstack1.east);
			\path[-latex', very thick] (holding.west) edge [out = 180, in = 0] node[above, black, near end] {\scriptsize -1} node[above, black, near start] {\scriptsize 1} ([yshift=3]unstack1.east);
		}			
		\end{tikzpicture}
	\end{frame}
	
	\begin{frame}
		\frametitle{Пример: текущая ситуация}
		\begin{columns}
			\begin{column}{0.7\textwidth}
				\centering
				\includegraphics[page=4,width=\textwidth]{examples/plan/plan_nets}
			\end{column}
			\begin{column}{0.3\textwidth}
				\centering
				\includegraphics[page=3,width=\textwidth]{examples/plan/block_world}
			\end{column}
		\end{columns}
	\end{frame}
	\subsection{Обучение и целеполагание}
	\begin{frame}
		\frametitle{Обучение в процессе планирования}
		
		Образование нового правила и сохранение ситуаций - образование новых каузальных матриц
		\par\bigskip
		\centering
		\includegraphics[page=6,width=0.6\textwidth]{examples/plan/plan_nets}
		\includegraphics[page=7,width=0.4\textwidth]{examples/plan/plan_nets}
	\end{frame}

	\begin{frame}
		\frametitle{Этап целеполагания}
		\begin{center}
			\includegraphics[width=0.8\textwidth]{algo/ru/gmap_ru}
		\end{center}
		\nocite{*}
		\printbibliography[keyword={goalres}, resetnumbers=true]
	\end{frame}
