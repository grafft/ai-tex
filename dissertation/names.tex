%%% Переопределение именований %%%
\renewcommand{\abstractname}{Аннотация}
\renewcommand{\alsoname}{см. также}
\renewcommand{\appendixname}{Приложение}
\renewcommand{\ccname}{исх.}
\renewcommand{\chaptername}{Глава}
\renewcommand{\contentsname}{Оглавление}
\renewcommand{\enclname}{вкл.}
\renewcommand{\figurename}{Рисунок}
\renewcommand{\headtoname}{вх.}
\renewcommand{\indexname}{Предметный указатель}
\renewcommand{\pagename}{Стр.}
\renewcommand{\partname}{Часть}
\renewcommand{\seename}{см.}
\renewcommand{\tablename}{Таблица}
\renewcommand{\appendixname}{Приложение}

\floatname{algorithm}{Алгоритм}
\algrenewcommand\algorithmicrequire{\textbf{Вход:}}
\algrenewcommand\algorithmicensure{\textbf{Выход:}}
\algrenewcommand\algorithmicforall{\textbf{для всех}}
\algrenewcommand\algorithmicwhile{\textbf{пока}}
\algrenewcommand\algorithmicif{\textbf{если}}
\algrenewcommand\algorithmicthen{\textbf{то}}
\algrenewcommand\algorithmicelse{\textbf{иначе}}
\algrenewcommand\algorithmicreturn{\textbf{вернуть}}
\algrenewcommand\algorithmicdo{}
\renewcommand{\algorithmiccomment}[1]{{\quad\sl // #1}}

%%% Теормоподобные окружения %%%
\theoremstyle{plain}
\newtheorem{Theorem}{Теорема}
\newtheorem{Lemma}[Theorem]{Лемма}
\newtheorem{Pred}{Утверждение}
\newtheorem{Corollary}{Следствие}
\newtheorem{Def}{Определение}
\newtheorem{Axiom}{Аксиома}
\newtheorem{Hypothesis}{Гипотеза}
\newtheorem{Assumption}{Предположение}
\newtheorem{Problem}{Задача}
\newtheorem{Example}{Пример}
\newtheorem{Fact}{Факт}
\newtheorem{Remark}{Замечание}
\newtheorem{Rule}{Правило}
\newtheorem{Condition}{Условие}
\newenvironment{Proof}%
{\par\noindent{\bf Доказательство.}}%
{\hfill$\scriptstyle\blacksquare$}

