\chapter*{Введение}							% Заголовок
\addcontentsline{toc}{chapter}{Введение}	% Добавляем его в оглавление
\textbf{Актуальность темы исследования.} 

Исследования картин мира (КМ) субъектов деятельности принадлежат одному из центральных направлений в когнитивной психологии. Высшие психические функции, в том числе связанные с приобретением и использованием знаний, являются, в широком смысле, продуктом работы КМ субъекта. Исследованию большого числа процессов, протекающих в КМ, в~том числе высших когнитивных, таких как категоризация и обобщение, целеполагание, планирование, принятие решения, творческие синтез и анализ, было посвящено значительное число работ на протяжении всей истории психологической науки. Следует отметить работы по восприятию Дж.\,А.~Фодора (J.\,A.~Fodor), Б.~Юлеза (B.~Julesz), Дж.\,Е.~Каттинга (J.\,E.~Cutting), С.~Гроссберга (S.~Grossberg), А.\,Р.~Лурия, Б.\,М.~Величковского, В.\,П.~Зинченко и памяти С.~Стернберга (S.~Sternberg), Л.~Джакоби (L.~Jacoby), Р.~Аткинсона (R.~Atkinson), Р.~Шиффрина (R.~Shiffrin), Е.~Тулвинга (E.~Tulving).

В последнее время исследованию когнитивных функций человека уделяется большое внимание не только в самой психологии, но и в нейрофизиологии и в~искусственном интеллекте. Нейрофизиологи основной своей задачей ставят поиск нейронного субстрата психических функций. При этом в качестве основного инструмента здесь выступает картирование участков коры головного мозга и отслеживание динамики активности различных участков при выполнении той или иной когнитивной задачи. Большое количество накопленного фактического материала используется для подтверждения целого ряда разрозненных моделей отдельных психических функций. Примерами могут служить работы по моделям внимания Я.\,Б.~Казановича, С.~Фринтропа (S.~Frintrop), С.~Коха (C.~Koch), Л.~Итти (L.~Itti), Дж.\,К.~Сосоза (J.\,K.~Tsotsos), А.~Торралба (A.~Torralba), Л.~Жэнга (L.~Zhang), Р.\,А.~Ренсинка (R.\,A.~Rensink). Единого аппарата для построения таких моделей на данный момент не существует, хотя имеется ряд работ Б.\,Дж.~Баарса (B.\,J.~Baars), Р.~Сана (R.~Sun), Дж.~Хокинса (J.~Hawkins), которые можно считать первыми попытками их создания.

Искусственный интеллект в начале своего становления как науки использовал для построения интеллектуальных алгоритмов данные психологов. Однако спустя некоторое время психологические соображения уже перестали рассматриваться как определяющие при разработке того или иного алгоритма. Центральное место стали занимать вопросы вычислительной эффективности и специализации в той или иной предметной области. В~связи с~тем, что в~большинстве интеллектуальных систем в~настоящее время требуется всё большая степень универсальности и автономности, начинается процесс возвращения к психологическим основам строения психики человека. Возникает задача построения моделей процессов, например, распознавания и планирования, на некоторой <<биологически инспирированной основе>>. К этому направлению относятся работы Дж.\,Р.~Андерсона (J.\,R.~Anderson), П.~Леирда (J.\,E.~Laird), П.~Ленгли (P.~Langley). Подтверждением повышенного интереса к этой теме служат организуемые в последнее время конференции и издаваемые журналы, посвящённые исключительно <<биологически правдоподобным>> архитектурам (например, ежегодные конференции BICA (Annual International Conference on Biologically Inspired Cognitive Architectures)\cite{BICAC2014} и журнал BICA \cite{BICAJ2014}).

Потребность в единой модели КМ субъекта деятельности для нейрофизиологов и исследователей в области искусственного интеллекта определяет актуальность данной работы. Такая модель требуется как для построения моделей когнитивных функций человека на нейронном уровне, подтверждаемых нейрофизиологическими данными о строении высшей нервной системы человека и данными об активности соответствующего определённой функции участка коры головного мозга, так и для построения абстрагированных от того или иного субстрата интеллектуальных алгоритмов, которые могли бы быть использованы в автономных системах свободной конфигурации.

Один из основных вопросов, возникающих при разработке модели КМ, заключается в описании базовых элементов картины мира и построении алгоритма их формирования в процессе деятельности субъекта, носителя КМ. В качестве психологической основы для построения модели элемента КМ были использованы, с одной стороны, культурно"--~исторический подход Л.\,Н.~Выготского и теория деятельности А.\,Н.~Леонтьева, с другой стороны "--- идеи прикладной семиотики, предложенные в работах Д.\,А.~Поспелова, А.~Мейстеля, Г.\,С.~Осипова. В качестве нейрофизиологических предпосылок были использованы концепции и нейронные схемы Д.~Георга (D.~George).

\textbf{Предмет исследования} "--- построение знаковых моделей картины мира и некоторых когнитивных функций субъекта деятельности.

\textbf{Целью исследования} является разработка моделей и алгоритмов формирования элементов знаковой картины мира, обладающих структурой, необходимой для построения моделей высших когнитивных функций, в том числе восприятия, внимания, планирования и целеполагания.

Для~достижения цели работы были поставлены следующие \textbf{задачи}:
\begin{enumerate}
  \item исследовать модель элемента картины мира субъекта, построенную на основе психологической теории деятельности,
  \item построить модель структурных компонент элемента картины мира, опирающуюся на нейрофизиологические данные, и исследовать её,
  \item исследовать структуру отношений и процессы самоорганизации на множестве элементов картины мира на синтаксическом уровне,
  \item исследовать процесс формирования и связывания основных компонент нового элемента картины мира и построить соответствующий алгоритм,
  \item исследовать сходимость процесса формирования и связывания основных компонент нового элемента картины мира.
\end{enumerate}

\textbf{Научная новизна и результаты, выносимые на~защиту.}
\begin{enumerate}
	\renewcommand\labelenumi{\theenumi.}
  \item Впервые построена модель структурных компонент элемента картины мира субъекта деятельности.
  \item Построены операторы распознавания в статическом, динамическом и иерархическом случаях в терминах алгебраической теории для образной компоненты элемента картины мира.
  \item Доказаны теоремы корректности линейных замыканий множеств построенных в работе операторов распознавания.
  \item Построен алгоритм формирования и связывания основных компонент нового элемента картины мира.
  \item Проведено исследование процесса формирования и связывания основных компонент нового элемента картины мира.
\end{enumerate}

\textbf{Практическая значимость.} Построение модели элементов картины мира субъекта деятельности, с~одной стороны, позволит создать универсальные интеллектуальные алгоритмы планирования поведения, целеполагания, локализации, распознавания и категоризации, применение которых в интеллектуальных системах повысит степень их автономности, а с~другой стороны, позволит объяснить некоторые патологические явления в мозге человека и дать рекомендации к их устранению.

\textbf{Методы исследования.} Теоретические результаты работы получены и обоснованы с использованием методов теории множеств, алгебраической теории распознавания образов, теории интеллектуальных динамических систем, теории деятельности.

\textbf{Достоверность результатов} подтверждена строгими математическими доказательствами утверждений и результатами вычислительных экспериментов.

\textbf{Апробация результатов исследования.}

Основные результаты работы докладывались~на: Международных конференциях по когнитивной науке (Томск, 2010~г.; Калининград, 2012~г., 2014~г.), II~Всероссийской научной конференции молодых учёных с международным участием <<Теория и практика системного анализа>> (Рыбинск, 2012~г.), IV~Международной конференции <<Системный анализ и информационные технологии>> (Абзаково, 2011~г.), V~съезде Общероссийской общественной организации <<Российское психологическое общество>> (Москва, 2012~г.), X~Международной конференции <<Интеллектуализация обработки информации>> (Крит, 2014~г.), I~конференции Международной ассоциации когнитивной семиотики (Лунд, 2014~г.), Общемосковском научном семинаре <<Проблемы искусственного интеллекта>>, на семинарах ИСА~РАН и ВЦ~РАН.

\textbf{Публикации.} Основные результаты по теме диссертации изложены в 14 печатных работах~\cite{PanovA2011,PanovA2012a,PanovA2012b,PanovA2012c,PanovA2013b,PanovA2014a,PanovT2010b,PanovT2012a,PanovT2012b,PanovT2013,PanovT2014a,PanovT2014b,PanovA2014c,PanovT2014c}, 4 из которых изданы в рецензируемых журналах из списка ВАК~РФ~\cite{PanovA2012c,PanovA2013b,PanovT2013,PanovA2014a}, 8 "--- в материалах всероссийских и международных конференций~\cite{PanovA2011,PanovA2012a,PanovA2012b,PanovT2010b,PanovT2012b,PanovT2014a,PanovT2014b,PanovT2014c}.

\textbf{Объем и структура работы.} Диссертация состоит из~введения, трёх глав, заключения и~двух приложений. Полный объём диссертации составляет 119 страниц с 21 рисунком. Список литературы содержит \totalcitnums\ наименование.

В \textbf{первой главе} приводится описание предметной области и анализ существующих предпосылок к построению моделей КМ. В~качестве психологических предпосылок рассматриваются культурно-историческое направление в психологии (Л.\,Н.~Выготский и А.\,Р.~Лурия), теория деятельности (А.\,Н.~Леонтьев) и модель психики Е.\,Ю.~Артемьевой. Среди нейрофизиологических моделей наибольшее внимание уделено исследованиям Б.\,Дж.~Баарса, Дж.~Хокинса и Д.~Георга.

Во \textbf{второй главе} рассматривается синтаксический уровень разрабатываемой модели КМ. Приводится формальное определение знака как элемента картины мира и схема процесса формирования нового знака. Приводится классификация типов отношений, возникающих на множестве знаков, и описываются процессы самоорганизации на сети элементов КМ.

В \textbf{третьей главе} рассматривается семантический уровень разрабатываемой модели КМ. Вводится понятие распознающего автомата, являющегося базовым математическим объектом, с помощью которого определяются все компоненты знака. Подробно рассматривается модель процесса восприятия и исследуются множества операторов распознавания, которые строятся при анализе работы образной компоненты знака. Приводится алгоритм процесса формирования и связывания образа и значения знака, проводится анализ сходимости этого процесса.

В \textbf{приложения} включены описания типов картин мира, свойства которых объясняются с помощью разрабатываемой модели (приложение \ref{AppendixA}) и пример описания одной из когнитивных функций (целеполагания) на синтаксическом уровне (приложение \ref{AppendixB}).

В \textbf{заключении} приводятся основные результаты, полученные в работе.
\clearpage