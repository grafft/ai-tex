\newcommand{\sfs}{\fontsize{14pt}{15pt}\selectfont}
\sfs % размер шрифта и расстояния между строками
\thispagestyle{empty}

\vspace{10mm}
\begin{flushright}
  \LargeНа правах рукописи
\end{flushright}

\vspace{30mm}
\begin{center}
{\Large\bf Панов Александр Игоревич}
\end{center}

\vspace{30mm}
\begin{center}
{\bf \LARGE Исследование методов, разработка моделей и алгоритмов формирования элементов знаковой картины мира субъекта деятельности
\par}

\vspace{30mm}
{\Large
Специальность: 05.13.17 "--- Теоретические основы информатики
}

\vspace{15mm}
\LARGEАвтореферат\par
\Largeдиссертации на соискание учёной степени\par
кандидата физико-математических наук
\end{center}

\vspace{40mm}
\begin{center}
{\Large Москва "--- 2015}
\end{center}

\newpage
% оборотная сторона обложки
\thispagestyle{empty}
\noindent Работа выполнена в лаборатории <<Динамические интеллектуальные системы>> Федерального государственного бюджетного учреждения науки Институт системного анализа Российской академии наук.

\begin{table} [h]  
  \begin{tabular}{ll}  
   \makecell[l]{\sfs Научный руководитель:\\~} &
   \makecell*[{{p{11cm}}}]{\sfs
   доктор физико-математических наук, профессор \\ \textbf{\sfs Осипов Геннадий Семёнович}}
      
\vspace{3mm} \\

   \makecell[l]{\sfs Официальные оппоненты: \vspace{6.65cm}} &
   \makecell[{{p{11cm}}}]{   
   \sfs \textbf{Фамилия Имя Отчество,} \\
   \sfs доктор физико-математических наук, профессор, \\
   \sfs Основное место работы c длинным длинным длинным длинным длинным длинным длинным длинным названием, \\ 
   \sfs старший научный сотрудник \vspace{1mm} \\
   \sfs \textbf{Фамилия Имя Отчество,} \\
   \sfs доктор физико-математических наук, \\
   \sfs Основное место работы c длинным длинным длинным длинным названием, \\    
   \sfs старший научный сотрудник
   }

\vspace{3mm} \\

   \makecell[l]{\sfs Ведущая организация:\\~\\~\\~} &
   \makecell*[{{p{11cm}}}]{\sfs
   Федеральное государственное бюджетное образовательное учреждение высшего профессионального образования с длинным длинным длинным длинным названием
   }
  \end{tabular}  
\end{table}

\noindent Защита состоится \underline{<<5>> мая 2015~г.}~в~\underline{15} часов \underline{00} минут на заседании 
диссертационного совета Д~002.017.02 в Федеральном государственном 
бюджетном учреждении науки Вычислительный центр им. А.\,А.~Дородницына Российской академии наук по адресу: 119333, Москва, ул.~Вавилова, д.~40 (конференц-зал). 

\vspace{5mm}
\noindent С диссертацией можно ознакомиться в библиотеке ВЦ~РАН и на 
официальном сайте ВЦ~РАН: http://www.ccas.ru/.

\vspace{5mm}
\noindentАвтореферат разослан \underline{<<15>> марта 2015~г.}

\vspace{5mm}
\begin{table} [h]
  \begin{tabular}{p{13cm}cr}
    \begin{tabular}{p{10cm}}
      \sfs Ученый секретарь  \\
      \sfs диссертационного совета Д~002.017.02\\
      \sfs доктор физико-математических наук
    \end{tabular} 
    & \begin{tabular}{c}
       %\includegraphics [height=2cm] {signature} 
    \end{tabular} 
    & \begin{tabular}{r}
       \\
       \\
       \sfs Рязанов~В.\,В.
    \end{tabular} 
  \end{tabular}
\end{table}
\newpage