%%% Макет страницы %%%
\geometry{a4paper,top=2cm,bottom=2cm,left=2.5cm,right=1cm}
\fancyhf{}
\renewcommand{\headrulewidth}{0pt}
\fancyhead[C]{\fontsize{14}{\baselineskip} \selectfont  \thepage}
\pagestyle{fancy}
% добавляем номер станицы сверху
\fancypagestyle{plain}{%
	\fancyhf{} 							% clear all header and footer fields
	\fancyhead[C]{\fontsize{14}{\baselineskip} \selectfont \thepage} % except the center
}
\parindent=1.2cm		% абзацный отступ
\renewcommand{\rmdefault}{ftm} % Включаем Times New Roman
\renewcommand{\baselinestretch}{1.5}



%%% Выравнивание и переносы %%%
\sloppy					% Избавляемся от переполнений
\clubpenalty=10000		% Запрещаем разрыв страницы после первой строки абзаца
\widowpenalty=10000		% Запрещаем разрыв страницы после последней строки абзаца
\hyphenpenalty=10000	% Запрещаем переносы во всем тексте


%%% Изображения %%%
\graphicspath{{../images/}} % Пути к изображениям



%%% Цвета гиперссылок %%%
\definecolor{linkcolor}{rgb}{0.9,0,0}
\definecolor{citecolor}{rgb}{0,0.6,0}
\definecolor{urlcolor}{rgb}{0,0,1}
\hypersetup{
    colorlinks, linkcolor={linkcolor},
    citecolor={citecolor}, urlcolor={urlcolor}
}



%%% Оглавление %%%
\renewcommand{\cftchapdotsep}{\cftdotsep}
\renewcommand\cftchapaftersnum{.} % добавляем точки после номеров в оглавлении
\renewcommand\cftsecaftersnum{.}
\renewcommand\cftsubsecaftersnum{.}



%%% Теормоподобные окружения %%%
\theoremstyle{plain}
\newtheorem{Theorem}{Теорема}
\newtheorem{Lemma}[Theorem]{Лемма}
\newtheorem{Pred}{Утверждение}
\newtheorem{Corollary}{Следствие}
\newtheorem{Def}{Определение}
\newtheorem{Axiom}{Аксиома}
\newtheorem{Hypothesis}{Гипотеза}
\newtheorem{Assumption}{Предположение}
\newtheorem{Problem}{Задача}
\newtheorem{Example}{Пример}
\newtheorem{Fact}{Факт}
\newtheorem{Remark}{Замечание}
\newtheorem{Rule}{Правило}
\newtheorem{Condition}{Условие}
\newenvironment{Proof}%
    {\par\noindent{\bf Доказательство.}}%
    {\hfill$\scriptstyle\blacksquare$}



%%% Настройка заголовков %%%
% чтобы поставить точечку после номера алгоритма и картинки в \caption:
\captionsetup[ruled]{labelsep=period}
\captionsetup[figure]{labelsep=period}
% Общие настройки
\titleformat{\chapter}[display]
{\normalfont\huge\bfseries\centering}
{\chaptertitlename\ \thechapter}{0pt}{\Huge}
\titleformat{\section}[block]
	{\normalfont\LARGE\bfseries\centering}
	{\thesection.}{10pt}{\LARGE}
\titleformat{\subsection}[block]
	{\normalfont\large\bfseries\centering}
	{\thesubsection.}{10pt}{\large}
\titlespacing*{\chapter}{0pt}{-10pt}{1\baselineskip}
\titlespacing*{\section}{0pt}{1\baselineskip}{1\baselineskip}
\titlespacing*{\subsection}{0pt}{1\baselineskip}{1\baselineskip}
% центрируем заголовок оглавления
\renewcommand{\cfttoctitlefont}{\hfill\Huge\bfseries}
\renewcommand{\cftaftertoctitle}{\hfill}
% центрируем заголовок списка картинок
\renewcommand{\cftloftitlefont}{\hfill\Huge\bfseries}
\renewcommand{\cftafterloftitle}{\hfill}
%use russian alphabet for subfigure counter 
\makeatletter
\def\thesubfigure{\textit{\asbuk{subfigure}}}
\providecommand\thefigsubsep{,~}
\def\p@subfigure{\@nameuse{thefigure}\thefigsubsep}
\makeatother


%%% Костыль для взаимодействия hyperref и appendix + русские буквы в индексации http://tex.stackexchange.com/questions/41649/how-to-make-appendix-and-hyperref-packages-work-together-with-cyrillic-non-asci %%%
\makeatletter
\let\oriAlph\Alph
\let\orialph\alph
\renewcommand{\@resets@pp}{\par
	\@ppsavesec
	\stepcounter{@pps}
	\setcounter{section}{0}%
	\if@chapter@pp
	\setcounter{chapter}{0}%
	\renewcommand\@chapapp{\appendixname}%
	\renewcommand\thechapter{\@Asbuk\c@chapter}%
	\else
	\setcounter{subsection}{0}%
	\renewcommand\thesection{\@Alph\c@section}%
	\fi
	\if@pphyper
	\if@chapter@pp
	\renewcommand{\theHchapter}{\theH@pps.\oriAlph{chapter}}%
	\else
	\renewcommand{\theHsection}{\theH@pps.\oriAlph{section}}%
	\fi
	\def\Hy@chapapp{appendix}%
	\fi
	\restoreapp
}
\makeatother



%%% Прочее %%%
% Нумерованный перечень со скобками
\renewcommand\labelenumi{\theenumi )}
\DeclareMathOperator*{\argmax}{arg\,max}
\algnewcommand\And{\textbf{and}}
% Счетчики
\newcounter{citnum}
\def\oldbibitem{} \let\oldbibitem=\bibitem
\def\bibitem{\stepcounter{citnum}\oldbibitem}