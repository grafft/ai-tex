\documentclass[%
candidate,      % тип документа
natbib,         % использовать пакет natbib для "сжатия" цитирований
subf,           % использовать пакет subcaption для вложенной нумерации рисунков
href,           % использовать пакет hyperref для создания гиперссылок
colorlinks=true % цветные гиперссылки
%,times         % шрифт Times как основной
%,fixint=false  % отключить прямые знаки интегралов
%,classified    % гриф секретности
%,libcat        % номер УДК
,facsimile     % отображать факсимиле диссертанта
]{disser}

\usepackage[
  a4paper, mag=1000,
  left=2.5cm, right=1cm, top=2cm, bottom=2cm, headsep=0.7cm, footskip=1cm
]{geometry}
\usepackage[T2A]{fontenc}
\usepackage[utf8]{inputenc}
\usepackage[english,russian]{babel}
%\usepackage{tabularx,longtable}
\ifpdf\usepackage{epstopdf}\fi

\usepackage[ruled]{algorithm}					% Пакеты листингов и алгоритмов
\usepackage[noend]{algpseudocode}
\usepackage{amsthm,amsopn}

% Номера страниц снизу и по центру
%\pagestyle{footcenter}
%\chapterpagestyle{footcenter}

% Точка с запятой в качестве разделителя между номерами цитирований
%\setcitestyle{semicolon}

% Ссылки на работы соискателя включаются в общий список литературы
\let\citemy=\cite

% Использовать полужирное начертание для векторов
\let\vec=\mathbf

% Путь к файлам с иллюстрациями
\graphicspath{{../../images/}}

\DeclareMathOperator*{\argmax}{arg\,max}

\sloppy
\begin{document}

% Переопределение стандартных заголовков
%\def\contentsname{Содержание}
%\def\conclusionname{Выводы}
%\def\bibname{Литература}
\theoremstyle{plain}
\newtheorem{Theorem}{Теорема}
\newtheorem{Def}{Определение}
\newtheorem{Pred}{Утверждение}
\newtheorem{Corollary}{Следствие}
\newenvironment{Proof}%
{
	\par\noindent{\bf Доказательство.}
}%
{
	\hfill$\scriptstyle\blacksquare$
}

\floatname{algorithm}{Алгоритм}
\algrenewcommand\algorithmicrequire{\textbf{Вход:}}
\algrenewcommand\algorithmicensure{\textbf{Выход:}}
\algrenewcommand\algorithmicforall{\textbf{для всех}}
\algrenewcommand\algorithmicwhile{\textbf{пока}}
\algrenewcommand\algorithmicif{\textbf{если}}
\algrenewcommand\algorithmicthen{\textbf{то}}
\algrenewcommand\algorithmicelse{\textbf{иначе}}
\algrenewcommand\algorithmicreturn{\textbf{вернуть}}
\algrenewcommand\algorithmicdo{}
\renewcommand{\algorithmiccomment}[1]{{\quad\sl // #1}}

% Включение файла с общим текстом диссертации и автореферата
% (текст титульного листа и характеристика работы).
% Общие поля титульного листа диссертации и автореферата
\institution{Название организации}

\topic{Методы эффективного решения комбинаторных задач на основе знакового представления знаний}

\author{Панов Александр Игоревич}

\specnum{05.13.17}
\spec{Теоретические основы информатики}
%\specsndnum{01.04.07}
%\specsnd{Физика конденсированного состояния}

\scon{Осипов Геннадий Семенович}
\sconstatus{д.~ф.-м.~н., проф.}
%\sconsnd{ФИО второго консультанта}
%\sconsndstatus{д.~ф.-м.~н., проф.}

\city{Москва}
\date{\number\year}

% Общие разделы автореферата и диссертации
\mkcommonsect{actuality}{Актуальность темы исследования.}{%
Текст об актуальности. Ссылка~\cite{}.
}

\mkcommonsect{development}{Степень разработанности темы исследования.}{
Текст о степени разработанности темы.
}

\mkcommonsect{objective}{Цели и задачи диссертационной работы:}{%
Список целей.

Для достижения поставленных целей были решены следующие задачи:
}

\mkcommonsect{novelty}{Научная новизна.}{%
Текст о новизне.
}

\mkcommonsect{value}{Теоретическая и практическая значимость.}{%
Результаты, изложенные в диссертации, могут быть использованы для ...
}

\mkcommonsect{methods}{Методология и методы исследования.}{%
Текст о методах исследования.
}

\mkcommonsect{results}{Положения, выносимые на защиту:}{%
Текст о положениях и результатах.
}

\mkcommonsect{approbation}{Степень достоверности и апробация результатов.}{%
Основные результаты диссертации докладывались на следующих конференциях:
}

\mkcommonsect{pub}{Публикации.}{%
Материалы диссертации опубликованы в $N$ печатных работах, из них $n_1$
статей в рецензируемых журналах~\citemy{Ivanov_1999_Journal_17_173,
Petrov_2001_Journal_23_12321,Sidorov_2002_Journal_32_1531}, $n_2$ статей в
сборниках трудов конференций и $n_3$ тезисов докладов.
}

\mkcommonsect{contrib}{Личный вклад автора.}{%
Содержание диссертации и основные положения, выносимые на защиту, отражают персональный вклад автора в опубликованные работы.
Подготовка к публикации полученных результатов проводилась совместно с соавторами, причем вклад диссертанта был определяющим. Все представленные в диссертации результаты получены лично автором.
}

\mkcommonsect{struct}{Структура и объем диссертации.}{%
Диссертация состоит из введения, обзора литературы, $n$ глав, заключения и библиографии.
Общий объем диссертации $P$ страниц, из них $p_1$ страницы текста, включая $f$ рисунков.
Библиография включает $B$ наименований на $p_2$ страницах.
}

% номер копии для грифа секретности
%\copynum{1}
% класс доступа
%\classlabel{Для служебного пользования}

% номер УДК
\libcatnum{004.81}

\title{ДИССЕРТАЦИЯ\\
на соискание учёной степени\\
кандидата физико-математических наук}

\maketitle

% Содержание
\tableofcontents

% Введение
\chapter*{Введение}							% Заголовок
\addcontentsline{toc}{chapter}{Введение}	% Добавляем его в оглавление
\textbf{Актуальность темы исследования.} 

Исследования картин мира (КМ) субъектов деятельности принадлежат одному из центральных направлений в когнитивной психологии. Высшие психические функции, в том числе связанные с приобретением и использованием знаний, являются, в широком смысле, продуктом работы КМ субъекта. Исследованию большого числа процессов, протекающих в КМ, в~том числе высших когнитивных, таких как категоризация и обобщение, целеполагание, планирование, принятие решения, творческие синтез и анализ, было посвящено значительное число работ на протяжении всей истории психологической науки. Следует отметить работы по восприятию Дж.\,А.~Фодора (J.\,A.~Fodor), Б.~Юлеза (B.~Julesz), Дж.\,Е.~Каттинга (J.\,E.~Cutting), С.~Гроссберга (S.~Grossberg), А.\,Р.~Лурия, Б.\,М.~Величковского, В.\,П.~Зинченко и памяти С.~Стернберга (S.~Sternberg), Л.~Джакоби (L.~Jacoby), Р.~Аткинсона (R.~Atkinson), Р.~Шиффрина (R.~Shiffrin), Е.~Тулвинга (E.~Tulving).

В последнее время исследованию когнитивных функций человека уделяется большое внимание не только в самой психологии, но и в нейрофизиологии и в~искусственном интеллекте. Нейрофизиологи основной своей задачей ставят поиск нейронного субстрата психических функций. При этом в качестве основного инструмента здесь выступает картирование участков коры головного мозга и отслеживание динамики активности различных участков при выполнении той или иной когнитивной задачи. Большое количество накопленного фактического материала используется для подтверждения целого ряда разрозненных моделей отдельных психических функций. Примерами могут служить работы по моделям внимания Я.\,Б.~Казановича, С.~Фринтропа (S.~Frintrop), С.~Коха (C.~Koch), Л.~Итти (L.~Itti), Дж.\,К.~Сосоза (J.\,K.~Tsotsos), А.~Торралба (A.~Torralba), Л.~Жэнга (L.~Zhang), Р.\,А.~Ренсинка (R.\,A.~Rensink). Единого аппарата для построения таких моделей на данный момент не существует, хотя имеется ряд работ Б.\,Дж.~Баарса (B.\,J.~Baars), Р.~Сана (R.~Sun), Дж.~Хокинса (J.~Hawkins), которые можно считать первыми попытками их создания.

Искусственный интеллект в начале своего становления как науки использовал для построения интеллектуальных алгоритмов данные психологов. Однако спустя некоторое время психологические соображения уже перестали рассматриваться как определяющие при разработке того или иного алгоритма. Центральное место стали занимать вопросы вычислительной эффективности и специализации в той или иной предметной области. В~связи с~тем, что в~большинстве интеллектуальных систем в~настоящее время требуется всё большая степень универсальности и автономности, начинается процесс возвращения к психологическим основам строения психики человека. Возникает задача построения моделей процессов, например, распознавания и планирования, на некоторой <<биологически инспирированной основе>>. К этому направлению относятся работы Дж.\,Р.~Андерсона (J.\,R.~Anderson), П.~Леирда (J.\,E.~Laird), П.~Ленгли (P.~Langley). Подтверждением повышенного интереса к этой теме служат организуемые в последнее время конференции и издаваемые журналы, посвящённые исключительно <<биологически правдоподобным>> архитектурам (например, ежегодные конференции BICA (Annual International Conference on Biologically Inspired Cognitive Architectures)\cite{BICAC2014} и журнал BICA \cite{BICAJ2014}).

Потребность в единой модели КМ субъекта деятельности для нейрофизиологов и исследователей в области искусственного интеллекта определяет актуальность данной работы. Такая модель требуется как для построения моделей когнитивных функций человека на нейронном уровне, подтверждаемых нейрофизиологическими данными о строении высшей нервной системы человека и данными об активности соответствующего определённой функции участка коры головного мозга, так и для построения абстрагированных от того или иного субстрата интеллектуальных алгоритмов, которые могли бы быть использованы в автономных системах свободной конфигурации.

Один из основных вопросов, возникающих при разработке модели КМ, заключается в описании базовых элементов картины мира и построении алгоритма их формирования в процессе деятельности субъекта, носителя КМ. В качестве психологической основы для построения модели элемента КМ были использованы, с одной стороны, культурно"--~исторический подход Л.\,Н.~Выготского и теория деятельности А.\,Н.~Леонтьева, с другой стороны "--- идеи прикладной семиотики, предложенные в работах Д.\,А.~Поспелова, А.~Мейстеля, Г.\,С.~Осипова. В качестве нейрофизиологических предпосылок были использованы концепции и нейронные схемы Д.~Георга (D.~George).

\textbf{Предмет исследования} "--- построение знаковых моделей картины мира и некоторых когнитивных функций субъекта деятельности.

\textbf{Целью исследования} является разработка моделей и алгоритмов формирования элементов знаковой картины мира, обладающих структурой, необходимой для построения моделей высших когнитивных функций, в том числе восприятия, внимания, планирования и целеполагания.

Для~достижения цели работы были поставлены следующие \textbf{задачи}:
\begin{enumerate}
  \item исследовать модель элемента картины мира субъекта, построенную на основе психологической теории деятельности,
  \item построить модель структурных компонент элемента картины мира, опирающуюся на нейрофизиологические данные, и исследовать её,
  \item исследовать структуру отношений и процессы самоорганизации на множестве элементов картины мира на синтаксическом уровне,
  \item исследовать процесс формирования и связывания основных компонент нового элемента картины мира и построить соответствующий алгоритм,
  \item исследовать сходимость процесса формирования и связывания основных компонент нового элемента картины мира.
\end{enumerate}

\textbf{Научная новизна и результаты, выносимые на~защиту.}
\begin{enumerate}
	\renewcommand\labelenumi{\theenumi.}
  \item Впервые построена модель структурных компонент элемента картины мира субъекта деятельности.
  \item Построены операторы распознавания в статическом, динамическом и иерархическом случаях в терминах алгебраической теории для образной компоненты элемента картины мира.
  \item Доказаны теоремы корректности линейных замыканий множеств построенных в работе операторов распознавания.
  \item Построен алгоритм формирования и связывания основных компонент нового элемента картины мира.
  \item Проведено исследование процесса формирования и связывания основных компонент нового элемента картины мира.
\end{enumerate}

\textbf{Практическая значимость.} Построение модели элементов картины мира субъекта деятельности, с~одной стороны, позволит создать универсальные интеллектуальные алгоритмы планирования поведения, целеполагания, локализации, распознавания и категоризации, применение которых в интеллектуальных системах повысит степень их автономности, а с~другой стороны, позволит объяснить некоторые патологические явления в мозге человека и дать рекомендации к их устранению.

\textbf{Методы исследования.} Теоретические результаты работы получены и обоснованы с использованием методов теории множеств, алгебраической теории распознавания образов, теории интеллектуальных динамических систем, теории деятельности.

\textbf{Достоверность результатов} подтверждена строгими математическими доказательствами утверждений и результатами вычислительных экспериментов.

\textbf{Апробация результатов исследования.}

Основные результаты работы докладывались~на: Международных конференциях по когнитивной науке (Томск, 2010~г.; Калининград, 2012~г., 2014~г.), II~Всероссийской научной конференции молодых учёных с международным участием <<Теория и практика системного анализа>> (Рыбинск, 2012~г.), IV~Международной конференции <<Системный анализ и информационные технологии>> (Абзаково, 2011~г.), V~съезде Общероссийской общественной организации <<Российское психологическое общество>> (Москва, 2012~г.), X~Международной конференции <<Интеллектуализация обработки информации>> (Крит, 2014~г.), I~конференции Международной ассоциации когнитивной семиотики (Лунд, 2014~г.), Общемосковском научном семинаре <<Проблемы искусственного интеллекта>>, на семинарах ИСА~РАН и ВЦ~РАН.

\textbf{Публикации.} Основные результаты по теме диссертации изложены в 14 печатных работах~\cite{PanovA2011,PanovA2012a,PanovA2012b,PanovA2012c,PanovA2013b,PanovA2014a,PanovT2010b,PanovT2012a,PanovT2012b,PanovT2013,PanovT2014a,PanovT2014b,PanovA2014c,PanovT2014c}, 4 из которых изданы в рецензируемых журналах из списка ВАК~РФ~\cite{PanovA2012c,PanovA2013b,PanovT2013,PanovA2014a}, 8 "--- в материалах всероссийских и международных конференций~\cite{PanovA2011,PanovA2012a,PanovA2012b,PanovT2010b,PanovT2012b,PanovT2014a,PanovT2014b,PanovT2014c}.

\textbf{Объем и структура работы.} Диссертация состоит из~введения, трёх глав, заключения и~двух приложений. Полный объём диссертации составляет 119 страниц с 21 рисунком. Список литературы содержит \totalcitnums\ наименование.

В \textbf{первой главе} приводится описание предметной области и анализ существующих предпосылок к построению моделей КМ. В~качестве психологических предпосылок рассматриваются культурно-историческое направление в психологии (Л.\,Н.~Выготский и А.\,Р.~Лурия), теория деятельности (А.\,Н.~Леонтьев) и модель психики Е.\,Ю.~Артемьевой. Среди нейрофизиологических моделей наибольшее внимание уделено исследованиям Б.\,Дж.~Баарса, Дж.~Хокинса и Д.~Георга.

Во \textbf{второй главе} рассматривается синтаксический уровень разрабатываемой модели КМ. Приводится формальное определение знака как элемента картины мира и схема процесса формирования нового знака. Приводится классификация типов отношений, возникающих на множестве знаков, и описываются процессы самоорганизации на сети элементов КМ.

В \textbf{третьей главе} рассматривается семантический уровень разрабатываемой модели КМ. Вводится понятие распознающего автомата, являющегося базовым математическим объектом, с помощью которого определяются все компоненты знака. Подробно рассматривается модель процесса восприятия и исследуются множества операторов распознавания, которые строятся при анализе работы образной компоненты знака. Приводится алгоритм процесса формирования и связывания образа и значения знака, проводится анализ сходимости этого процесса.

В \textbf{приложения} включены описания типов картин мира, свойства которых объясняются с помощью разрабатываемой модели (приложение \ref{AppendixA}) и пример описания одной из когнитивных функций (целеполагания) на синтаксическом уровне (приложение \ref{AppendixB}).

В \textbf{заключении} приводятся основные результаты, полученные в работе.
\clearpage

% Основная часть
%% Глава 1
\input{chapter1}
%% Глава 2
\input{chapter2}
%% Параграф 3.1
\input{section3.1}
%% Параграф 3.2
\input{section3.2}

% Заключение
\chapter*{Заключение}						% Заголовок
\addcontentsline{toc}{chapter}{Заключение}	% Добавляем его в оглавление

В~работе были получены следующие основные результаты.

\begin{enumerate}
	\renewcommand\labelenumi{\theenumi.}
	\item Впервые была построена модель структурных компонент элемента картины мира субъекта деятельности.
	\item Впервые была проведена постановка задач распознавания в терминах алгебраической теории для образной компоненты элемента картины мира в динамическом и иерархическом случаях.
	\item Были доказаны теоремы корректности некоторых множеств построенных в работе операторов распознавания.
	\item Был построен алгоритм итерационного процесса формирования нового элемента картины мира.
	\item Было проведено оригинальное исследование итерационного процесса формирования нового элемента картины мира.
\end{enumerate}

\clearpage

% Список литературы
\bibliography{../../biblio/umain}
\bibliographystyle{ugost2008}

% Список иллюстративного материала
\listoffigures

% Приложения
\appendix
\chapter{Типы картин мира} \label{AppendixA}

В соответствие с предыдущим параграфом в результате работы механизмов самоорганизации на множестве знаков формируются три основных типа структур. Каждую из них в соответствии с \cite{Osipov1990} будем называть неоднородной семантической сетью или, поскольку это не приводит к недоразумениям, семантической сетью. Рассмотрим три таких сети.
\begin{enumerate}
	\renewcommand\labelenumi{\theenumi.}
	\item Семантическую сеть $H_P=\langle 2^P,\mathfrak R_P\rangle$ на множестве образов, где $\mathfrak R_P=\{R_1,R_2,R_3,R_4\}$ "--- семейство отношений на образах.
	\item Семантическую сеть $H_A=\langle 2^A,\mathfrak R_A\rangle$ на множестве личностных смыслов, где $\mathfrak A=\{R_5\}$ "--- семейство отношений на личностных смыслах.
	\item Семантическую сеть $H_M=\langle 2^M,\mathfrak R_M\rangle$ на множестве значений знаков, где $\mathfrak R_M=\{R_1^\prime,R_3^\prime,R_6,R_7\}$ "--- семейство отношений на значениях.
\end{enumerate}

Тройку объектов $H=\langle H_P,H_A,H_M\rangle$ будем называть семиотической сетью. Переходы между сетями $H_P$, $H_A$, $H_M$ реализуются, как следует из предыдущего, посредством процедур $\Psi_m^a$, $\Psi_a^p$ и $\Psi_p^m$.

Уровень имён знаков может наследовать каждую из описанных выше семантических сетей. Благодаря такому наследованию можно говорить о формировании той или иной семантической сети на уровне знаков (не только на уровне их компонент).

Предлагается выделять три типа картины мира: рациональную, житейскую и мифологическую \cite{Chudova2012a}. 
Мы видели, что на сети $H_P$ можно определить операции обобщения (и классификации) по признакам (раздел \ref{subsect_2_3_1}). Именно эти операции характерны для рациональной картины мира. На основании этих соображений и ряда психологических экспериментов (описание которых остаётся за пределами настоящего доклада) можно полагать, что именно сеть на множестве образов (и ее наследование на уровень имён знаков) лежит в основе рациональной картины мира. Здесь надо подчеркнуть важность слов <<в основе>>. Все типы картин мира используют сети на образах, на смыслах и сценариях, но есть некоторая <<управляющая>> сеть, которая служит для формулирования цели, поиска подходящих действий, вызова сценариев и изменения личностных смыслов. Например, в рациональной картине мира в сети на образах выполняется выработка цели, затем в сети на значениях находятся подходящие роли в сценарии как условия выполнения действий для достижения цели, далее учитываются смыслы объектов, которые могут быть мотивами или препятствиями, или средствами для достижения цели \cite{Chudova2014}. Отметим, что могут быть описаны и вырожденные картины мира, в которых используются не три, а только две сети (например, только $H_A$ и $H_M$ для нигилистической картины мира \cite{Chudova2012a}). 

Житейская картина мира характеризуется следованием некоторым стереотипам или сценариям поведения. Таким образом, наследование на уровень имён знаков сети на значениях приводит к формированию житейской картины мира. Здесь также следует отметить, что сеть на значениях является лишь ведущей: моделирование, например, картины мира чиновников реализуется на двух сетях "--- сценариев и личностных смыслов. Поэтому при возникновении нового предмета потребности (например, выделение бюджета на науку и культуру) находится сценарий, в котором смысл цели из амбивалентного превращается в смысл препятствия. Поскольку в этом процессе не присутствуют образы, то речь идёт о вырожденной картине мира. В общем случае в житейской картине мира выбранный сценарий (на сети значений) пополняется образами тех объектов (в том числе партнёров), которые наилучшим образом (в соответствии с оценкой на сети смыслов) могут исполнять записанные в сценарии роли (например, начальник подбирает исполнителей в новую группу для <<хорошего>> выполнения нового вида работ или жених и невеста составляют список гостей на свадьбу в соответствии со своими представлениями о том, как должна выглядеть <<хорошая>> свадьба).

В мифологической картине мира каждая роль имеет неизменный смысл и заданный образ, т.е. ведущей в этом случае является сеть на смыслах. Иначе говоря, наследование сети $H_A$ на уровень имён знаков приводит к формированию мифологической картины мира.

\input{app-b}

\end{document}