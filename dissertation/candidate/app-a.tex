\chapter{Типы картин мира} \label{AppendixA}

В соответствие с предыдущим параграфом в результате работы механизмов самоорганизации на множестве знаков формируются три основных типа структур. Каждую из них в соответствии с \cite{Osipov1990} будем называть неоднородной семантической сетью или, поскольку это не приводит к недоразумениям, семантической сетью. Рассмотрим три таких сети.
\begin{enumerate}
	\renewcommand\labelenumi{\theenumi.}
	\item Семантическую сеть $H_P=\langle 2^P,\mathfrak R_P\rangle$ на множестве образов, где $\mathfrak R_P=\{R_1,R_2,R_3,R_4\}$ "--- семейство отношений на образах.
	\item Семантическую сеть $H_A=\langle 2^A,\mathfrak R_A\rangle$ на множестве личностных смыслов, где $\mathfrak A=\{R_5\}$ "--- семейство отношений на личностных смыслах.
	\item Семантическую сеть $H_M=\langle 2^M,\mathfrak R_M\rangle$ на множестве значений знаков, где $\mathfrak R_M=\{R_1^\prime,R_3^\prime,R_6,R_7\}$ "--- семейство отношений на значениях.
\end{enumerate}

Тройку объектов $H=\langle H_P,H_A,H_M\rangle$ будем называть семиотической сетью. Переходы между сетями $H_P$, $H_A$, $H_M$ реализуются, как следует из предыдущего, посредством процедур $\Psi_m^a$, $\Psi_a^p$ и $\Psi_p^m$.

Уровень имён знаков может наследовать каждую из описанных выше семантических сетей. Благодаря такому наследованию можно говорить о формировании той или иной семантической сети на уровне знаков (не только на уровне их компонент).

Предлагается выделять три типа картины мира: рациональную, житейскую и мифологическую \cite{Chudova2012a}. 
Мы видели, что на сети $H_P$ можно определить операции обобщения (и классификации) по признакам (раздел \ref{subsect_2_3_1}). Именно эти операции характерны для рациональной картины мира. На основании этих соображений и ряда психологических экспериментов (описание которых остаётся за пределами настоящего доклада) можно полагать, что именно сеть на множестве образов (и ее наследование на уровень имён знаков) лежит в основе рациональной картины мира. Здесь надо подчеркнуть важность слов <<в основе>>. Все типы картин мира используют сети на образах, на смыслах и сценариях, но есть некоторая <<управляющая>> сеть, которая служит для формулирования цели, поиска подходящих действий, вызова сценариев и изменения личностных смыслов. Например, в рациональной картине мира в сети на образах выполняется выработка цели, затем в сети на значениях находятся подходящие роли в сценарии как условия выполнения действий для достижения цели, далее учитываются смыслы объектов, которые могут быть мотивами или препятствиями, или средствами для достижения цели \cite{Chudova2014}. Отметим, что могут быть описаны и вырожденные картины мира, в которых используются не три, а только две сети (например, только $H_A$ и $H_M$ для нигилистической картины мира \cite{Chudova2012a}). 

Житейская картина мира характеризуется следованием некоторым стереотипам или сценариям поведения. Таким образом, наследование на уровень имён знаков сети на значениях приводит к формированию житейской картины мира. Здесь также следует отметить, что сеть на значениях является лишь ведущей: моделирование, например, картины мира чиновников реализуется на двух сетях "--- сценариев и личностных смыслов. Поэтому при возникновении нового предмета потребности (например, выделение бюджета на науку и культуру) находится сценарий, в котором смысл цели из амбивалентного превращается в смысл препятствия. Поскольку в этом процессе не присутствуют образы, то речь идёт о вырожденной картине мира. В общем случае в житейской картине мира выбранный сценарий (на сети значений) пополняется образами тех объектов (в том числе партнёров), которые наилучшим образом (в соответствии с оценкой на сети смыслов) могут исполнять записанные в сценарии роли (например, начальник подбирает исполнителей в новую группу для <<хорошего>> выполнения нового вида работ или жених и невеста составляют список гостей на свадьбу в соответствии со своими представлениями о том, как должна выглядеть <<хорошая>> свадьба).

В мифологической картине мира каждая роль имеет неизменный смысл и заданный образ, т.е. ведущей в этом случае является сеть на смыслах. Иначе говоря, наследование сети $H_A$ на уровень имён знаков приводит к формированию мифологической картины мира.
