\documentclass[landscape,a0paper,fontscale=0.285]{baposter} % Adjust the font scale/size here

\usepackage{cmap}								% Поддержка поиска русских слов в PDF (pdflatex)
%\usepackage[T2A]{fontenc}       				% Поддержка кириллицы
%\usepackage[utf8]{inputenc}					% Выбор языка и кодировки
\usepackage[english, russian]{babel}
\usepackage{fontspec}
\setmainfont{Qlassik Medium}
\setsansfont{Qlassik Medium}

\usepackage{graphicx} 							% Required for including images
\graphicspath{{../../images/}} 					% Directory in which figures are stored

\usepackage{enumitem} 							% Used to reduce itemize/enumerate spacing
\usepackage[font=small,labelfont=bf]{caption} % Required for specifying captions to tables and figures

\usepackage{multicol} 							% Required for multiple columns
\setlength{\columnsep}{1.5em} 					% Slightly increase the space between columns
\setlength{\columnseprule}{0mm} 				% No horizontal rule between columns

\usepackage{tikz} 								% Required for flow chart
\usetikzlibrary{shapes,arrows} 					% Tikz libraries required for the flow chart in the template

\usepackage[
%	autolang=hyphen,
	language=auto,
	autolang=other,
	backend=biber,
	style=gost-numeric
]{biblatex}
\addbibresource{strl.bib}

\DeclareSourcemap{
	\maps[datatype=bibtex, overwrite]{
		\map{
			\step[fieldset=langid, fieldvalue=english]
			\step[fieldset=doi, null]
			\step[fieldset=issn, null]
			\step[fieldset=isbn, null]
			\step[fieldset=url, null]
			\step[fieldsource=language, fieldset=langid, origfieldval]
		}
	}
}

\newcommand{\compresslist}{ 					% Define a command to reduce spacing within itemize/enumerate environments, this is used right after \begin{itemize} or \begin{enumerate}
	\setlength{\itemsep}{1pt}
	\setlength{\parskip}{0pt}
	\setlength{\parsep}{0pt}
}

\definecolor{lightblue}{rgb}{0.145,0.6666,1} 	% Defines the color used for content box headers

\begin{document}

\begin{poster}
{
	headerborder=closed, % Adds a border around the header of content boxes
	colspacing=1em, % Column spacing
	bgColorOne=white, % Background color for the gradient on the left side of the poster
	bgColorTwo=white, % Background color for the gradient on the right side of the poster
	borderColor=lightblue, % Border color
	headerColorOne=black, % Background color for the header in the content boxes (left side)
	headerColorTwo=lightblue, % Background color for the header in the content boxes (right side)
	headerFontColor=white, % Text color for the header text in the content boxes
	boxColorOne=white, % Background color of the content boxes
	textborder=roundedsmall, % Format of the border around content boxes, can be: none, bars, coils, triangles, rectangle, rounded, roundedsmall, roundedright or faded
	eyecatcher=true, % Set to false for ignoring the left logo in the title and move the title left
	headerheight=0.15\textheight, % Height of the header
	headershape=roundedright, % Specify the rounded corner in the content box headers, can be: rectangle, small-rounded, roundedright, roundedleft or rounded
	headerfont=\Large\bf\textsc, % Large, bold and sans serif font in the headers of content boxes
	%textfont={\setlength{\parindent}{1.5em}}, % Uncomment for paragraph indentation
	linewidth=2pt % Width of the border lines around content boxes
}
%----------------------------------------------------------------------------------------
%	TITLE SECTION 
%----------------------------------------------------------------------------------------
%
{\includegraphics[height=4em]{isa-frc/isa-logo.png}} 
{\bf\textsc{Требуется исследователь и разработчик в области многоагентных систем, искусственного интеллекта и интеллектуального управления}\vspace{0.5em}} 
{\textsc{Федеральный исследовательский центр <<Информатика и управление>> РАН}} 
{\includegraphics[height=4em]{isa-frc/isa-logo.png}} 

\headerbox{О проекте}{name=about,column=0,span=2,row=0}{
	
	Целью  проекта является повышение степени автономности технических объектов, реализующих сложное поведение в динамической среде, направленное на достижение совместной цели. Рассматриваемая модельная задача - автоматизация управления (на стратегическом, тактическом и реактивном уровнях) группой беспилотных летательных аппаратов мультироторного типа (квадрокоптеров). Для решения рассматриваемой задачи и достижения цели проекта используются методы и подходы искусственного интеллекта, когнитивных наук, теории управления и др.
	
	В задачи потенциального кандидата в рамках проекта будут входить:
	\begin{enumerate}\compresslist
		\item разработка многоагентной системы эмуляции поведения коалиции агентов (беспилотных летательных аппаратов) при решении общей задачи,
		\item исследование и разработка алгоритмов коммуникации и согласования действий агентов и процессов в модели внешней среды.
	\end{enumerate}
	
	\vspace{0.3em} % When there are two boxes, some whitespace may need to be added if the one on the right has more content
}

\headerbox{Идеальный кандидат}{name=ideal,column=2,span=2,row=0}{
	
	\begin{multicols}{2}
		\vspace{1em}
		\begin{center}
			\includegraphics[width=0.8\linewidth]{placeholder}
			\captionof{figure}{Figure caption}
		\end{center}
		
		Мотивированный, способный к самостоятельному решению технических и исследовательских задач. Необходимы: знание английского (на уровне чтения статей по теме проекта), хорошие навыки программирования на языке высокого уровня (C++, Java, Python) и знания в области проектирования программных систем.
	\end{multicols}
	
	%------------------------------------------------
	
	\begin{multicols}{2}
		\vspace{1em}
		Дополнительным плюсом будут владение средствами web-программирования и знания в области многоагентных систем.
		
		Приветствуются студенты старших курсов и аспиранты.
		
		\begin{center}
			\includegraphics[width=0.8\linewidth]{placeholder}
			\captionof{figure}{Figure caption}
		\end{center}
	
	\end{multicols}
}

\headerbox{References}{name=references,column=1,span=2,above=bottom}{
	
	\renewcommand{\section}[2]{}%
	\renewcommand*{\bibfont}{\scriptsize}
	
	\nocite{*} % Insert publications even if they are not cited in the poster
	\printbibliography

}

\headerbox{Замечания}{name=remarks,column=0,aligned=references,above=bottom}{ 
	
	Еще что-то
	
}

\headerbox{Контактная информация}{name=contacts,column=3,aligned=references,above=bottom}{ 

	\begin{description}\compresslist
	\item[Web] www.isa.ru
	\item[Email] pan@isa.ru
	\item[Phone] +7 (495) 135 1842
	\end{description}
}


\headerbox{О коллективе}{name=collective,column=2,span=2,row=0,below=ideal,above=references}{

	\begin{multicols}{2}
		Продолжительность: начальный контракт на полтора года (с возможным продолжением еще на два года).
		
		Возможно зачисление (на бюджетное место) в аспирантуру на очную форму обучения с предоставлением места в аспирантском общежитии (Дом аспирантов и стажеров РАН).
		Возможно зачисление в штат института или выполнение работы по договору гражданско-правового характера.
		
		Возможно участие в схожих по тематике проектах лаборатории (искусственный интеллект, многоагентные системы, интеллектуальное управление, когнитивное компьютерное моделирование).
		
	\end{multicols}
}

\headerbox{Детали}{name=details,column=0,span=2,below=about,bottomaligned=collective}{ 
	
	Группа планирования целенаправленного поведения и управления, которая работает над данным проектом, включает в себя специалистов по различным направлениям искусственного интеллекта: моделирование когнитивных функций человека, планирование траекторий, интеллектуальное управление. Лаборатория Интеллектуальных динамических систем, в которую входит исследовательская группа проекта, возглавляется президентом Российской ассоциации искусственного интеллекта, д.ф-.м.н. профессором Геннадием Семеновичем Осиповым. В лаборатории ведутся исследования по следующим направлениям: семантический поиск и анализ текстов на естественном языке, интеллектуальная обработка данных и управление, моделирование целенаправленного поведения, доказательная медицина. В лаборатории имеется высококлассное оборудования для осуществления высокопроизводительных вычислений.
	
	Институт системного анализа РАН является признанным лидером в ряде традиционных и новых междисциплинарных направлений отечественной и мировой науки. Основными направлениями теоретических и прикладных исследований Института являются: управление, информатика и информационные технологии, математическое моделирование, искусственный интеллект и принятие решений.
	
}

\end{poster}

\end{document}