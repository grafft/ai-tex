\documentclass[a4paper, 12pt]{article}

\usepackage{geometry}
\usepackage{textcomp}					% "true" символы типа copyright

\usepackage{cmap}						% Улучшенный поиск русских слов в полученном pdf-файле
\usepackage[T2A]{fontenc}				% Поддержка русских букв
\usepackage[utf8]{inputenc}				% Кодировка utf8
\usepackage[english, russian]{babel}	% Языки: русский, английский
\usepackage[unicode]{hyperref}			% Русский язык для оглавления pdf
\usepackage{bookmark}					% Оглавление в pdf
\usepackage{soulutf8}					% Для разрядки

\usepackage{amssymb,amsmath,amsthm}
\usepackage{graphicx} 					% Подключаем пакет работы с графикой
\usepackage[ruled]{algorithm}
\usepackage[noend]{algpseudocode}

\usepackage{titlesec}					% Форматирование заголовков
%\usepackage{abstract}					% Форматрирование абстракта	

\geometry{a4paper,top=2cm,bottom=2cm,left=2.5cm,right=1cm}	% Геомтерия страницы
\graphicspath{{../../images/}} 			% Пути к изображениям

% Настройка теоремоподобных окружений
\theoremstyle{plain}
\newtheorem{Theorem}{Теорема}
\newtheorem{Lemma}[Theorem]{Лемма}
\newtheorem{Pred}{Утверждение}
\newtheorem{Corollary}{Следствие}
\newtheorem{Def}{Определение}
\newenvironment{Proof}%
	{\par\noindent{\bf Доказательство.}}%
	{\hfill$\scriptstyle\blacksquare$}

\floatname{algorithm}{}
\newcommand{\algrule}[1][.2pt]{\par\vskip.2\baselineskip\hrule height #1\par\vskip.2\baselineskip}
\algrenewcommand\algorithmicrequire{\textbf{Вход:}}
\algrenewcommand\algorithmicensure{\textbf{Выход:}}
\algrenewcommand\algorithmicforall{\textbf{для всех}}
\algrenewcommand\algorithmicwhile{\textbf{пока}}
\algrenewcommand\algorithmicif{\textbf{если}}
\algrenewcommand\algorithmicthen{\textbf{то}}
\algrenewcommand\algorithmicelse{\textbf{иначе}}
\algrenewcommand\algorithmicreturn{\textbf{вернуть}}
\algrenewcommand\algorithmicfunction{\textbf{процедура}}
\algrenewcommand\algorithmicdo{}
\renewcommand{\algorithmiccomment}[1]{{\quad\sl // #1}}

\renewcommand\labelenumi{\theenumi )}	% Нумерованный перечень со скобками
\AtBeginDocument{\renewcommand{\abstractname}{\vspace{-2\baselineskip}}}    		% clear the title
%\renewcommand{\absnamepos}{empty} % originally center

\makeatletter
	\bibliographystyle{gost2008p}
	\renewcommand{\@biblabel}[1]{#1.}	% Заменяем библиографию с квадратных скобок на точку:
\makeatother

\titleformat{\section}[runin]{\normalfont\bfseries}{\thesection.}{1pt}{}[.]
\titleformat{\subsection}[runin]{\normalfont}{\thesubsection.}{1pt}{\so}[.]
%\titleformat{command}[shape]{format}			   {label}		 {sep}{before}[after]

\DeclareMathOperator*{\argmax}{arg\,max}
\DeclareMathOperator*{\argmin}{arg\,min}

\title{\hbox{\normalsize\textit{УДК 004.81}}\hbox{}\textbf{\Large\MakeUppercase{Управление поведением как функция сознания. II. Самосознание и синтез плана}}\footnote{Работа выполнена при поддержке РНФ (грант \No\ 14-11-00692).}}
\author{\textbf{\textcopyright~2015~г. Г.\,С.~Осипов, А.\,И.~Панов, Н.\,В.~Чудова}\\\normalsize\textit{Москва, Институт системного анализа РАН}}
\date{}

\begin{document}
	\vspace*{-5\baselineskip}			% Убираем лишние пробелы перед заголовком статьи
	{\let\newpage\relax\maketitle}
	
	\begin{abstract}
		\noindent Рассматривается семантический уровень описания функций, которые в психологии принято относить к функциям сознания и самосознания. Исследуется механизм работы компонент знака, введённых в первой части статьи.На основе описания знака на семантическом уровне исследуется сходимость основного итерационного процесса образования знака "--- связывания образной компоненты знака и его значения. Введение алгоритмов работы компонент знака позволяет построить алгоритм процесса синтеза плана поведения, а также построить новую архитектуру интеллектуальных агентов, обладающих, в частности, способностями к распределению ролей в коалициях.
	\end{abstract}	
	
	\section*{Введение}
	Связь с первой статьёй \cite{PanovA2014a}. Напомнить про строение знака и нейрофизиологические исследования.
	
	\section{Семантические уровень}
	\subsection{Определение компонент знака}
	Таким образом, $R$-автомат $R_i^j$ является бесконечным автоматом Миля с переменной структурой и конечной памятью и определяется следующим набором $R_i^j=<X_i^j\times \hat{X}_i^{j+1}, 2^{\mathcal Z_i^j}, X_i^{*j}\times \hat{X}_i^j,\varphi_i^j,\vec\eta_i^j,>$, где
	\begin{itemize}
		\item $X_i^j$ "--- множество входных сигналов, 
		\item $X_i^{*j}$ "--- множество выходных сигналов, 
		\item $\hat{X}_i^{j+1}$ "--- множество управляющих сигналов с верхнего уровня иерархии,
		\item $\hat{X}_i^j$ "--- множество управляющих сигналов на нижний уровень иерархии,
		\item $2^{\mathcal Z_i^j}$ "--- множество состояний (множество подмножеств множества матриц предсказания),
		\item $\varphi_i^j:X_i^j\times \hat{X}_i^{j+1}\to 2^{\mathcal Z_i^j}$ "--- функция переходов,
		\item $\vec\eta_i^j:2^{\mathcal Z_i^j} \to X_i^{*j}\times \hat{X}_i^j$ "--- вектор"--~функция выходов.
	\end{itemize}
	
	\begin{algorithm}[H]
		\caption{Алгоритм $\mathfrak{A}_{th}$}\label{alg:th}
		\begin{algorithmic}[1]
			\Require $\tau_s, \hat{x}_i^{j+1}(\tau_s), \omega_i^j$;
			\Ensure $\varphi_{i\Delta t}^j, \vec\eta_{i\Delta t}^j$;
			\algrule
			\State $\hat{F}^*=\varnothing,Z^*=\varnothing,t=0$; \Comment{активные функции распознавнаия и матрицы предсказания}
			\State $c_1\in(0,1), c_2\in(0,1)$; \Comment{пороговые константы}
			
			\Statex \Comment{определение начального состояния}
			
			\ForAll{компонент $\hat{x}_{ik}^{j+1}$ вектора $\hat{x}_i^{j+1}(\tau_s)=(\hat{x}_{i1}^{j+1},\hat{x}_{i2}^{j+1},\dots,\hat{x}_{il}^{j+1})$} \label{alst:init_start}
			\If{$\hat{x}_{ik}^{j+1}{\ge}c_1$} \label{alst:select_f}
			\State $\hat{F}^*:=\hat{F}^*\cup\{\hat{f}_k\}$;
			\EndIf
			\EndFor
			
			\State $\bar x_i^j:=\omega_i^j(\tau_s)$;
			
			\ForAll{функций распознавания $\hat{f}_k\in\hat{F}^*$}
			\ForAll{$Z_r^k\in\mathcal{Z}_k$, соответствующих функции распознавания $\hat{f}_k$,}
			\If{$\frac{\|\bar{z}_1^r-\bar{x}_i^j\|}{\|\bar{z}_1^r\|+\|\bar{x}_i^j\|}<c_2$} \label{alst:select_z}
			\State $Z^*:=Z^*\cup\{Z_r^k\}$;
			\EndIf
			\EndFor
			\EndFor
			
			\State $\varphi_i^j(\bar x_i^j,\hat{x}_i^{j+1}(\tau_s)) := Z^*$; \Comment{значение функции переходов в начальный момент времени}\label{alst:init_state}
			\State $\bar N:=(|\{Z_r^1|Z_r^1\in Z^*\}|,\dots,|\{Z_r^{l_i^j}|Z_r^{l_i^j}\in Z^*\}|)$; \label{alst:init_calc_out1}
			\State $\eta(Z^*)=\bar{x}_i^{*j}:=W(\bar N)$; \Comment{значение функции выходов в начальный момент времени} \label{alst:init_calc_out3}
			\State $\hat x_i^j=W(\sum_{\hat f_k\in\hat F^*}\hat x_{ik}^{j+1}\sum_{Z_r^k\in Z^*}\bar z_2^r)$; \label{alst:init_end}
			
			\Statex \Comment{оновной цикл}
			\State $t=1$;
			\While{$t\leqslant{h_i^j}-1$} \label{alst:cycle_start}
			\State $\bar{x}_i^j:=\omega(\tau_s+t)$;
			
			\ForAll{матриц предсказания $Z_r^k$ из множества $Z^*$}
			\If{$\frac{\|\bar{z}_{t+1}^r-\bar{x}_i^j\|}{\|\bar{z}_{t+1}^r\|+\|\bar{x}_i^j\|}\geqslant{c_2}$} \label{alst:update_z}
			\State $Z^*:=Z^*\setminus\{Z_r^k\}$;
			\EndIf
			\EndFor
			
			\State $\varphi_i^j(\bar x_i^j,\hat{x}_i^{j+1}(\tau_s)) := Z^*$; \Comment{значение функции переходов в момент времени $t$}\label{alst:calc_state2}
			\State $\bar N=(|\{Z_r^1|Z_r^1\in Z^*\}|,\dots,|\{Z_r^{l_i^j}|Z_r^{l_i^j}\in Z^*\}|)$; \label{alst:calc_out1}
			\State $\eta(Z^*)=\bar{x}_i^{*j}:=W(\bar N)$;\Comment{значение функции выходов в момент времени $t$} \label{alst:calc_out3}
			
			\State $t=t+1$;
			\If{$t\leqslant{h}_i^j-2$}
			\State $\hat{x}_i^j:=W(\sum_{\hat f_k\in\hat F^*}\hat x_{ik}^{j+1}\sum_{Z_r^k\in Z^*}\bar z_t^r)$; \label{alst:calc_state1}
			\EndIf
			\EndWhile \label{alst:cycle_end}
		\end{algorithmic}	
	\end{algorithm}
	\subsection{Семантический уровень обобщения}
	
	\section{Алгоритм образования знака}

	\begin{algorithm}
		\caption{Алгоритм $\mathfrak{A}_{pm}$}\label{alg:pm}
		\begin{algorithmic}[1]
			\Require $\tilde m^0=\{f_p\}, \Psi_p^m, \hat F\subseteq \{f_k\}$;
			\algrule
			
			\State $\tilde p^{*(0)} := \varnothing$;
			\State $Z^{*(0)} := \varnothing$;
			\State $t := 0$;
			\ForAll{$f^{(t)}\in \hat F$}
			\If{$\exists \tilde m^{(t)}\in \tilde M$ такое, что $(\tilde p(f^{(t)}),\tilde m^{(t)})\in\Psi_p^m$ \textbf{and} $\tilde m^{(t)}$ выполним в условиях признака $f_p$ \textbf{and} $\nexists f: f\in\tilde p^{*(t)},(\tilde p(f),\tilde m(f))\in\Psi_p^m, \tilde m^0$ конфликтует с $\tilde m^{(t)}$}
			\State $\tilde p^{*(t)}=\tilde p^{*(t)}\cup\{f^{(t)}\}$;
			
			\If{$\exists R_i^j$ такой, что $f^{(t)}\in F_i^j$}
			\State $R_i^{j(t)}:=R_i^j$;
			\Else
			\State $R_i^{j(t)}:=\argmax\limits_{\{R\}} (F_i^j\cap\tilde p^{(t)}), F_i^{j(t)}:=F_i^{j(t)}\cup f^{(t)}$;
			\EndIf
			
			\State $\bar z_s:=(z_{s1},z_{s2},\dots,z_{sq}), z_{sk}=1$, если $k$ -- индекс признака $f^{(t)}$ во входном векторе распознающего блока $R_i^{j(t)}$ и $z_{sk}=0$ иначе;
			\State $Z^{*(t)}:=Z^{*(t)}\cup\bar z_s$;
			\State $Z_p^{(t)}:=(\bar z_1^{c(t)},\bar z_2^{e(t)},\dots,\bar z_{2\cdot k-1}^{c(t)},\bar z_{2\cdot k}^{e(t)})$, где $\bar z_i^{c(t)}=\bigvee\limits_{\tilde m_j^{(t)}}(\bar z_j^{c(t)}\rightarrow F_p^j),$ 
			\\\hspace{3.0cm}$\bar z_i^{e(t)}=\bigvee\limits_{\tilde m_j^{(t)}}(\bar z_j^{e(t)}\Rightarrow\bar z_j^e)$;
			\EndIf
			
			\State $\tilde m^{*(t)}=\{f_p^{(t)}\}$;
			\State $\mathcal Z^{*(t)}=\{Z^{*(t)}\}$;
			\State $t=t+1$;
			\EndFor
			
			\Return $\Psi_p^m$, определённая на паре $(\tilde p, \tilde m)$, где $\tilde p=\lim\limits_{t\rightarrow|\hat F|}\tilde p^{*(t)}$, $\tilde m=\lim\limits_{t\rightarrow|\hat F|}\tilde m^{*(t)}$, $f^*, Z^*=\lim\limits_{t\rightarrow|\hat F|}Z^{*(t)},\mathcal Z^*=\{Z^*\}$;
		\end{algorithmic}			
	\end{algorithm}
	
	\section{Самосознание и его функции} Здесь я бы предложил дать психологическое описание функций самосознания и определения функций оценки $\Phi_a$ и $\Phi_p$.
	
	\section{Алгоритм планирования}	Планом $Plan$ будем называть такую последовательность пар <<ситуация "--- действие>> , в которой.
	
	Целевая ситуация строится исходя из образа действия, связанного с личностным смыслом, который был определён в процессе целеполагания для целевого знака.
	\begin{algorithm}
		\caption{Алгоритм $\mathfrak{A}_{bp}$}\label{alg:beh_plan}
		\begin{algorithmic}[1]
			\Require начальная ситуация $F_{sit}$, целевая ситуация $F_{goal}$, функции оценки $\Phi_a$ и $\Phi_p$;
			\Ensure план $Plan$;
			\algrule
			\State $Plan=\Call{Planning}{\varnothing,F_{goal}}$;
			
			\Function{Planning}{$Plan, F_{cur}$}
				\State $\Delta=F_{sit}\setminus F_{cur}$; \Comment{текущая невязка состояний}
				
				\State $F_{for} = \argmin\limits_{F\in 2^{F_{sit}}}|\bigcap\limits_{f_p\in F}F_A(f_p)\setminus\Delta|$; \Comment{находим множество наиболее подходящих параллельных действий}
				\ForAll $f_j\in F_{for}$
					\If{$\exists f_k\in F_{for}$ такой, что $f_k\not =f_i$ и $f_k$ конфликтует с $f_j$}
						\State $F_{for}= F_{for}\setminus\{f_k\}$; \Comment{Удаляем конфликтующие признаки}
					\EndIf
				\EndFor
				
				\State $F_a^{for} = \varnothing$; \Comment{текущее множестов личностных смыслов}
				\ForAll $f_p\in F_{for}$
					\State $F_a^{for} = F_a^{for}\cup \{\Call{Interior}{f_p}\}$;\Comment{интериоризация значения}
				\EndFor
				\State $\tilde F_a^{for}=\Phi_a(F_a^{for},f_{goal})$; \Comment{выбор предпочитаемых действий}
				\If{$\bigcup\limits_{f\in \tilde F_a^{for}}F_C(f)\subseteq F_{sit}$}
					\State \Return $Plan\cup{\tilde F_a^{for}}$;		\Comment{возвращаем обновленный план}
				\Else
					\State $\Delta^* = \Phi_p(\Delta, f_{goal})$; \Comment{Ранжирование критических признаков}
					\State $\tilde F_a^{back} = \varnothing$; 
					\ForAll $f_k\in\Delta^*$ 
						\State $m_k = \tilde m(f_k)$; \Comment{определение значение $k$-го знака}
						\State $F_a^{back} = \varnothing$;
						\ForAll $f_p\in m_k$
							\State $F_a^{back}=F_a^{back}\cup\{\Call{Interior}{f_p}\}$;
						\EndFor 
						\State $\tilde F_a^{back}=\tilde F_a^{back}\cup\Phi_a(F_a^{back}, f_{goal})$; \Comment{выбор предпочитаемых действий}
					\EndFor
					
					\ForAll $f_j\in \tilde F_a^{back}$
						\If{$\exists f_k\in \tilde F_a^{back}$ такой, что $f_k\not =f_i$ и $f_k$ конфликтует с $f_j$}
							\State $\tilde F_a^{back} = \tilde F_a^{back}\setminus\{f_k\}$; \Comment{Удаляем конфликтующие признаки}
						\EndIf
					\EndFor
					
					
					\If{$\Delta\not\subseteq\bigcup\limits_{f\in\tilde F_a^{back}}F_A(f)$}
						\State\Return невозможно построить план;
					\Else
						\State \Return \Call{Planning}{$Plan, \bigcup\limits_{f\in F_a^{back}}F_C(f)$};						
					\EndIf
				\EndIf

			\EndFunction
		\end{algorithmic}
	\end{algorithm}
	
	\section*{Заключение}
	
	\titleformat{\section}{\normalfont\centering\MakeUppercase}{\thesection.}{1pt}{}[]
	
	%	\nocite{*}
	\inputencoding{cp1251}
	\bibliography{../../biblio/main}
	\inputencoding{utf8}
\end{document}