\documentclass[a4paper, 12pt]{article}

\usepackage{geometry}
\usepackage{textcomp}					% "true" символы типа copyright

\usepackage{cmap}						% Улучшенный поиск русских слов в полученном pdf-файле
\usepackage[T2A]{fontenc}				% Поддержка русских букв
\usepackage[utf8]{inputenc}				% Кодировка utf8
\usepackage[english, russian]{babel}	% Языки: русский, английский
\usepackage[unicode]{hyperref}			% Русский язык для оглавления pdf
\usepackage{bookmark}					% Оглавление в pdf
\usepackage{soulutf8}					% Для разрядки

\usepackage{amssymb,amsmath,amsthm}
\usepackage{graphicx} 					% Подключаем пакет работы с графикой
\usepackage[ruled]{algorithm}
\usepackage[noend]{algpseudocode}

\usepackage{color}

\usepackage{titlesec}					% Форматирование заголовков
%\usepackage{abstract}					% Форматрирование абстракта	

\geometry{a4paper,top=2cm,bottom=2cm,left=2.5cm,right=1cm}	% Геомтерия страницы
\graphicspath{{../../images/}} 			% Пути к изображениям

% Настройка теоремоподобных окружений
\theoremstyle{plain}
\newtheorem{Theorem}{Теорема}
\newtheorem{Lemma}[Theorem]{Лемма}
\newtheorem{Pred}{Утверждение}
\newtheorem{Corollary}{Следствие}
\newtheorem{Def}{Определение}
\newenvironment{Proof}%
	{\par\noindent{\bf Доказательство.}}%
	{\hfill$\scriptstyle\blacksquare$}

\floatname{algorithm}{}
\newcommand{\algrule}[1][.2pt]{\par\vskip.2\baselineskip\hrule height #1\par\vskip.2\baselineskip}
\algrenewcommand\algorithmicrequire{\textbf{Вход:}}
\algrenewcommand\algorithmicensure{\textbf{Выход:}}
\algrenewcommand\algorithmicforall{\textbf{для всех}}
\algrenewcommand\algorithmicwhile{\textbf{пока}}
\algrenewcommand\algorithmicif{\textbf{если}}
\algrenewcommand\algorithmicthen{\textbf{то}}
\algrenewcommand\algorithmicelse{\textbf{иначе}}
\algrenewcommand\algorithmicreturn{\textbf{вернуть}}
\algrenewcommand\algorithmicfunction{\textbf{процедура}}
\algrenewcommand\algorithmicdo{}
\renewcommand{\algorithmiccomment}[1]{{\quad\sl // #1}}

\renewcommand\labelenumi{\theenumi )}	% Нумерованный перечень со скобками
\AtBeginDocument{\renewcommand{\abstractname}{\vspace{-2\baselineskip}}}    		% clear the title
%\renewcommand{\absnamepos}{empty} % originally center

\makeatletter
	\bibliographystyle{gost2008p}
	\renewcommand{\@biblabel}[1]{#1.}	% Заменяем библиографию с квадратных скобок на точку:
\makeatother

\titleformat{\section}[runin]{\normalfont\bfseries}{\thesection.}{1pt}{}[.]
\titleformat{\subsection}[runin]{\normalfont}{\thesubsection.}{1pt}{\so}[.]
%\titleformat{command}[shape]{format}			   {label}		 {sep}{before}[after]

\DeclareMathOperator*{\argmax}{arg\,max}
\DeclareMathOperator*{\argmin}{arg\,min}

\title{\hbox{\normalsize\textit{УДК 004.81}}\hbox{}\textbf{\Large\MakeUppercase{Управление поведением как функция сознания. II. Самосознание и синтез плана}}\footnote{Работа выполнена при поддержке РНФ (грант \No\ 14-11-00692).}}
\author{\textbf{\textcopyright~2015~г. Г.\,С.~Осипов, А.\,И.~Панов, Н.\,В.~Чудова}\\\normalsize\textit{Москва, Институт системного анализа РАН}}
\date{}

\begin{document}
	\vspace*{-5\baselineskip}			% Убираем лишние пробелы перед заголовком статьи
	{\let\newpage\relax\maketitle}
	
	\begin{abstract}
		\noindent Рассматривается семантический уровень описания функций, которые в психологии принято относить к функциям сознания и самосознания. Исследуется механизм работы компонент знака, введённых в первой части статьи.На основе описания знака на семантическом уровне исследуется сходимость основного итерационного процесса образования знака "--- связывания образной компоненты знака и его значения. Введение алгоритмов работы компонент знака позволяет построить алгоритм процесса синтеза плана поведения, а также построить новую архитектуру интеллектуальных агентов, обладающих, в частности, способностями к распределению ролей в коалициях.
	\end{abstract}	
	
	\section*{Введение}Связь с первой статьёй.
	
	\noindent\colorbox{yellow}{
		\parbox{\dimexpr\linewidth-2\fboxsep}{Правильное введение и плавный переход к образной компоненте.}
	}
	
	Как было подробно изложено в первой части статьи \cite{PanovA2014a}, в качестве базовых психологических теорий, в которых даётся не только качественное описание свойств когнитивных функций, но и приводятся структурные описания лежащих в их основе психических образований, в предложенном подходе  использованы культурно-исторический подход Выготского"--~Лурии \cite{Luria1970,Vygotsky2005}, теория деятельности Леонтьева \cite{Leontiev1975} и модель психики Артемьевой \cite{Artemyeva1980}. Согласно приведённым теориям высшие сознательные когнитивные функции осуществляются в рамках так называемой мотивированной предметной деятельности, когда объекты и процессы внешней  среды опосредованы для субъекта специальными образованиями, называемыми знаками. Процесс задействования знака в той или иной когнитивной функции имеет три образующих: образ, значение и личностный смысл. Образная составляющая отвечает за функции воспроизведения и отличения опосредуемого предмета или процесса в ходе деятельности. Составляющая значения представляет собой место данного знака в той или иной надпсихологической знаковой системе, которая отражает в функциональном смысле наработанные общей исторической практикой коллектива"--~владельца данной знаковой системы способы использования опосредуемого предмета и процесса. Наконец, составляющая личностного смысла несёт в себе собственный опыт действования субъекта с денотатом знака, который выражается в том числе и в интегральной оценке роли этого денотата в его текущей деятельности: способствует ли данные процесс или объект удовлетворению текущего мотива.
	
	Трёхкомпонентная структура элемента индивидуального знания, которая как было сказано выше, в психологии называется знаком, подтверждается и теми работами нейрофизиологов, в которых предпринимается попытка построить общую теорию работы мозга человека. Так в теории повторного входа Эделмена \cite{Edelmen1981} и Иваницкого \cite{Ivanitsky1996} утверждается, что образование осознанного ощущения или фиксация входного потока информации происходит только в том случае, когда активированное сенсорным входом возбуждение через ассоциативные зоны коры от гиппокампа, а затем от гипоталамуса накладывается на сенсорный след в проекционной коре. Такой <<круг ощущений>> \ref{fig:ivan_cyrcle}, проходящий за характерное время в 150-300 мс, последовательно активирует три компоненты индивидуального знания: образную (проекционная и сенсорная зоны коры), компоненту значения (гиппокамп) и личностного смысла (гипоталамус).
	
	\begin{figure}[h]
		\centering
		\includegraphics[width=0.7\linewidth]{ivanitsky_cyrcle}
		\caption{<<Круг ощущений>> по Иваницкому (адаптировано из \cite{Ivanitsky1996}).}
		\label{fig:ivan_cyrcle}
	\end{figure}
	
	Кроме того, по современным нейрофизиологическим представлениям строение коры головного мозга практически однородно во всем своем объеме (наличие колонок некоротекса). При этом связи между достаточно малыми зонами коры (так называемый коннектом), явно указывают на иерархичность ее строения и на присутствие как восходящих, так и обратных, нисходящих связей. Отсюда следует, что компоненты элемента индивидуального знания должны обладать иерархическим однородным строением с восходящими потоками информации и нисходящей обратной связью. Кроме того, образная компонента должна иметь такую функцию распознавания, которая кроме категоризации статических объектов и динамических процессов использует обратную связь для предсказания сигнала в следующий момент времени.
	
	
	\section{Семантические уровень}
	В качестве математической модели базовой составляющей всех компонент элемента индивидуального знания был предложен следующий бесконечный автомат Миля с переменной структурой и конечной памятью (распознающий автомат или $R$-автомат):  
	\begin{equation}
		R_i^j=<X_i^j\times \hat{X}_i^{j+1}, 2^{\mathcal Z_i^j}, X_i^{*j}\times \hat{X}_i^j,\varphi_i^j,\vec\eta_i^j,>,
	\end{equation}
	где (см. рис. \ref{fig:rb_io})
	\begin{itemize}
		\item $X_i^j$ "--- множество входных сигналов, 
		\item $X_i^{*j}$ "--- множество выходных сигналов, 
		\item $\hat{X}_i^{j+1}$ "--- множество управляющих сигналов с верхнего уровня иерархии,
		\item $\hat{X}_i^j$ "--- множество управляющих сигналов на нижний уровень иерархии,
		\item $2^{\mathcal Z_i^j}$ "--- множество состояний (множество подмножеств множества матриц предсказания),
		\item $\varphi_i^j:X_i^j\times \hat{X}_i^{j+1}\to 2^{\mathcal Z_i^j}$ "--- функция переходов,
		\item $\vec\eta_i^j:2^{\mathcal Z_i^j} \to X_i^{*j}\times \hat{X}_i^j$ "--- вектор"--~функция выходов.
	\end{itemize}
	Множество входных признаков распознающего автомата $R_i^j$ будем обозначать $F_i^j$, множество выходных "--- $F_i^{*j}$.
	
	В качестве функции распознавания $k$-ого выходного признака $\hat f_k$ в $R$-автомате удобно использовать набор булевых матриц предсказания $Z_k=\{Z_1^k,Z_2^k,\dots,Z_m^k\}$, в которых каждый столбец $\bar z_u^r$ является вектором предсказания входных признаков в момент времени $\tau_{s+u}$, где $\tau_s$ "--- начало вычислительного цикла (момент действия управляющего сигнала $\bar x_i^{j+1}$). Сама матрица $Z_r^k$ задаёт последовательность битовых векторов, наличие бита в котором свидетельствует о присутствии распознаваемого функцией $\hat f_k$ признака. Алгоритм $\mathcal A_{th}$ вычисления функции переходов $\varphi_i^j$ и выходной функции $\vec\eta_i^j$ по начальному моменту времени $\tau_s$, управляющему воздействию $\hat x_i^{j+1}(\tau_s)$ и входному воздействию $\omega_i^j$ представлен ниже.
	
	\begin{algorithm}[H]
		\caption{Алгоритм $\mathfrak{A}_{th}$}\label{alg:th}
		\begin{algorithmic}[1]
			\Require $\tau_s, \hat{x}_i^{j+1}(\tau_s), \omega_i^j$;
			\Ensure $\varphi_{i\Delta t}^j, \vec\eta_{i\Delta t}^j$;
			\algrule
			\State $\hat{F}^*=\varnothing,Z^*=\varnothing,t=0$; \Comment{активные функции распознавнаия и матрицы предсказания}
			\State $c_1\in(0,1), c_2\in(0,1)$; \Comment{пороговые константы}
			
			\Statex \Comment{определение начального состояния}
			
			\ForAll{компонент $\hat{x}_{ik}^{j+1}$ вектора $\hat{x}_i^{j+1}(\tau_s)=(\hat{x}_{i1}^{j+1},\hat{x}_{i2}^{j+1},\dots,\hat{x}_{il}^{j+1})$} \label{alst:init_start}
			\If{$\hat{x}_{ik}^{j+1}{\ge}c_1$} \label{alst:select_f}
			\State $\hat{F}^*:=\hat{F}^*\cup\{\hat{f}_k\}$;
			\EndIf
			\EndFor
			
			\State $\bar x_i^j:=\omega_i^j(\tau_s)$;
			
			\ForAll{функций распознавания $\hat{f}_k\in\hat{F}^*$}
			\ForAll{$Z_r^k\in\mathcal{Z}_k$, соответствующих функции распознавания $\hat{f}_k$,}
			\If{$\frac{\|\bar{z}_1^r-\bar{x}_i^j\|}{\|\bar{z}_1^r\|+\|\bar{x}_i^j\|}<c_2$} \label{alst:select_z}
			\State $Z^*:=Z^*\cup\{Z_r^k\}$;
			\EndIf
			\EndFor
			\EndFor
			
			\State $\varphi_i^j(\bar x_i^j,\hat{x}_i^{j+1}(\tau_s)) := Z^*$; \Comment{значение функции переходов в начальный момент времени}\label{alst:init_state}
			\State $\bar N:=(|\{Z_r^1|Z_r^1\in Z^*\}|,\dots,|\{Z_r^{l_i^j}|Z_r^{l_i^j}\in Z^*\}|)$; \label{alst:init_calc_out1}
			\State $\eta(Z^*)=\bar{x}_i^{*j}:=W(\bar N)$; \Comment{значение функции выходов в начальный момент времени} \label{alst:init_calc_out3}
			\State $\hat x_i^j=W(\sum_{\hat f_k\in\hat F^*}\hat x_{ik}^{j+1}\sum_{Z_r^k\in Z^*}\bar z_2^r)$; \label{alst:init_end}
			
			\Statex \Comment{оновной цикл}
			\State $t=1$;
			\While{$t\leqslant{h_i^j}-1$} \label{alst:cycle_start}
			\State $\bar{x}_i^j:=\omega(\tau_s+t)$;
			
			\ForAll{матриц предсказания $Z_r^k$ из множества $Z^*$}
			\If{$\frac{\|\bar{z}_{t+1}^r-\bar{x}_i^j\|}{\|\bar{z}_{t+1}^r\|+\|\bar{x}_i^j\|}\geqslant{c_2}$} \label{alst:update_z}
			\State $Z^*:=Z^*\setminus\{Z_r^k\}$;
			\EndIf
			\EndFor
			
			\State $\varphi_i^j(\bar x_i^j,\hat{x}_i^{j+1}(\tau_s)) := Z^*$; \Comment{значение функции переходов в момент времени $t$}\label{alst:calc_state2}
			\State $\bar N=(|\{Z_r^1|Z_r^1\in Z^*\}|,\dots,|\{Z_r^{l_i^j}|Z_r^{l_i^j}\in Z^*\}|)$; \label{alst:calc_out1}
			\State $\eta(Z^*)=\bar{x}_i^{*j}:=W(\bar N)$;\Comment{значение функции выходов в момент времени $t$} \label{alst:calc_out3}
			
			\State $t=t+1$;
			\If{$t\leqslant{h}_i^j-2$}
			\State $\hat{x}_i^j:=W(\sum_{\hat f_k\in\hat F^*}\hat x_{ik}^{j+1}\sum_{Z_r^k\in Z^*}\bar z_t^r)$; \label{alst:calc_state1}
			\EndIf
			\EndWhile \label{alst:cycle_end}
		\end{algorithmic}	
	\end{algorithm}
		
	В алгоритме используется стандартная функция $W$ нормировки весовых значений:
	\begin{equation}
		W(\bar x)=\left(\frac{x_1}{\max\limits_i x_i},\dots,\frac{x_n}{\max\limits_i x_i}\right),
	\end{equation} 
	где $\bar x=(x_1,\dots,x_n)$ "--- вектор с ненормированными компонентами.
	
	\begin{figure}[H]
		\centering
		\includegraphics[width=1.0\linewidth]{rb_io}
		\caption{Входные и выходные сигналы распознающего автомата.}
		\label{fig:rb_io}
	\end{figure}
	
	\subsection{Определение компонент знака}
		Для определения компонент знака через описанный в предыдущем разделе $R$-автомат необходимо ввести ряд вспомогательных понятий. 
		
		Введём семейство бинарных отношений $\{\sqsubset,\sqsubset^1,\sqsubset^2,\dots\}$, определённых на декартовом произведении $\{f_k\}\times \{f_k\}$. Будем считать, что <<признак $f_1$ является составляющим признака $f_2$>> или <<признак $f_2$ измеряется по признаку $f_1$>>, $(f_1,f_2 )\in\sqsubset$ или $f_1\sqsubset f_2$, в том случае, если $f_1\dashv R_1^j, f_2\dashv R_2^{j+1}$, $R_2^{j+1}$ "--- родительский блок по отношению к $R_1^j$ и в множестве матриц предсказания $\mathcal Z_2$ признака $f_2$ существует как минимум одна матрица $Z_r^2$, содержащая некоторый столбец $\bar z_u^r$ с элементом $z_{uv}^r\not=0$, где $v$ "--- индекс признака $f_1$ во входном векторе для распознающего блока $R_2^{j+1}$.
		
		Пара признаков $(f_1,f_2)\in\sqsubset^t$ или $f_1\sqsubset^t f_2$, где $t\in\{1,2,\dots\}$, в~том случае, если $f_1\dashv R_1^j, f_2\dashv R_2^{j+1}$, $R_2^{j+1}$ "--- родительский блок по отношению к $R_1^j$ и в множестве матриц предсказания $\mathcal Z_2$ признака $f_2$ существует как минимум одна матрица $Z_r^2$, содержащая $t$–ый столбец $\bar z_t^r$ с~элементом $z_{tv}^r\not=0$, где $v$ "--- индекс признака $f_1$ во~входном векторе для распознающего блока $R_2^{j+1}$.
		
		Каждый элемент векторов"--~столбцов соотносится с~признаком из~входного множества признаков распознающего блока, что означает задание типа для каждого элемента вектора"--~столбца. Будем обозначать тип $k$-го элемента вектора-столбца распознающего блока $R_i^j$ как $f_i^j(k)\in F_i^j$, $k\in(1,q_i^j)$. 
		
		Введём два выделенных из множества $\{f_k\}$ признака: $f_c$ "--- <<условие>> и $f_e$ "--- <<эффект>>, распознаваемые одним распознающим блоком $R_0^1$: $F_0^{*1}=\{f_c,f_e\}$. Те признаки, которые распознаются распознающими блоками, выступающими родительскими по отношению к блоку $R_0^1$, будем называть процедурными признаками, остальные "--- объектными признаками.
		
		Для любого процедурного признака выполняются следующие естественные условия:
		\begin{itemize}
			\item условие всегда предшествует эффекту,
			\item условие всегда влечёт за собой эффект и
			\item все условия всегда отделены от своих эффектов.
		\end{itemize}
		
		Иными словами, если $f_1$ "--- процедурный признак, то если в столбце $\bar z_u^r$ матрицы предсказания $Z_r^1$ элемент $z_{uv}^r$, соответствующий признаку $f_c$, не равен $0$, то в этом столбце соответствующий признаку $f_e$ элемент вектора "--- нулевой, в столбце $z_{u+t}^r, t>0$ наоборот "--- элемент $z_{u+t,v}^r$, соответствующий признаку $f_c$, равен $0$, а соответствующий признаку $f_e$ элемент "--- не нулевой. Те столбцы матрицы предсказания $Z$, в которых соответствующий признаку $f_e$ элемент вектора не нулевой, будем называть \textit{столбцами эффектов}, а те столбцы матрицы предсказания $Z$, в которых не равен нулю элемент вектора, соответствующий признаку $f_c$ "--- \textit{столбцами условий}. 
		
		Пополним семейство отношений $\{\sqsubset,\sqsubset^1,\sqsubset^2,\dots\}$ двумя отношениями: $\sqsubset^c$ и $\sqsubset^e$, принадлежность к~которым пары признаков $(f_1,f_2)$ свидетельствует о~том, что признак $f_1$ присутствует соответственно в~столбце условий и эффектов как минимум в~одной матрице предсказания процедурного признака $f_2$.
		
		\begin{Def}
			Если $f_1$ "--- признак, то подмножество $\tilde p(f_1)$ множества $\{f_k\}$ таких признаков, что $\forall f_i\in\tilde p(f_1) f_i\sqsubset f_1$, будем называть перцептом признака $f_1$.
		\end{Def}
		
		На множестве всех перцептов $\tilde P$ введём величину $\rho_p(\tilde p(f_1),\tilde p(f_2))$, вычисляемую по~следующему правилу:
		\begin{itemize}
			\item если $f_1$ и $f_2$ распознаются разными распознающими блоками, т.~е. $f_1\dashv R_1^j, f_2\dashv R_2^i$, то $\rho_p(\tilde p(f_1),\tilde p(f_2))=\infty$,
			\item если $f_1$ и $f_2$ распознаются одним и тем~же распознающим блоком $R_1^j$ со~множеством входных признаков $F_1^j$ мощности $q$ и характерным временем $h$, то
			\begin{equation}
				\rho_p(\tilde p(f_1),\tilde p(f_2))=\min\limits_{\substack{Z_r^1\in Z_1\\Z_s^2\in Z_2}}\frac{1}{q\cdot h}\sum\limits_{u=1}^h\|\bar z_u^r-\bar z_u^s\|.
			\end{equation} 
		\end{itemize}
		
		\begin{Pred}
			Величина $\rho_p$ является метрикой на множестве перцептов $\tilde P$.
		\end{Pred}
		
		\begin{Def}
			Если $f_1$ "--- признак, $f_2$ "--- процедурный признак, $f_1\sqsubset^c f_2$, то будем называть $f_2$ функциональным значением признака $f_1$. Множество всех функциональных значений признака $f_1$ будем обозначать $\tilde m(f_1)$.
		\end{Def}
		
		На множестве всех функциональных значений $\tilde M$ введём величину $\rho_m(\tilde m(f_1),\tilde m(f_2))$, вычисляемую по следующему правилу:
		\begin{equation}
			\rho_m(\tilde m_1(f_1),\tilde m_2(f_2 ))=\min\limits_{\substack{f_i\in\tilde m(f_1 )\\f_j\in\tilde m(f_2 )}}\rho_p(\tilde p(f_i ),\tilde p(f_j )).
		\end{equation}
		
		\begin{Pred}
			Величина $\rho_m$ является метрикой на множестве функциональных значений $\tilde M$.
		\end{Pred}
		
	\subsection{Семантический уровень обобщения} Определение ролевой структуры для алгоритма планирования.
	
	\noindent\colorbox{yellow}{
		\parbox{\dimexpr\linewidth-2\fboxsep}{Здесь я бы предложил дать определение типам обобщения на семантическом уровне.}
	}
	
	\section{Связывание образа и значения}
		Для формальной записи итерационного процесса связывания образа и значения в алгоритме образования знака из первой части статьи рассмотрим подробнее строение матрицы предсказания процедурного признака. Матрицу предсказания $Z_r^p$ процедурного признака $f_p$ всегда можно представить в следующем виде:
		\begin{equation}
		Z_r^p=(\bar z_1^{r,c},\dots,\bar z_{j_1}^{r,c},\bar z_{j_{1+1}}^{r,e},\dots,\bar z_{i_1}^{r,e},\dots,\dots,\bar z_{i_{k-1}+1}^{r,c},\dots,\bar z_{j_k}^{r,c},\bar z_{j_k+1}^{r,e},\dots,\bar z_{i_k}^{r,e}),
		\end{equation}
		где $\bar z_j^{r,c}$ "--- столбцы причин, $\bar z_i^{r,e}$ "--- столбцы следствий. 
		
		Величину $k$ будем называть актностью процедурного признака. В~дальнейшем будем рассматривать простые матрицы предсказаний $k$-актного процедурного признака:
		\begin{equation}
		Z_r^p=(\bar z_1^{r,c},\bar z_2^{r,e},\dots,\dots,\bar z_{2\cdot k-1}^{r,c},\bar z_{2\cdot k}^{r,e}).
		\end{equation}
		Краткая форма $k$-актного процедурного признака $f_p$ имеет матрицу предсказания, в которой оставлены только первый столбец условий и последний столбец эффектов.
		
		Любой одноактный процедурный признак $f_p$, распознаваемый блоком $R_i^j$, можно представить в виде правила $r_p=(F_C(f_p),F_A(f_p),F_D(f_p))$, в котором:
		\begin{itemize}
			\item $F_C (f_p )\subseteq F_i^j$ "--- множество признаков "--- условий правила: $\forall f\in F_C(f_p)$ $f\sqsubset^c f_p$;
			\item $F_A(f_p)\subseteq F_i^j$ "--- множество добавляемых правилом признаков: $\forall f\in F_A(f_p)$ $f\sqsubset^e f_p,f\notin F_C$;
			\item $F_D(f_p)\subseteq F_i^j$ "--- множество удаляемых правилом признаков: $\forall f\in F_D(f_p)$ $f\notin F_A,f\in F_C$.
		\end{itemize}
		
		Очевидно, выполняются следующие соотношения: $F_A(f_p)\cap F_D(f_p)=\varnothing, F_A(f_p)\cap F_C(f_p)=\varnothing, F_D(f_p)\subseteq F_C(f_p)$.
		
		\begin{Def}
			Процедурный признак $f_p^1$ c матрицей предсказания $Z=(\bar z_1^c,\bar z_2^e)$ выполняется на векторе $z$ длины $q$, если $z\cdot \bar z_1^c=\bar z_1^c$.
		\end{Def}
		Будем говорить, что процедурный признак $f_p^1$ выполним в~условиях процедурного признака $f_p^2$, если 
		\begin{itemize}
			\item оба признака распознаются одним и тем~же распознающим блоком $R_i^j$ и признак  $f_p^1$ выполняется на~столбце условий матрицы предсказания признака $f_p^2$,
			\item $f_p^1\dashv R_1^{j_1}, f_p^2\dashv R_2^{j_2}$, множества $F_C(f_p^1 )$ и $F_C(f_p^2)$ состоят из~одних и тех~же признаков, образуемый вектор $\tilde z$ (той же мощности, что и множество $F_1^{j_1}$) элементы которого, соответствующие признакам из $F_C(f_p^2)$ принимаются равными $1$,  остальные "--- $0$, и признак $f_p^1$ выполним на~векторе $\tilde z$. 
		\end{itemize}
		
		\begin{Def}
			Будем говорить, что два процедурных признака $f_p^1$ и $f_p^2$ конфликтуют, если выполнено как минимум одно из~следующих условий:
			\begin{itemize}
				\item $F_D(f_p^1)\cap F_A(f_p^2)\not=\varnothing$,
				\item $F_D(f_p^2)\cap F_A(f_p^1)\not=\varnothing$,
				\item $F_D(f_p^1)\cap F_C(f_p^2)\not=\varnothing$,
				\item $F_D(f_p^2)\cap F_C(f_p^1)\not=\varnothing$.
			\end{itemize}
		\end{Def}
		
		\begin{Def}
			Результатом операции сохраняющего приведения вектор"--~столбца $\bar z_1$ к~множеству входных признаков $F_{i_2}^{j_2}$ будем называть такой вектор $\bar z_3$ длины $q_{i_2}^{j_2}$, элемент которого $z_{3k}=1$, если $f_{i_1}^{j_1}(k)=f_{i_2}^{j_2}(k)$ и $z_{1k}=1$, иначе $z_{3k}=0$, и обозначать $(\bar z_1\rightarrow F_{i_2}^{j_2})=\bar z_3$.
		\end{Def}
		
		\begin{Def}
			Результатом операции сужающего приведения вектор"--~столбца $\bar z_1$ к~некоторому столбцу $\bar z_2$ распознающего блока $R_{i_2}^{j_2}$ будем называть такой вектор $\bar z_3$ длины $q_{i_2}^{j_2}$, элемент которого $z_{3k}=1$, если $f_{i_1}^{j_1}(k)=f_{i_2}^{j_2}(k)$, $z_{2k}=1$ и $z_{1k}=1$, иначе $z_{3k}=0$, и обозначать $(\bar z_1\Rightarrow \bar z_2)=\bar z_3$.
		\end{Def}
		
		Будем считать, что у субъекта имеется опыт наблюдения, который выражается в виде отношения $\Psi_p^m$: $\tilde p\Psi_p^m \tilde m$, или $\Psi_p^m(\tilde p)=\tilde m$, в том случае, если $\tilde p\in\tilde P$ является перцептом некоторого признака $f$, а $\tilde m\in\tilde M$ "--- функциональным значением того же признака $f$.
		
		Ниже представлен алгоритм доопределения функции $\Psi_p^m$, который и отражает собой суть итерационного процесса во время образования знака согласно алгоритму из первой части статьи. Доопределение проводится на~новую пару $(\tilde p,\tilde m)$, где функциональное значение $\tilde m$ строится в сравнении с эталоном $\tilde m^0$, а перцепт $\tilde p$ формируется на основе подмножества составляющих признаков $\hat F$. Доопределение функции $\Psi_p^m$ означает формирование нового признака $f^*$, т.~е. его первой матрицы предсказания $Z^*$ в~рамках распознающего блока $R^*$.
				
	\begin{algorithm}
		\caption{Алгоритм $\mathfrak{A}_{pm}$}\label{alg:pm}
		\begin{algorithmic}[1]
			\Require $\tilde m^0=\{f_p\}, \Psi_p^m, \hat F\subseteq \{f_k\}$;
			\algrule
			
			\State $\tilde p^{*(0)} := \varnothing$;
			\State $Z^{*(0)} := \varnothing$;
			\State $t := 0$;
			\ForAll{$f^{(t)}\in \hat F$}
			\If{$\exists \tilde m^{(t)}\in \tilde M$ такое, что $(\tilde p(f^{(t)}),\tilde m^{(t)})\in\Psi_p^m$ \textbf{and} $\tilde m^{(t)}$ выполним в условиях признака $f_p$ \textbf{and} $\nexists f: f\in\tilde p^{*(t)},(\tilde p(f),\tilde m(f))\in\Psi_p^m, \tilde m^0$ конфликтует с $\tilde m^{(t)}$}
			\State $\tilde p^{*(t)}=\tilde p^{*(t)}\cup\{f^{(t)}\}$;
			
			\If{$\exists R_i^j$ такой, что $f^{(t)}\in F_i^j$}
			\State $R_i^{j(t)}:=R_i^j$;
			\Else
			\State $R_i^{j(t)}:=\argmax\limits_{\{R\}} (F_i^j\cap\tilde p^{(t)}), F_i^{j(t)}:=F_i^{j(t)}\cup f^{(t)}$;
			\EndIf
			
			\State $\bar z_s:=(z_{s1},z_{s2},\dots,z_{sq}), z_{sk}=1$, если $k$ -- индекс признака $f^{(t)}$ во входном векторе распознающего блока $R_i^{j(t)}$ и $z_{sk}=0$ иначе;
			\State $Z^{*(t)}:=Z^{*(t)}\cup\bar z_s$;
			\State $Z_p^{(t)}:=(\bar z_1^{c(t)},\bar z_2^{e(t)},\dots,\bar z_{2\cdot k-1}^{c(t)},\bar z_{2\cdot k}^{e(t)})$, где $\bar z_i^{c(t)}=\bigvee\limits_{\tilde m_j^{(t)}}(\bar z_j^{c(t)}\rightarrow F_p^j),$ 
			\\\hspace{3.0cm}$\bar z_i^{e(t)}=\bigvee\limits_{\tilde m_j^{(t)}}(\bar z_j^{e(t)}\Rightarrow\bar z_j^e)$;
			\EndIf
			
			\State $\tilde m^{*(t)}=\{f_p^{(t)}\}$;
			\State $\mathcal Z^{*(t)}=\{Z^{*(t)}\}$;
			\State $t=t+1$;
			\EndFor
			
			\Return $\Psi_p^m$, определённая на паре $(\tilde p, \tilde m)$, где $\tilde p=\lim\limits_{t\rightarrow|\hat F|}\tilde p^{*(t)}$, $\tilde m=\lim\limits_{t\rightarrow|\hat F|}\tilde m^{*(t)}$, $f^*, Z^*=\lim\limits_{t\rightarrow|\hat F|}Z^{*(t)},\mathcal Z^*=\{Z^*\}$;
		\end{algorithmic}			
	\end{algorithm}
	
	\begin{Theorem}[о корректности алгоритма $\mathfrak A_{pm}$]
		Алгоритм $\mathfrak A_{pm}$ корректен, т.~е. последовательность функциональных значений $\langle\tilde m^{*(0)},\tilde m^{*(1)},\dots\rangle$, которая строится с помощью алгоритма $\mathfrak A_{pm}$ для функционального значения $\tilde m^0$, сходится к $\tilde m^0$.
	\end{Theorem}
	
	\begin{Proof}
		Рассмотрим два элемента последовательности $\tilde m^{*(t)}=\{f_p^{(t)}\}$ и $\tilde m^{*(t+1)}=\{f_p^{(t+1)}\}$. Соответствующие матрицы предсказания будут иметь следующий вид:
		\begin{eqnarray}
		Z_p^{(t)}=(\bar z_1^{c(t)},\bar z_2^{e(t)},\dots,\dots,\bar z_{2\cdot k-1}^{c(t)},\bar z_{2\cdot k}^{e(t)}),\\
		Z_p^{(t+1)}=(\bar z_1^{c(t+1)},\bar z_2^{e(t+1)},\dots,\dots,\bar z_{2\cdot k-1}^{c(t+1)},\bar z_{2\cdot k}^{e(t+1)}).
		\end{eqnarray}
		Если на шаге 1 и 2 алгоритма $\mathfrak A_{pm}$ на $(t+1)$-й итерации не был найден подходящий признак, то матрицы $Z_p^{(t)}$ и $Z_p^{(t+1)}$ равны. Рассмотрим случай, когда был найден подходящий признак $f^{(t+1)}$ с функциональным значением $\tilde m^{(t+1)}=\{\tilde f_p^{(t+1)}\}$ с соответствующей матрицей предсказания $\tilde Z_p^{(t+1)}=(\bar z^{c(t+1)},\bar z^{e(t+1)})$.
		
		Т.~к. выполнено условие шага 1, то признак $\tilde f_p^{(t+1)}$ выполним на некотором $(2\cdot s-1$-м столбце условий матрицы предсказания признака $f_p$. Это означает, что матрицы $Z_p^{(t)}$ и $Z_p^{(t+1)}$ будут отличать только в двух вектор-столбцах $(2\cdot s-1)$-м и $(2\cdot s)$-м:
		\begin{equation}
		\bar z_{2\cdot s-1}^{c(t+1)}=\bar z_{2\cdot s-1}^{c(t)}\vee (\bar z^{c(t+1)}\rightarrow F_p^j),\bar z_{2\cdot s}^{e(t+1)}=\bar z_{2\cdot s}^{e(t)}\vee(\bar z^{e(t+1)}\Rightarrow \bar z_{2\cdot s}^e).
		\end{equation}
		По определению расстояние между функциональными значениями $\tilde m^{(t)}$ и $\tilde m^0$ примет следующее значение:
		\begin{eqnarray}
		\rho_m(\tilde m^{(t)},\tilde m^0)=\min\limits_{\substack{f_i\in\tilde m^{(t)}\\f_j\in\tilde m^0}}\rho_p(\tilde p(f_i),\tilde p(f_j ))=\rho_p(\tilde p(f_p^{(t)}),\tilde p(f_p))=\nonumber \\
		=\frac{1}{q\cdot h}\sum\limits_{\substack{\bar z_u^1\in Z_p^{(t)}\\\bar z_u^2\in Z_p}}\|\bar z_u^1-\bar z_u^2\|.
		\end{eqnarray}
		Аналогично для $\tilde m^{(t+1)}$:
		\begin{equation}
		\rho_m(\tilde m^{(t+1)},\tilde m^0)=\frac{1}{q\cdot h}\sum_{\substack{\bar z_u^1\in Z_p^{(t+1)}\\\bar z_u^2\in Z_p}}\|\bar z_u^1-\bar z_u^2\|.
		\end{equation}
		Рассмотрим разность 
		\begin{eqnarray}
		\rho_m(\tilde m^{(t)},\tilde m^0)-\rho_m(\tilde m^{(t+1)},\tilde m^0)=\frac{1}{q\cdot h}(\|\bar z_{2\cdot s-1}^{c(t)}-\bar z_{2\cdot s-1}^c\|+\|\bar z_{2\cdot s}^{e(t)}-\bar z_{2\cdot s}^e\|-\nonumber \\
		-\|\bar z_{2\cdot s-1}^{c(t+1)}-\bar z_{2\cdot s-1}^c\|-\|\bar z_{2\cdot s}^{e(t+1)}-\bar z_{2\cdot s}^e\|)=\frac{1}{q\cdot h}(\|\bar z_{2\cdot s-1}^{c(t)}-\bar z_{2\cdot s-1}^c\|+\nonumber \\
		+\|\bar z_{2\cdot s}^{e(t)}-\bar z_{2\cdot s}^e\|-\|\bar z_{2\cdot s-1}^{c(t)}\vee(\bar z^{c(t+1)}\rightarrow F_p^j)-\bar z_{2\cdot s-1}^c\|-\nonumber \\
		-\|\bar z_{2\cdot s}^{e(t)}\vee(\bar z^{e(t+1)}\Rightarrow\bar z_{2\cdot s}^e)-\bar z_{2\cdot s}^e\|),
		\end{eqnarray}
		где $\bar z_{2\cdot s-1}^c,\bar z_{2\cdot s}^e$ "--- столбцы матрицы предсказания процедурного признака $f_p$, соответствующего функциональному значению $\tilde m^0$.
		
		Так как $\tilde f_p^{(t+1)}$ выполним на $(2\cdot s-1)$–м столбце условий матрицы предсказания признака $f_p$, то после применении операции приведения $\bar z^{c(t+1)}\rightarrow F_p^j$ в результирующем векторе единицы появляются только на тех же местах что и в векторе $\bar z_{2\cdot s-1}^c$. 
		
		Это означает, что в векторе $\bar z_{2\cdot s-1}^{c(t)}\vee(\bar z^{c(t+1)}\rightarrow F_p^j)$ по сравнению с вектором $\bar z_{2\cdot s-1}^{c(t)}$  единицы находятся только в тех же местах, что и в векторе $\bar z_{2\cdot s-1}^c$, а новых нулей не появляется. В следствие чего разность $\|\bar z_{2\cdot s-1}^{c(t)}-\bar z_{2\cdot s-1}^c\|-\|\bar z_{2\cdot s-1}^{c(t)}\vee(\bar z^{c(t+1)}\rightarrow F_p^j)-\bar z_{2\cdot s-1}^c\|$ всегда больше нуля.
		
		Так как для столбцов эффектов применяется операция сужающего приведения, которая оставляет единицы только на тех местах, на которых одновременно находятся единицы в приводимом векторе и векторе, к которому осуществляется приведение. В связи с этим разность $\|\bar z_{2\cdot s}^{e(t)}-\bar z_{2\cdot s}^e\|-\|\bar z_{2\cdot s}^{e(t)}\vee(\bar z^{e(t+1)}\Rightarrow\bar z_{2\cdot s}^e)-\bar z_{2\cdot s}^e\|$ также больше нуля.
		
		Так как обе разности в скобках выражения для $\rho_m(\tilde m^{(t)},\tilde m^0)-\rho_m(\tilde m^{(t+1)},\tilde m^0)$ больше нуля, то отсюда следует, что функциональное значение $\tilde m^{(t+1)}$ ближе к $\tilde m^0$. В виду произвольности выбора итерации $t$, это приводит к сходимости всей последовательности $\langle\tilde m^{*(0)},\tilde m^{*(1)},\dots\rangle$. 
	\end{Proof}
	
	\section{Самосознание и его функции} Для алгоритма планирования.
	
	\noindent\colorbox{yellow}{
		\parbox{\dimexpr\linewidth-2\fboxsep}{Здесь я бы предложил дать психологическое описание функций самосознания и определения функций оценки $\Phi_a$ и $\Phi_p$.}
	}
	
	\section{Алгоритм планирования}	Планом $Plan$ будем называть такую последовательность личностных смыслов, в которой действие, представляемое очередным личностным смыслом не конфликтует с предыдущим в цепочке действием.
	
	Целевая ситуация строится исходя из образа процедурного признака, связанного с личностным смыслом, который был определён в процессе целеполагания для целевого знака (см. первую часть статьи).
	
	На странице \pageref{alg:beh_plan} представлен алгоритм планирования поведения.
	\begin{algorithm}
		\caption{Алгоритм $\mathfrak{A}_{bp}$}\label{alg:beh_plan}
		\begin{algorithmic}[1]
				\Require начальная ситуация $F_{sit}$, целевая ситуация $F_{goal}$, функции оценки $\Phi_a$ и $\Phi_p$;
	\Ensure план $Plan$;
	\algrule
	\State $Plan=\Call{Planning}{\varnothing,F_{goal}}$;
	
	\Function{Planning}{$Plan, F_{cur}$}
		\State $\Delta=F_{sit}\setminus F_{cur}$; \Comment{текущая невязка состояний}
		
		\State $F_{for} = \argmin\limits_{F\in 2^{F_{sit}}}|\bigcap\limits_{f_p\in F}F_A(f_p)\setminus\Delta|$; \Comment{находим множество наиболее подходящих параллельных действий}
		\ForAll $f_j\in F_{for}$
			\If{$\exists f_k\in F_{for}$ такой, что $f_k\not =f_i$ и $f_k$ конфликтует с $f_j$}
				\State $F_{for}= F_{for}\setminus\{f_k\}$; \Comment{Удаляем конфликтующие признаки}
			\EndIf
		\EndFor
		
		\State $F_a^{for} = \varnothing$; \Comment{текущее множестов личностных смыслов}
		\ForAll $f_p\in F_{for}$
			\State $F_a^{for} = F_a^{for}\cup \{\Call{Interior}{f_p}\}$;\Comment{интериоризация значения}
		\EndFor
		\State $\tilde F_a^{for}=\Phi_a(F_a^{for},f_{goal})$; \Comment{выбор предпочитаемых действий}
		\If{$\bigcup\limits_{f\in \tilde F_a^{for}}F_C(f)\subseteq F_{sit}$}
			\State \Return $Plan\cup{\tilde F_a^{for}}$;		\Comment{возвращаем обновленный план}
		\Else
			\State $\Delta^* = \Phi_p(\Delta, f_{goal})$; \Comment{Ранжирование критических признаков}
			\State $\tilde F_a^{back} = \varnothing$; 
			\ForAll $f_k\in\Delta^*$ 
				\State $m_k = \tilde m(f_k)$; \Comment{определение значение $k$-го знака}
				\State $F_a^{back} = \varnothing$;
				\ForAll $f_p\in m_k$
					\State $F_a^{back}=F_a^{back}\cup\{\Call{Interior}{f_p}\}$;
				\EndFor 
				\State $\tilde F_a^{back}=\tilde F_a^{back}\cup\Phi_a(F_a^{back}, f_{goal})$; \Comment{выбор предпочитаемых действий}
			\EndFor
			
			\ForAll $f_j\in \tilde F_a^{back}$
				\If{$\exists f_k\in \tilde F_a^{back}$ такой, что $f_k\not =f_i$ и $f_k$ конфликтует с $f_j$}
					\State $\tilde F_a^{back} = \tilde F_a^{back}\setminus\{f_k\}$; \Comment{Удаляем конфликтующие признаки}
				\EndIf
			\EndFor
			
			
			\If{$\Delta\not\subseteq\bigcup\limits_{f\in\tilde F_a^{back}}F_A(f)$}
				\State\Return невозможно построить план;
			\Else
				\State \Return \Call{Planning}{$Plan, \bigcup\limits_{f\in F_a^{back}}F_C(f)$};						
			\EndIf
		\EndIf

	\EndFunction
		\end{algorithmic}
	\end{algorithm}
	
	\section*{Заключение} Подведение общих итогов.
	
	\noindent\colorbox{yellow}{
		\parbox{\dimexpr\linewidth-2\fboxsep}{Про архитектуру агентов и распределение ролей.}
	}
	
	\titleformat{\section}{\normalfont\centering\MakeUppercase}{\thesection.}{1pt}{}[]
	
	%	\nocite{*}
	\inputencoding{cp1251}
	\bibliography{../../biblio/main}
	\inputencoding{utf8}
\end{document}