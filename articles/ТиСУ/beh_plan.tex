\documentclass[a4paper, 12pt]{article}

\usepackage{geometry}
\usepackage{textcomp}					% "true" символы типа copyright

\usepackage{cmap}						% Улучшенный поиск русских слов в полученном pdf-файле
\usepackage[T2A]{fontenc}				% Поддержка русских букв
\usepackage[utf8]{inputenc}				% Кодировка utf8
\usepackage[english, russian]{babel}	% Языки: русский, английский
\usepackage[unicode]{hyperref}			% Русский язык для оглавления pdf
\usepackage{bookmark}					% Оглавление в pdf
\usepackage{soulutf8}					% Для разрядки

\usepackage{amssymb,amsmath,amsthm}
\usepackage{graphicx} 					% Подключаем пакет работы с графикой
\usepackage{algorithm,algpseudocode}

\usepackage{titlesec}					% Форматирование заголовков
%\usepackage{abstract}					% Форматрирование абстракта	

\geometry{a4paper,top=2cm,bottom=2cm,left=2.5cm,right=1cm}	% Геомтерия страницы
\graphicspath{{../../images/}} 			% Пути к изображениям

% Настройка теоремоподобных окружений
\theoremstyle{plain}
\newtheorem{Theorem}{Теорема}
\newtheorem{Lemma}[Theorem]{Лемма}
\newtheorem{Pred}{Утверждение}
\newtheorem{Corollary}{Следствие}
\newtheorem{Def}{Определение}
\newenvironment{Proof}%
	{\par\noindent{\bf Доказательство.}}%
	{\hfill$\scriptstyle\blacksquare$}

\floatname{algorithm}{}

\renewcommand\labelenumi{\theenumi )}	% Нумерованный перечень со скобками
\AtBeginDocument{\renewcommand{\abstractname}{\vspace{-2\baselineskip}}}    		% clear the title
%\renewcommand{\absnamepos}{empty} % originally center

\makeatletter
	\bibliographystyle{gost2008p}
	\renewcommand{\@biblabel}[1]{#1.}	% Заменяем библиографию с квадратных скобок на точку:
\makeatother

\titleformat{\section}[runin]{\normalfont\bfseries}{\thesection.}{1pt}{}[.]
\titleformat{\subsection}[runin]{\normalfont}{\thesubsection.}{1pt}{\so}[.]
%\titleformat{command}[shape]{format}			   {label}		 {sep}{before}[after]

\title{\hbox{\normalsize\textit{УДК 004.81}}\hbox{}\textbf{\Large\MakeUppercase{Управление поведением как функция сознания. II. Семантический уровень и синтез плана}}\footnote{Работа выполнена при поддержке РНФ (грант \No\ 14-11-00692).}}
\author{\textbf{\textcopyright~2015~г. Г.\,С.~Осипов, А.\,И.~Панов, Н.\,В.~Чудова}\\\normalsize\textit{Москва, Институт системного анализа РАН}}
\date{}

\begin{document}
	\vspace*{-5\baselineskip}			% Убираем лишние пробелы перед заголовком статьи
	{\let\newpage\relax\maketitle}
	
	\begin{abstract}
		\noindent Рассматривается семантический уровень описания функций, которые в психологии принято относить к функциям сознания и самосознания. Исследуется механизм работы компонент знака, введённых в первой части статьи. С использованием введённого описания семантического уровня исследуется сходимость основного итерационного процесса образования знака "--- связывания образной компоненты знака и его значения. Введение алгоритмов работы компонент знака позволяет построить алгоритм процесса синтеза плана поведения, а также построить новую архитектуру интеллектуальных агентов, обладающих, в частности, способностями к распределению ролей в коалициях.
	\end{abstract}	
	
	\section*{Введение}
	Связь с первой статьёй \cite{PanovA2014a}. Напомнить про строение знака и нейрофизиологические исследования.
	
	\section{Семантические уровень}
	\subsection{Определение компонент знака}
	\subsection{Семантика обобщения}
	\section{Алгоритм образования знака}
	
	\section{Алгоритм планирования}
	
	\section*{Заключение}
	
	\titleformat{\section}{\normalfont\centering\MakeUppercase}{\thesection.}{1pt}{}[]
	
	%	\nocite{*}
	\inputencoding{cp1251}
	\bibliography{../../biblio/main}
	\inputencoding{utf8}
\end{document}